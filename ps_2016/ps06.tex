\documentclass[11pt]{article}
\usepackage[top=1in, bottom=1in, left=1in, right=1in]{geometry}
\usepackage{amsmath}
\usepackage{amssymb}
\usepackage{graphicx}
\usepackage{titlesec}
\titleformat{\subsection}[runin]{\normalfont\large\bfseries}{\thesubsection}{1em}{}

\def\Te{{T_e}}
\def\Tp{{T_p}}
\def\tep{{t_{ep}}}
\def\trec{{t_{rec}}}
\begin{document}
\pagestyle{empty}
%\parindent=0pt

\section*{\centering Problem Set 6}

\section{Bremsstrahlung Spectrum}

In deriving the spectrum of bremsstrahlung emission, we made an assumption that, in an
individual collision of an electron moving at velocity $v$ with an ion of charge $Ze$
we could relate the impact parameter $b$ to a unique frequency of emission, $\nu$.  The
relationship we derived was $\nu\sim v/4b$.
In this problem, we are going to numerically examine the validity of that assumption.

Starting from what we wrote for the quiz, write a python program that simulates a collision event.  For this simulation,
assume an ion of charge $Ze$ is located at the origin and does not move.  Inject an
electron beginning at something like $\vec x=(-500a_0,100a_0)$ (Bohr radii) moving with velocity
$3\cdot10^7$ cm/s in the $\hat x$ direction.  Simulate a time interval of 0.2 ps, with many thousands
of time steps.  At each time step, use the position of the electron to compute the force
vector acting on the electron.  Assuming that force vector acts for time $\Delta t$ corresponding
to the time resolution of your simulation, compute what the velocity will be in the next time step of
your simulation.  Similarly, use the current velocity to predict the position of the electron in the
next step of your simulation.

\subsection{}

Now plot, as a function of time, the acceleration along the $\hat x$ and $\hat y$ directions.  In the style of
the lecture on thermal bremsstrahlung, estimate the time interval corresponding to an oscillation period in the
EM wave that results from accelerating the electron.  Deduce a frequency of emitted radiation.

\subsection{}
Use the Fourier transform of your acceleration profile versus time to deduce the power spectrum of the radiation.  How
should you combine emission from accelerations in the $\hat x$ and $\hat y$ directions?  At
what frequency does the power spectrum peak?  To what degree is emission concentrated at a single frequency?

\subsection{}
How well does your result motivate the approximation we made in lecture?  For whatever deficiencies you see in our approximation,
discuss how that might affect our estimate of the emissivity $j_\nu$ of free-free emission.

\section{Blowing Str\"omgren Bubbles}\label{p2}

Consider a lone O star emitting $\eta$ Lyman limit photons per second. It sits
inside hydrogen gas of infinite extent and of number density $n$. The star
ionizes an HII region--a.k.a. a ``Str\"omgren sphere," after Bengt Str\"omgren, who
understood that such spheres have sharp boundaries inside of which hydrogen is nearly
completely ionized.

Every Lyman limit photon goes towards ionizing a neutral hydrogen atom. That
is, every photon emitted by the star goes towards maintaining the Str\"omgren
bubble. Put yet another way, no Lyman limit photon emitted by the star travels
past the radius of the Str\"omgren sphere.

The rate at which Lyman limit photons are emitted by the central star equals
the rate of radiative recombinations in the ionized gas. The sphere is nearly
completely ionized\footnote{The
sphere cannot be 100\% ionized because then there would be no neutrals to
absorb the Lyman limit photons that are continuously streaming out of the star.}.
These facts of photo-ionization equilibrium determine the
approximate radius of the Str\"omgren sphere.

The temperature inside the sphere is about 10,000 K. (A class on the interstellar medium can show you why.)

\subsection{}\label{p2partb}

What is the timescale, $\trec$, over which a free proton radiatively recombines in
the sphere? That is, how long would a free proton have to wait before
undergoing a radiative recombination? Give both a symbolic expression, and a
numerical evaluation for $n = 1~{\rm cm}^{-3}$.

\subsection{}

If the star were initially ``off," and the gas surrounding it initially neutral,
what is the timescale for the Str\"omgren sphere to develop after the star were
turned ``on"? That is, how long does the star take to blow an ionized bubble?
Think simply and to order-of-magnitude; you should get the same answer as \ref{p2partb}.

\subsection{}

How thick might the boundary of a Str\"omgren sphere be?

\section{Time to Relax in the Str\"omgren Sphere}

It is often assumed that velocity distributions of particles are Maxwellian.
The validity of this assumption rests on the ability of particles to collide
elastically with one another and share their kinetic energy. For a Maxwellian
to be appropriate, the timescale for a collision must be short compared to
other timescales of interest. This problem tests these assumptions for the case
of the Str\"omgren sphere of nearly completely ionized hydrogen of problem \ref{p2}.

\subsection{}\label{p3parta}

Establishing the electron (kinetic) temperature: what is the timescale, $t_e$, for
free electrons in the Str\"omgren sphere to collide with one another? Consider
collisions occurring at relative velocities typical of those in an electron
gas at temperature $\Te$. Work only to order-of-magnitude and express your answer
in terms of $n$, $\Te$, and other fundamental constants.

\subsection{}\label{p3partb}

Establishing the proton (kinetic) temperature: repeat \ref{p3parta}, but for
protons, and consider collisions at relative velocities typical of those in a
proton gas of temperature $\Tp$. Call the proton relaxation time $t_p$.

\subsection{}\label{p3partc}

Establishing a common (kinetic) temperature: suppose that initially, $\Te>\Tp$.
What is the timescale over which electrons and protons equilibrate to a
common kinetic temperature? This is not merely the timescale for a proton to
collide with an electron. You must consider also the amount of energy exchanged
between an electron and proton during each encounter. Estimate, to
order-of-magnitude, the time it takes a cold proton to acquire the same kinetic
energy as a hot electron. Call this time $\tep$. Again, express your answer
symbolically.

Hint: you might find it helpful to switch the charge on the electron and
consider head-on collisions between the positive electron and positive proton.

\subsection{}\label{p3partd}

Numerically evaluate $t_e/\trec$, $t_p/\trec$, and $\tep/\trec$, for $\Te\sim\Tp\sim 10^4$ K
(but $\Te\ne\Tp$ so that $\tep\ne 0$). Is assuming a Maxwellian distribution of
velocities at a common temperature for both electrons and ions a good
approximation in Str\"omgren spheres?

\end{document}
