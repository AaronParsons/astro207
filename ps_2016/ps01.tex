\documentclass[11pt]{article}
\usepackage{fullpage}
\def\e{{\rm e}}

\begin{document}
\pagestyle{empty}
\parindent=0pt

\section*{\centering Problem Set 1}

\section{Practice with $j_\nu$, $\alpha_\nu$, $S_\nu$, $I_\nu$}

In this problem, we are going to solve the Radiative Transport Equation numerically along the line of
sight through an interstellar dust cloud.  Please generate plots and attach a print-out of your code.
Don't forget important things like axis labels, units, comments, etc.

Suppose we have an interstellar cloud of gas and dust.  Let's assume this
cloud is a uniform, plane-parallel slab 100 pc thick, with a temperature of 50 K.  As
is typical for cool molecular clouds, it is mostly molecular hydrogen (H$_2$), with a number density
of $\sim 10$ cm$^{-3}$. (This density is usually measured using CO emission as a proxy for H$_2$).  
Such clouds in the Milky Way typically have a dust to gas
ratio $\rho_{\rm dust}/\rho_{\rm gas}$ of 0.01.  For this problem, we'll assume that the dust consists
uniformly of spherical grains of radius 0.1 $\mu$m and that the density of material within these
grains is $\sim$3 g/cm$^3$.  (These are all realistic values, by the way.)

\subsection{}

First, calculate the number density of dust grains in this molecular cloud.

\subsection{Dust Extinction}

Next, imagine we have a backlight with a specific intensity of $I_\nu=3\e-32$ ergs/s Hz sr cm$^2$ at $\nu=1$ THz.
For now, let's assume the dust is perfectly absorptive with a geometrical cross-section and 
(unrealistically) does not re-radiate anything.  Plot
the specific intensity you'd measure as a function of distance through the cloud.  
If you need a hint on how to code this, read on; otherwise, have at it!

You may want to make an
array $s$ measuring distance through the cloud and an identically sized array $I$ measuring intensity at each point listed
in $s$.  Start $I[0]$ with the initial intensity.  Then loop through the remainder of $I$, at each point using the increment
$\Delta s=s[i]-s[i-1]$ and the extinction coefficient $\alpha_\nu$ to figure how much intensity was absorbed out of $I$.  Use
this to update $I[i]$, and continue on through the array.

\subsection{Dust Emission}

Now we'll add in the thermal emission of the dust.  To estimate $j_\nu$ for dust, assume that each dust grain
radiates as a blackbody at a temperature of 50 K across its visible surface (i.e. its geometric cross-section).
Just consider emission at 1 THz.  Plot, as a function of distance through the cloud the specific intensity you'd
measure from dust emission alone (no backlight).  Don't forget self-absorption!

\subsection{Extinction and Emission}

Now combine the previous two sections to show the total specific intensity from both the attenuated backlight
and the thermal dust emission.  Construct a source function for the dust ($S_\nu\equiv\frac{j_\nu}{\alpha_\nu}$)
and show how, as a function of optical depth, $I_\nu$ asymptotes to $S_\nu$.


\section{Brightnesses, Magnitudes, and Photons}
\vspace{-6pt}

In the old days, astronomers classified the brightness of celestial
objects using the magnitude system, which roughly corresponded to the
sensitivity of the human eye. Today we continue to use this ancient system
because\dots well, I don't know, but we do. The magnitude of
an object in a given filter $i$ is
\begin{equation}m_i = −2.5 \log_{10}(F_i/F_{0,i}) + m_{0,i}
\end{equation}
where $F_i = \int_0^\infty{F_\nu(\nu)\phi(\nu)~d\nu}$
is the observed flux integrated over a given filter, and
$\phi(\nu)$ is the filter transmission function (a number
between 0 and 1). The magnitude system is calibrated by specifying $F_{0,i}$ and $m_{0,i}$ which
may be done in a number of ways. A common system is 
Vega-magnitudes, in which the star Vega is defined to have $m\approx0$ in all bands.

\subsection{}
\vspace{-6pt}
Grab the filter transmission functions and the spectrum of Vega
(bessel\_V.dat and vega\_spectrum.dat in the github repository, noting that
the Vega spectrum is given as $F_\lambda$, with units ergs s$^{-1}$ cm$^{-2}$ \AA$^{-1}$).
Numerically integrate the apparent V-band magnitude of Vega to determine how many
photons the world's largest ground-based optical telescope (Keck, with a mirror diameter of 10m)
would collect per second from Vega.

\subsection{}
\vspace{-6pt}
Use the fact that Vega is about 8 pc away, and has a diameter of approximately 2.5 $R_\odot$, to
determine the specific intensity of Vega in cgs units, assuming Vega appears as a uniform disk on the sky.

\subsection{}
\vspace{-6pt}
If we were twice as close to Vega (say, 4 pc), figure how many photons per second Keck would receive
in V band, and re-determine the specific intensity of Vega.  Are the scalings of photons per second and
specific intensity with distance consistent with one another?  How so?

\section{Dust Bowl}
\vspace{-6pt}

Dust particles are about 100 microns in diameter.  In the years of the Dust
Bowl in the 1930s, it was not uncommon for dust clouds to completely obscure
the Sun.  According to Wikipedia:
\begin{quote}
On April 14, 1935, known as ``Black Sunday", 20 of the worst ``black
blizzards" occurred across the entire sweep of the Great Plains, from Canada
south to Texas. The dust storms caused extensive damage and turned the day to
night; witnesses reported they could not see five feet in front of them at
certain points.
\end{quote}

\subsection{}
\vspace{-6pt}
Using this observation, estimate the number density of dust particles in the air.
We'll do a more sophisticated
treatment of dust grains later in the class, but for now, assume these dust grains to be perfectly
opaque, and large enough to be treated geometrically (without diffraction, etc.).

\subsection{}
\vspace{-6pt}
If these clouds extended half a mile into the air, how much topsoil was lost to each one of these clouds?

\end{document}
