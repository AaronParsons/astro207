\documentclass[11pt]{article}
\usepackage{fullpage}
\usepackage{graphicx}
\def\e{{\rm e}}

\begin{document}
\pagestyle{empty}
\parindent=0pt

\section*{\centering Problem Set 4}

\section{Comparing CO and H$_2$}

Clouds of molecular hydrogen (as H$_2$ is often called) are where the gas
in our galaxy cools down enough to form stars.  As you might imagine, if your
job is to understand star formation, this is where you'd look.  In contrast
to the hydrogen clouds traced by the 21cm line, these are colder ($T\sim 10$ K)
and more dense ($n\sim100$ cm$^{-3}$).

Cooling H$_2$ turns out to be difficult.  In this problem, we examine why.

\subsection{}
\vspace{-8pt}
Estimate the energies and frequencies of the first H$_2$ rotational and vibrational transitions.

\subsection{}
\vspace{-8pt}
Estimate Einstein A coefficients for these two transitions, using that H$_2$ is a symmetric molecule
(e.g. quadrupole radiator).

\subsection{}
\vspace{-8pt}
Assuming that collisions are setting population statistics for excited/de-excited states to a temperature
of 10 K and assuming these clouds are optically thin, which transition (vibrational or rotational) contributes
the most to cooling this gas?  To order of magnitude long would it take to cool to 5K?

\subsection{}
\vspace{-8pt}
Now suppose there is some CO sprinkled in with the H$_2$.  A typical CO/H$_2$ ratio in the Milky Way
is $6\e-5$.  Using the $J=1\rightarrow0$ transition of CO that we examined in class and assuming
collisions set the population statistics, how does the presence of CO change the cooling time you
derived above?


\section{Good Rovibrations}
\vspace{-8pt}

\begin{figure}
\includegraphics{Vib_rot_CO.png}
\caption{The vibration-rotation line spectrum of CO.}
\label{fig:co}
\end{figure}

Let's see if we can get a simulation to reproduce the rovibrational spectrum of CO pictured in
Figure \ref{fig:co}.
In order to get some of the answers below, it's assumed that you'll be doing some coding to work out
the numerical details.

\subsection{}
\vspace{-8pt}
First, use the center wavenumber to estimate which vibrational transition we are talking about for CO.
Assume we end in the $n=0$ state.

\subsection{}
\vspace{-8pt}
Using that each rotational transition is a $\Delta J=\pm1$ transition and using the spin degeneracy
of each $J$ state ($g=2J+1$), identify the state $J$ at which $E_J\sim kT$.  What does this imply $T$ is?
Don't forget that the total energy is the sum of the vibrational and rotational energies.

\subsection{}
\vspace{-8pt}
Estimate the Einstein A coefficients for this $\Delta n$ transitions as a function of $J$.  Is the
variation sufficient that it needs to be included in predictions of line strength?

\subsection{}
\vspace{-8pt}
Now let's cook up the full simulation.  For this, determine the moment of inertia $I$ for CO as
a function of $n$ (the vibrational level) and $J$ (the rotational level), as we worked out on our
quiz.  Then populate the $n$ and $J$ states according to Boltzmann statistics, using the $T$ you found above.
Finally, use the population of each state and (depending on your answer above) the transition strength
to determine the relative intensity of each transition as a function of frequency.  Plot it and compare to
Figure \ref{fig:co}.

\subsection{}
Are there remaining discrepancies?  What effects haven't you included that would improve your simulation?

\end{document}
