\documentclass[11pt]{article}
\usepackage{fullpage}

\begin{document}
\pagestyle{empty}
\parindent=0pt

\section*{\centering Problem Set 3}


\def\Tex{{T_{\rm ex}}}
\section{Hyperfine Emission from Neutral Hydrogen}
\vspace{-8pt}

Neutral hydrogen in the electronic ground state can be in one of two hyperfine
states. Denote the number density of atoms in the ground hyperfine level
(singlet state) as $n_0$, and the number density of atoms in the excited hyperfine
level (triplet state) as $n_1$. DEFINE the excitation temperature, $\Tex$, of the
transition as
\begin{equation}
\frac{n_1}{n_0}=\frac{g_1}{g_0}e^{-h\nu/k\Tex}.
\end{equation}
Here, $h\nu=hc/\lambda$ is the mean energy difference between the levels, and
$g_0=1$ and $g_1=3$ are the statistical weights of the levels. The excitation
temperature is merely another way of expressing the ratio of ground state to
excited state populations. By definition, if a gas is in local thermodynamic
equilibrium (LTE) at some gas kinetic temperature $T$, then $\Tex = T$; the level
populations are distributed in Boltzmann fashion at the local temperature $T$.
Some people refer to the excitation temperature for the $\lambda = 21$ cm transition as
the {\it spin temperature}. But use of the term ``excitation temperature" is general
to any line transition; it is simply a measure of how excited an atom is.

\def\Ts{{T_*}}
\subsection{}
\vspace{-8pt}
Define $\Ts=h\nu/k$ and compute its value.
It is likely that $\Tex\gg\Ts$. For the remainder of this problem, work in the $\Ts/\Tex\ll1$
limit.

\subsection{}
\vspace{-8pt}
Write down the absorption coefficient, $\alpha_\nu$ (units of per length), for
this transition. Express your answer in terms of $\phi(\nu)$(the line profile
function), $A_{21}$ (Einstein A coefficient), $\lambda$, whatever densities you need, and
$\Ts/\Tex$. Do not forget the correction for stimulated emission.

\subsection{}
\vspace{-8pt}
Write down the volume emissivity, $j_\nu$ (units of erg s$^{-1}$ cm$^{-3}$ Hz$^{-1}$ sr$^{-1}$), for
this transition. Use whatever quantities defined above that you need.

\subsection{}
\vspace{-8pt}
Write down the source function, $S_\nu$, for this transition.

\subsection{}
\vspace{-8pt}
Write down the specific intensity, $I_\nu$, of a cloud of HI that is optically
thin along the line-of-sight (l-o-s). Take the l-o-s dimension of the cloud to
be $L$, and give the answer only to leading order in $\tau\ll1$, where $\tau$ is the
optical depth at an arbitrary wavelength.

Does your answer depend on $\Tex$?  If someone gives you a spectrum of the 21-cm
line that appears in emission and tells you that the line was emitted from an
optically thin cloud, what physical quantities can you infer from the spectrum?

\subsection{}
\vspace{-8pt}
Write down the optical depth of the cloud. Does your answer depend on $\Tex$?

\subsection{} % XXX might want to cover line profiles before doing this one
\vspace{-8pt}
For what column densities is a could of HI marginally optically
thick? Use a gas density of $n=1~{\rm cm}^{-3}$, a gas temperature of $T = 100$ K, and an
excitation temperature $\Tex = T$ . Assume the line is only thermally broadened (i.e.
$\Delta\nu$ is the typical Doppler shift from the thermal motion of atoms, as per the quiz).


\def\He{{^3{\rm He}^+}}
\section{Hyperfine $\He$}
\vspace{-8pt}

Observations of the hyperfine transition in $\He$ are used to probe the $\He/H$ abundance
in the galaxy.  This abundance reflects the primordial yield from big bang nucleosynthesis
and galactic chemical evolution.

\subsection{}
\vspace{-8pt}
Estimate, using the scaling relations presented in class and whatever facts you
remember, the wavelength of the ground-state electronic (i.e. Ly-$\alpha$ equivalent) transition of $\He$.

\subsection{}
\vspace{-8pt}
Estimate, using the scaling relations presented in class and whatever facts you
remember, the wavelength of the ground-state hyperfine transition of $\He$.  Compare
to the true answer of 3.46 cm.

\subsection{}
\vspace{-8pt}
Estimate $A_{21}$ for the ground-state electronic (Ly-$\alpha$-like) and hyperfine (21cm-like) transitions of $\He$.

\section{Rotating Magnetic Dipole}
\vspace{-8pt}

Rybicki \& Lightman Problem 3.1.

A pulsar is convetionally believed to be a rotating neutron star.  Such
a star is likely to have a strong magnetic field, $B_0$, since it traps
lines of force during its collapse.  If the magnetic axis of the neutron
star does not line up with the rotation axis, there will be magnetic dipole
radiation from the time-changing magnetic dipole, $m(t)$.  Assume that the mass
and radius of the neutron star ar $M$ and $R$, respectively; that the angle
between the magnetic and rotation axes is $\alpha$; and that the rotational
angular velocity is $\omega$.

\vspace{-8pt}
\subsection{}
\vspace{-8pt}
Find an expression for the radiated power $P$ in terms of $\omega$,
$R$, $B_0$, and $\alpha$.

\vspace{-8pt}
\subsection{}
\vspace{-8pt}
Assuming that the rotational energy of the pulsar is the ultimate source
of the radiated power, find an expression for the slow-down timescale,
$\tau$, of the pulsar.

\vspace{-8pt}
\subsection{}
\vspace{-8pt}
For $M=1 M_\odot$, $R=10^6$ cm, $B_0=10^{12}$ gauss, $\alpha=90^\circ$,
find $P$ and $\tau$ for $\omega=10^4$ s$^{-1}$, $10^3$ s$^{-1}$,
$10^2$ s$^{-1}$.  The highest rate $\omega=10^4$ s$^{-1}$ is believed
to be typical of newly formed pulsars.

Hint: The Larmor power formula gives the power radiated by an accelerating
electric dipole moment. This problem is similar except that it concerns an
accelerating magnetic dipole moment.

\end{document}
