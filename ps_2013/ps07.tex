\documentclass[11pt]{article}
\usepackage[top=1in, bottom=1in, left=1in, right=1in]{geometry}
\usepackage{amsmath}
\usepackage{amssymb}
\usepackage{titlesec}
\titleformat{\subsection}[runin]{\normalfont\large\bfseries}{\thesubsection}{1em}{}

\begin{document}
\pagestyle{empty}
%\parindent=0pt

\section*{\centering Problem Set 7 (part 1)}

\section{Pulsar Dispersion Measure}

A radio astronomer uses the Green Bank Telescope to observe a pulsar 
in a band from 1 to 2 GHz, recording a frequency power spectrum about every 10 ms.  The resulting
(noisy) frequency spectrum is recorded as pulsar.dat (available on the course GitHub repo).  It looks like 
noise, but
I promise you that, for the right dispersion measure, there is a pulse in there.

\subsection{}
De-disperse the measured power spectra to determine the 
DM of the pulsar.  You may have to try many DMs to find the one that works.
To help you, the beginning of the file is timed to the arrival time without dispersion delay.  For bonus
points, you could make your code not depend on this fact.

Just a warning: be careful about the difference between $\omega$ and $\nu$.  The reported frequencies in
the data file are $\nu$, of course.

\subsection{}
For an assumed density of electrons in the interstellar medium of 0.03 cm$^{-3}$, calculate the 
distance to this pulsar. Does this seem reasonable (what scale would you use to judge ``reasonable")?

\subsection{}
For such an assumed electron density, is the radio astronomer safely observing above the plasma cut-off frequency?

\subsection{}
In an earlier problem set (the Eddington limit), 
we made use of the Thomson cross-section, $\sigma_T=6.65\cdot10^{-25}~{\rm cm}^2$,
for a photon scattering off an electron. We will discuss Thomson scattering in more detail later, but
for now,
calculate the optical depth to Thomson scattering along the line-of-sight to this pulsar.

\section{Faraday Rotation}

Consider the propagation of light thorugh a magnetized plasma.  The magnetic field is uniform: $\vec B_0=B_0\hat z$.
Light travels parallel to $\hat z$.

An electron in the plasma feels a force from the electromagnetic wave and a force from the externally imposed
$B$-field.  Its equation of motion reads
\begin{equation}
m\dot \vec v=-e\vec E-\frac{e}{c}\vec v\times \vec B_0
\end{equation}
where the electric field $\vec E$ can be decomp[osed into right-circularly polarized (RCP) and
left-circularly-polarized (LCP) waves:
\begin{equation}
\vec E=E_0(\hat x\mp i\hat y)e^{i(k_\mp z-\omega t)}
\end{equation}
where it is understood that the real part of $\vec E$ should be taken.  The upper sign ($-$) corresponds
to RCP waves, while the lower sign ($+$) corresponds to LCP waves.

In the equation of motion above, we have neglected the Lorentz force from the wave's
$B$-field, since it is small (by $v/c$) compared to the force from the wave's $E$-field.

\subsection{}
Prove that the solution of the equation of motion reads
\begin{equation}
\vec v=\frac{-ie}{m(\omega\pm\omega_{cyc})}\vec E
\end{equation}
where $\omega_{cyc}\equiv eB_0/mc$.  This is the hardest part of the derivation for the dispersion
relation of RCP and LCP waves.

\subsection{}
Based on Rybicki \& Lightman Problem 8.3.

The signal from a pulsed, polarized source is measured to have an arrival time delay that varies
with frequency as $dt_p/d\omega=1.1\cdot10^{-5}~{\rm s}^2$, and a Faraday rotation that
varies with frequency as $d\Delta\theta/d\omega=1.9\cdot10^{-4}~{\rm s}$.  The measurements are
made around the frequency $\omega=10^8~{\rm s}^{-1}$, and the source is at unknown distance from
the earth.  Find the mean magnetic field, $\langle B_\parallel\rangle$, in the interstellar space
between the earth and the source:
\begin{equation}
\langle B_\parallel\rangle\equiv\frac{\int{n B_\parallel~ds}}{\int{n~ds}}
\end{equation}

\section*{\centering Problem Set 7 (part 2)}

\section{The Orion Nebula}

\subsection{}

Sketch the flux density, $F_\nu$, for the Orion Nebula due to its free-free
emission from a wavelength of $\lambda=100~\mu m$ to $\lambda=100~cm$. Express $F_\nu$ in Jys
($10^{-23}$ is cgs units). Plot the spectrum on a log-log plot, and indicate
power-law indices where appropriate.  

Take the density of electrons to be 
$n_e=2\cdot10^3~{\rm cm}^{-3}$,
the electron and proton temperatures to be $T=T_e=T_p=8\cdot10^3~K$,
the dimension of the ionized cloud to be $R=1pc$, and the distance to the Orion
Nebula to be $d = 500 pc$. Take whatever geometry (cube, sphere) for the cloud is
most convenient.  

{\it Please do not forget free-free self-absorption.}

\subsection{} OPTIONAL 
Overlay on your plot $F_\nu$ from pure electron-electron
scatterings. Here aim only for order-of-magnitude accuracy. Keep in mind that
electron-electron collisions involve time-varying quadrupole moments, not
time-varying dipole moments, and recall the quadrupole vs. dipole scalings from
our discussion of Einstein A's.  

For this part, consider the emissivity ($j_\nu$)
purely from electron-electron scatterings. Drop the emissivity from
electron-proton scatterings. However, for the absorptivity ($\alpha_\nu$), consider the
contributions from both electron-proton and electron-electron scatterings. If
you think one absorption process is more important than the other, justify why
you think that is the case and proceed by including just the one.

\end{document}
