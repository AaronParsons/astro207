\documentclass[11pt]{article}
\usepackage[top=1in, bottom=1in, left=1in, right=1in]{geometry}
\usepackage{amsmath}
\usepackage{amssymb}
\usepackage{titlesec}
\titleformat{\subsection}[runin]{\normalfont\large\bfseries}{\thesubsection}{1em}{}

\def\Te{{T_e}}
\def\Tp{{T_p}}
\def\tep{{t_{ep}}}
\def\trec{{t_{rec}}}
\begin{document}
\pagestyle{empty}
\parindent=0pt

\section*{\centering Problem Set 6 (part 1)}

\section{Saha and the Redshift of Recombination}

Based on Rybicki \& Lightman 9.4.

The thermal de Broglie wavelength of electrons at temperature $T$ is
defined by $\lambda=h/(2\pi mkT)^{1/2}$.  The degree of degeneracy of
the electrions can be measured by the number of electrons in a cube $\lambda$
on a side:
\begin{equation}
\xi\equiv n_e\lambda^3\approx 4.1\cdot10^{-16}~n_e T^{-\frac32}
\end{equation}
For many cases of physical interest, the electrons are very non-degenerate,
with the quantity $\gamma\equiv{\rm ln}~\xi^{-1}$ being of order 10 to 30.
We want to investigate the consequences of the Blotzmann and Saha equations of
$\gamma$ being large and only weakly dependent on temperature.  For the present
purposes, assume that the partition functions are independent of temperature
and of order unity.

\subsection{}
\label{sec:p1a}
Show that the value of temperature at which the stage of ionization passes from
$j$ to $j+1$ (i.e. when the $j+1$th electron is knocked off) is given approximately by
\begin{equation}
kt\sim\frac\chi\gamma
\end{equation}
where $\chi$ is the ionization potential between stages $j$ and $j+1$.  Therefore,
this temperature is much smaller than the ionization potential expressed in
temperature units.

\subsection{}
The rapidity with which the ionization stage changes is measured by the
temperature range $\Delta T$ over which the ration of populations 
$N_j/N_{j+1}$ changes substantially.  Show that
\begin{equation}
\frac{\Delta T}{T}\sim\left[\frac{d{\rm log}(N_{j+1}/N_j)}{d\rm{log} T}\right]^{-1}\sim\gamma^{-1}
\end{equation}
Therefore, $\Delta T$ is much smaller than $T$ itself, and the change occurs rapidly.

\subsection{}
Using the Boltzmann equation and the result from \S\ref{sec:p1a}, show that when $\gamma$
is large, an atom or ion stays mostly in its ground state before being ionized.

\subsection{}
Calculate the redshift, $z_{rec}$, at which recombination occurred 
in the early universe. Use the temperature-redshift relation
\begin{equation}
T_{\rm photon} = T_0(1+z) 
\end{equation}
and the relation between baryonic number density and redshift
\begin{equation}
n=n_0(1 + z)^3.
\end{equation}

Here $T_0 = 2.73K$ is the temperature of the cosmic microwave photon gas today,
and $n_0 = 10^{-7} cm^{-3}$ is the number density of baryons (read: hydrogen atoms)
today, grossly averaged over the entire universe. You may define the epoch of
recombination to be when 0.5 of the protons have recombined to form neutral
hydrogen.  

(Redshift is a cosmologist’s measure of time. A redshift $z = 0$
corresponds to today. As one goes back in time, the redshift increases. Photons
that have travelled from an epoch corresponding to a redshift $z$ have their
original wavelengths, $\lambda$, stretched to a longer wavelength, $\lambda^\prime$, by the expansion
of the universe. By definition, $\lambda^\prime/\lambda=1+z$.)

\section*{\centering Problem Set 6 (part 2)}

\section{Blowing Str\"omgren Bubbles}\label{p2}

Consider a lone O star emitting $\eta$ Lyman limit photons per second. It sits
inside hydrogen gas of infinite extent and of number density n. The star
ionizes an HII region—a.k.a. a ``Str\"omgren sphere," after Bengt Str\"omgren, who
understood that such spheres have sharp boundaries—in which hydrogen is nearly
completely ionized.

Every Lyman limit photon goes towards ionizing a neutral hydrogen atom. That
is, every photon emitted by the star goes towards maintaining the Str\"omgren
bubble. Put yet another way, no Lyman limit photon emitted by the star travels
past the radius of the Str\"omgren sphere.

The rate at which Lyman limit photons are emitted by the central star equals
the rate of radiative recombinations in the ionized gas. The sphere is nearly
completely ionized\footnote{The 
sphere cannot be 100\% ionized because then there would be no neutrals to
absorb the Lyman limit photons that are continuously streaming out of the star.}.
These facts of photo-ionization equilibrium determine the
approximate radius of the Str\"omgren sphere.

The temperature inside the sphere is about 10000 K. (A class on the interstellar medium can show you why.)

\subsection{} OPTIONAL

Derive a symbolic expression for the radius of the Str\"omgren sphere using the
above variables and whatever variables were introduced in lecture.

\subsection{}\label{p2partb}

What is the timescale, $\trec$, over which a free proton radiatively recombines in
the sphere? That is, how long would a free proton have to wait before
undergoing a radiative recombination? Give both a symbolic expression, and a
numerical evaluation for $n = 1~{\rm cm}^{-3}$.

\subsection{}

If the star were initially ``off," and the gas surrounding it initially neutral,
what is the timescale for the Str\"omgren sphere to develop after the star were
turned ``on"? That is, how long does the star take to blow an ionized bubble?
Think simply and to order-of-magnitude; you should get the same answer as \ref{p2partb}.

\section{Time to Relax in the Str\"omgren Sphere}

It is often assumed that velocity distributions of particles are Maxwellian.
The validity of this assumption rests on the ability of particles to collide
elastically with one another and share their kinetic energy. For a Maxwellian
to be appropriate, the timescale for a collision must be short compared to
other timescales of interest. This problem tests these assumptions for the case
of the Str\"omgren sphere of nearly completely ionized hydrogen of problem \ref{p2}.

\subsection{}\label{p3parta}

Establishing the electron (kinetic) temperature: what is the timescale, $t_e$, for
free electrons in the Str\"omgren sphere to collide with one another? Consider
collisions occurring at relative velocities typical of those in an electron
gas at temperature $\Te$. Work only to order-of-magnitude and express your answer
in terms of $n$, $\Te$, and other fundamental constants.

I think it is fair to say that this problem is done most easily in cgs units.

\subsection{}\label{p3partb}

Establishing the proton (kinetic) temperature: repeat \ref{p3parta}, but for
protons, and consider collisions at relative velocities typical of those in a
proton gas of temperature $\Tp$. Call the proton relaxation time tp.

\subsection{}\label{p3partc}

Establishing a common (kinetic) temperature: suppose that initially, $\Te>\Tp$.
What is the timescale over which electrons and protons equilibrate to a
common kinetic temperature? This is not merely the timescale for a proton to
collide with an electron. You must consider also the amount of energy exchanged
between an electron and proton during each encounter. Estimate, to
order-of-magnitude, the time it takes a cold proton to acquire the same kinetic
energy as a hot electron. Call this time $\tep$. Again, express your answer
symbolically.

Hint: you might find it helpful to switch the charge on the electron and
consider head-on collisions between the positive electron and positive proton.

\subsection{}\label{p3partd}

Numerically evaluate $t_e/\trec$, $t_p/\trec$, and $\tep/\trec$, for $\Te\sim\Tp\sim 10^4$ K
(but $\Te\ne\Tp$ so that $\tep\ne 0$). Is assuming a Maxwellian distribution of
velocities at a common temperature for both electrons and ions a good
approximation in Str\"omgren spheres?

\end{document}
