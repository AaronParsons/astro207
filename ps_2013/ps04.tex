\documentclass[11pt]{article}
\usepackage[top=1in, bottom=1in, left=1in, right=1in]{geometry}
\usepackage{amsmath}
\usepackage{titlesec}
\titleformat{\subsection}[runin]{\normalfont\large\bfseries}{\thesubsection}{1em}{}

\begin{document}
\pagestyle{empty}
\parindent=0pt

\section*{\centering Problem Set 4 (part 1)}

\section{Eddington Limit}

Based on Rybicki \& Lightman, Problem 1.4

\subsection{}
Derive the condition for an optically thin cloud to be ejected by radiation
pressure from a luminous object of mass $M$ and luminosity $L$:
\begin{equation}
M/L < \kappa/4\pi Gc
\end{equation}
Remember that $\kappa$ is the mass absorption coefficient (of the cloud).

\subsection{}
Calculate the terminal velocity $v$ of the cloud, based on radiation and
gravity alone, if it starts at rest at a distance $R$ from the object.  Show that
\begin{equation}
v^2=\frac{2GM}{R}\left(\frac{\kappa L}{4\pi GMc}-1\right).
\end{equation}

\subsection{}
One can estimate the (minimum) value of $\kappa$ for hydrogen from
Thomson scattering off free elections when the hydrogen is ionized.
The Thomson cross-section is $\sigma_T=6.6\cdot10^{-25}~{\rm cm}^2$.
The mass scattering coefficient is therefore $>\sigma_T/m_{\rm H}$,
where $m_{\rm H}$ is the mass of a hydrogen atom.  Show that the maximum
luminosity that a central mass $M$ can have and not spontaneously eject
hydrogen by radiation pressure is
\begin{equation}
L_{\rm EDD}=4\pi GMcm_{\rm H}/\sigma_T = 1.25\cdot10^{38}~{\rm erg}~{\rm s}^{-1} (M/M_\odot),
\end{equation}
where $M_\odot$ is the mass of the Sun.  This is called the {\it Eddington Limit}.

\def\Tex{{T_{\rm ex}}}
\section{Hyperfine Emission from Neutral Hydrogen}

Neutral hydrogen in the electronic ground state can be in one of two hyperfine
states. Denote the number density of atoms in the ground hyperfine level
(singlet state) as $n_0$, and the number density of atoms in the excited hyperfine
level (triplet state) as $n_1$. DEFINE the excitation temperature, $\Tex$, of the
transition as
\begin{equation}
\frac{n_1}{n_0}=\frac{g_1}{g_0}e^{-h\nu/k\Tex}.
\end{equation}
Here, $h\nu=hc/\lambda$ is the mean energy difference between the levels, and
$g_0=1$ and $g_1=3$ are the statistical weights of the levels. The excitation
temperature is merely another way of expressing the ratio of ground state to
excited state populations. By definition, if a gas is in local thermodynamic
equilibrium (LTE) at some gas kinetic temperature $T$, then $\Tex = T$; the level
populations are distributed in Boltzmann fashion at the local temperature $T$.
Some people refer to the excitation temperature for the $\lambda = 21$ cm transition as
the {\it spin temperature}. But use of the term ``excitation temperature" is general
to any line transition; it is simply a measure of how excited an atom is.

\def\Ts{{T_*}}
\subsection{}
Define $\Ts=h\nu/k$ and compute its value.
It is likely that $\Tex\gg\Ts$. For the remainder of this problem, work in the $\Ts/\Tex\ll1$
limit.

\subsection{}
Write down the absorption coefficient, $\alpha_\nu$ (units of per length), for
this transition. Express your answer in terms of $\phi(\nu)$(the line profile
function), $A_{21}$ (Einstein A coefficient), $\lambda$, whatever densities you need, and
$\Ts/\Tex$. Do not forget the correction for stimulated emission.

\subsection{}
Write down the volume emissivity, $j_\nu$ (units of erg s$^{-1}$ cm$^{-3}$ Hz$^{-1}$ sr$^{-1}$), for
this transition. Use whatever quantities defined above that you need.

\subsection{}
Write down the source function, $S_\nu$, for this transition.

\subsection{}
Write down the specific intensity, $I_\nu$, of a cloud of HI that is optically
thin along the line-of-sight (l-o-s). Take the l-o-s dimension of the cloud to
be $L$, and give the answer only to leading order in $\tau\ll1$, where $\tau$ is the
optical depth at an arbitrary wavelength.

Does your answer depend on $\Tex$?  If someone gives you a spectrum of the 21-cm
line that appears in emission and tells you that the line was emitted from an
optically thin cloud, what physical quantities can you infer from the spectrum?

\subsection{}
Write down the optical depth of the cloud. Does your answer depend on $\Tex$?

\subsection{} % XXX might want to cover line profiles before doing this one
How large would $L$ have to be for the cloud to be marginally optically
thick? Use a gas density of $n=1~{\rm cm}^{-3}$, a gas temperature of $T = 100$ K, and an
excitation temperature $\Tex = T$ . Assume the line is only thermally broadened.

\def\He{{^3{\rm He}^+}}
\section*{\centering Problem Set 4 (part 2)}

\section{Hyperfine $\He$}
Observations of the hyperfine transition in $\He$ are used to probe the $\He/H$ abundance
in the galaxy.  This abundance reflects the primordial yield from big bang nucleosynthesis
and galactic chemical evolution.

\subsection{}
Estimate, using the scaling relations presented in class and whatever facts you
remember, the wavelength of the ground-state electronic (i.e. Ly-$\alpha$ equivalent) transition of $\He$.

\subsection{}
Estimate, using the scaling relations presented in class and whatever facts you
remember, the wavelength of the ground-state hyperfine transition of $\He$.  Compare
to the true answer of 3.46 cm.

\subsection{}
Here is a way of estimating an Einstein $A_{21}$ coefficient.  $A_{21}^{-1}$ is 
essentially the expected lifetime of an excited state.  A reasonable semi-classical estimate
of this timescale is $t_{\rm life}\sim\frac{E}{P}$, where
$E$ is the energy of the transition, and $P$ is the rate at which
energy is being radiated --- the power.
An accelerated charge radiates power according to Lamor's formula:
\begin{equation}
P=\frac23\frac{e^2\ddot x^2}{c^2}
\end{equation}
where $\ddot x$ is the second derivative of position with time, or acceleration.  Estimate
$A_{21}$ for the ground-state electronic (Ly-$\alpha$-like) transition of $\He$.

\end{document}
