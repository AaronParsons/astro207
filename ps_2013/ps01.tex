\documentclass[11pt]{article}
\usepackage{fullpage}

\begin{document}
\pagestyle{empty}
\parindent=0pt

\section*{\centering Problem Set 1}

\section{Adding waves}
Suppose we have two 1D electromagnetic ``plane'' waves:
\begin{eqnarray}
E_1(x) &= E_1e^{i(k_1x+\omega_1 t)}\nonumber\\
E_2(x) &= E_2e^{i(k_2x+\omega_2 t)}\nonumber
\end{eqnarray}
\subsection{Suppose $E_1=E_2$ and $k_1=k_2$, but $\omega_1\ne\omega_2$.  Show that the energy density
of these waves is, on average, the same if they are superimposed versus evaluated separately.}
\vspace{0.25in}
\subsection{Show the same for plane waves travelling in opposite directions (i.e.
$E_1=E_2$, and $\omega_1=\omega2$, but $k_1=-k_2$).}

\section{Order-of-magnitude electronic transitions}

\subsection{Provide an order-of-magnitude derivation estimating the energy required to ionize hydrogen,
assuming a classical Bohr atom.}
\vspace{0.25in}
\subsection{In what waveband would this transition emit/absorb a photon?}
\vspace{0.25in}

\section{Cyclotron}

Suppose you have a (non-relativistic) electron spinning circles in a magnetic field.

\subsection{Show that the frequency of oscillation does not depend on the velocity of the electron.}
\vspace{0.25in}
\subsection{In what waveband would an electron spiralling in the Earth's magnetic field emit?}
\vspace{0.25in}

\section{Fourier Transforms}

For each of the Fourier transforms that AstroBaki says you should ``just know'':

\subsection{Numerically construct an example waveform and plot it (assume the x axis is time).}
\vspace{0.25in}
\subsection{Plot the Fourier transform of that waveform with correct frequencies identified.}
\vspace{0.25in}
\subsection{Identify the analytic function (with correct numerical coefficients) that corresponds to
the input waveform and its Fourier transform.}
\vspace{0.25in}

\end{document}
