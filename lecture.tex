\documentclass[11pt]{article}
\title{UC Berkeley, Astro 201, Radiative Processes in Astrophysics}
\author{E. Chiang, transcribed by Aaron Parsons}
\date{Fall, 2004}
\usepackage{graphicx}

\def\.{\dot}
\def\^{\hat}
\def\_{\bar}
\def\~{\tilde}
\def\hf{{1 \over 2}}
\def\imply{\Rightarrow}
\def\inv#1{{1 \over #1}}
\def\ddt{{d \over dt}}
\parindent=0mm
\parskip=3pt
\def\sqr#1#2{{\vcenter{\vbox{\hrule height.#2pt
    \hbox{\vrule width.#2pt height#1pt \kern#1pt
        \vrule width.#2pt}
    \hrule height.#2pt}}}}
\def\square{\mathchoice\sqr34\sqr34\sqr{2.1}3\sqr{1.5}3}    
\def\mean#1{\left\langle #1\right\rangle}
\def\sigot{\sigma_{12}}
\def\sigto{\sigma_{21}}
\def\eval#1{\big|_{#1}}
\def\tr{\nabla}
\def\dce{\vec\tr\times\vec E}
\def\dcb{\vec\tr\times\vec B}
\def\wz{\omega_0}
\def\ef{\vec E}
\def\ato{{A_{21}}}
\def\bto{{B_{21}}}
\def\bot{{B_{12}}}
\def\bfield{{\vec B}}
\def\ap{{a^\prime}}
\def\xp{{x^{\prime}}}
\def\yp{{y^{\prime}}}
\def\zp{{z^{\prime}}}
\def\tp{{t^{\prime}}}
\def\upx{{u_x^\prime}}
\def\upy{{u_y^\prime}}
\def\e#1{\cdot10^{#1}}

\usepackage{fullpage}
\usepackage{amsmath}
\usepackage{eufrak}
\begin{document}
\maketitle

\section*{Lecture 1}

\subsection*{Units}

Here are some terms pertaining to telescope observations:\par
aperture area ($\Delta A$), solid angle on sky ($\Delta \Omega 
[\square ",\square^\circ , sr]$), exposure time ($\Delta t$), collects 
energy ($\Delta E$), over waveband ($\Delta \nu [Hz^{-1}]$), but 
$\Delta\lambda\ne {c\over \Delta\nu}$.
$${\Delta E\over\Delta t\Delta\nu\Delta A\Delta\Omega} \iff
{dE \over dt d\nu dA d\Omega} \equiv I_\nu$$
$I_\nu$ is the specific intensity per unit frequency.

Flux density is power per unit frequency passing through a 
differential area whose normal is $\^n$. Thus, flux density is:
$$\boxed{F_\nu\equiv\int I_\nu\cos\theta d\Omega}$$

\subsection*{ Proof that Specific Intensity is conserved along a ray }

The power received by the telescope is:
$$P_{rec}=I_\nu d\Omega dA$$
where
$I_\nu(\alpha,\delta )$ is the intensity as a function of right-ascension
($\alpha$) and declination ($\delta$).  Say that $\Sigma_\nu(\alpha,\delta)$ 
is the surface luminosity of a patch of sky (that is, the emitted intensity).
Then power emitted by patch of sky is:
$$P_{emit}=\Sigma_\nu{dA\over r^2}d\tilde A$$
Recognizing that $d\tilde A=d\Omega r^2$:
$$\Sigma_\nu=I_\nu$$
This derivation assumes that we are in a {\it vacuum} and that the 
{\it frequencies of photons are constant}.
If frequencies change, then though specific intensity $I_\nu$ is not conserved,
${I_\nu\over\nu^3}$ is.
Also, for redshift $z$,
$$I_\nu\propto\nu^3 \propto\inv{(1+z)^3}$$
so intensity decreases with redshift. Finally:
$${I_\nu\over\eta^2}$$
is conserved along a ray, where $\eta$ is the index of refraction.

\subsection*{ The Blackbody }

A blackbody is the simplest source: it absorbs and reemits radiation with
100\% efficiency.  The frequency content of blackbody radiation is given by
the {\it Planck Function}:
\def\ehv{e^{{h\nu \over kT}}}
$$B_\nu={h\nu\over\lambda^2}{2\over(\ehv-1)}$$
$$\boxed{B_\nu={2h\nu^3\over c^2(\ehv-1)}\ne B_\lambda}$$
\centerline{(The Planck Function for Black Body Radiation)}\par

Derivation:\par
The \# density of photons having frequency between $\nu$ and $\nu+d\nu$ has
to equal the \# density of phase-space cells in that region, multiplied by
the occupation \# per cell.  Thus:
$$n_\nu d\nu={4\pi\nu^2d\nu\over c^3}{2\over\ehv-1}$$
However,
$$h\nu{n_\nu c\over 4\pi}=I_\nu=B_\nu$$
so we have it.  In the limit that $h\nu\gg kT$:
$$B_\nu\approx{2h\nu^3\over c^2}e^{-{h\nu\over kT}}$$
\centerline{Wein tail}\par
If $h\nu\ll kT$:
$$B_\nu\approx{2kT\over\lambda^2}$$
\centerline{Rayleigh-Jeans tail}
Note that this tail peaks at $\sim {3kT\over h}$. Also,
$$\nu B_\nu=\lambda B_\lambda$$

\section*{ Lecture 2}

\subsection*{ Reiteration: Conservation of specific intensity }

Conservation of specific intensity told us the intensity collected by your
telescope is absolutely equal to the surface intensity emitted into the angle
leftrightarrowonding to your pixel on the sky. {\it Specific intensity is
distance independent}.
$1 m^2$ of sky emitted into angle $1\square " =$ specific intensity
measured by a $1m^2$ telescope pointed at a $1\square "$ pixel of sky.

\subsection*{ Units}

We have flux, flux density, surface brightness, and spectral energy 
distribution to describe luminosity.  When in doubt, look at the units.
Surface brightness, like specific intensity, is distance independent. It is
simply specific intensity integrated over frequency.

\subsection*{ Lenses}

If you look at something through a lens, the specific intensity of a point
is conserved.  In the lens, the specific intensity may be greater, because
$I_\nu\over\eta^2$ is conserved (remember that $\eta$ is the index of
refraction of the medium), but once the light goes out again, even if the
object looks bigger, the specific intensity is the same.  {\it This does not
violate conservation of energy because the light is being focused to
a smaller area.}\par

Now if something is {\it inside} the lens (say, a bowl of water), then
the lens bends light rays closer to each other, increasing the specific
intensity.  Thus, the object is emitting/scattering higher density photons,
and appears more luminous, and the object also appears larger.  This means
that there must be some places where the object, omitting the lens, would have
been visible, but now isn't.

\subsection*{ Blackbodies}

The CMB is the perfect blackbody because in the past, the surface of the 
black body (the edge of the universe) was not allowed to leak out energy.
This begs the question:
does it matter what the boundary of the black body is made of?
The answer is: No. In the short term, the boundary may imperfectly reflect 
a photon,
but then the resulting change in the temperature of boundary dictates that
it will tend to reemit the energy into the photon gas. Overall:
$$I_\nu I_{emitted}=I_\nu I_{absorbed}=B_\nu f_{abs}$$
However, $f_{emission}B_\nu\equiv I_\nu I_{emitted}$, so:
$$\boxed{f_{emission}=f_{abs}}$$
\centerline{(Kirchoff's Law)}\par
Note that Kirchoff's Law is true regardless of equilibrium.

\subsection*{ Fundamental equation of transfer}

The fundamental equation of transfer is governed by emission and extinction. 
Extinction is brought about by absorption (which changes photon energy) or
by scattering (which does not).  Examples of scattering are Thomson scattering
off of cold electrons, Rayleigh scattering in the atmosphere, and Line
scattering (reemission in a different direction). 
An example of absorption is photoionization (where a photon ionizes an atom, 
say by knocking off an electron).\par

\begin{itemize}
\item  Absorption:\par
Let's say radiation $I_\nu$ passes through a region $ds$ of 
absorption/scattering on its way to us. Then:
$$dI_\nu=-\alpha_\nu I_\nu ds$$
where $\alpha_\nu$ is the {\it extinction} coefficient (units of $cm^{-1}$).
We may compute $\alpha_\nu$ a couple different ways:
$$\alpha_\nu=\overbrace{n}^{\#\ density}
\overbrace{\sigma_\nu}^{cross\ section}
=\overbrace{\rho}^{mass\ density}\overbrace{\kappa_\nu}^{opacity}$$
Solving for intensity:
$$\begin{aligned}I_\nu(s)&=I_\nu(0)e^{-n\sigma_\nu s}\\ 
&=I_\nu(0)e^{-\tau_\nu}\\ \end{aligned}$$
where $\tau_\nu$ is the {\it optical depth} at $\nu$.
Optical depth is often computed as:
$$\tau_\nu=n\sigma_\nu s=N\sigma_\nu$$ 
where $N$, the {\it column density}, is in $cm^{-2}$ and is the \# of 
extinguishers per unit area.
Similarly, 
$$\tau_\nu=\rho\kappa_\nu s=\Sigma\kappa_\nu$$
where $\Sigma$ is the mass surface density.
$$\tau_\nu\begin{cases}\ll 1 &optically\ thin\\
\gg 1 &optically\ thick\end{cases}$$

\def\mfp{\lambda_{mfp,\nu}}
The {\it Mean Free Path} is given by: $\mfp=\alpha_\nu^{-1}=\inv{n\sigma_nu}
=\inv{\rho K_\nu}$. Thus:
$$\tau_\nu={s\over\mfp}$$
That is, the optical depth is the number of mean-free-paths deep a medium is.
For Poisson processes, the probability of absorption is given by:
$$P(n)={e^{-{s\over\mfp}}\left({s\over\mfp}\right)^n\over n!}$$
Therefore:
$$I_\nu(s)=I_\nu(0)e^{-\alpha_\nu s}$$

\item Emission:\par
If $j_\nu $ is the emissivity, then the contribution of the emissivity of
a medium to the flux is:
$$dI_\nu=j_\nu ds$$
\item Emission and Extinction together:
$$\boxed{{dI_\nu\over ds}=j_\nu-\alpha_\nu I_\nu}$$
\centerline{(Fundamental Equation of Transfer)}\par
It is often convenient to express this in terms of optical depth.  
Dividing by $\alpha_\nu$ and recognizing $d\tau_\nu = ds \alpha_\nu$:
$$\begin{aligned}{dI\nu\over d\tau_\nu}&={j_\nu\over\alpha_\nu}-I_\nu\\ 
&=S_\nu-I_\nu\\
\end{aligned}$$
where $S_\nu$ is a ``source function''. In general,
$$S_\nu\eval{scattering}\propto\int{I_\nu d\Omega}$$
There is a formal solution for $I_\nu$.  Let's define 
$\tilde I\equiv Ie^{\tau\nu}$ and $\tilde S\equiv Se^{\tau\nu}$.  Then:
$${d\tilde I\over d\tau_\nu}=\tilde S$$
$$\tilde I(\tau _\nu)=\tilde I(0)+\int_0^{\tau_\nu}{\tilde Sd\tilde\tau_\nu}$$
$$\boxed{I_\nu(\tau_\nu)=\overbrace{I_\nu(0)e^{-\tau_\nu}}^{atten\ bg\ light}
+\overbrace{\int_0^{\tau_\nu}{S_\nu(\tau_\nu^\prime)
\underbrace{e^{-(\tau_\nu-\tau_\nu^\prime)}}_{self-absorption}
d\tau_\nu^\prime}}^{glowing\ medium}}$$
If $S_\nu$ is constant with $\tau_\nu$, then:
$$I_\nu(\tau_\nu)=I_\nu(0)e^{-\tau_\nu}+S_\nu(1-e^{-\tau_\nu})$$
That second term on the righthand side can be approximated as $S_\nu\tau_\nu$
for $\tau_\nu\ll 1$, since self-absorption is negligible.  Similarly, for
$\tau_\nu\gg 1$, it may be approximated as $S_\nu$.
The source function $S_\nu$ is everything.  It has both the absorption and
emission coefficients embedded in it.
\end{itemize}

\def\eboltz{e^{{-h\nu _0 \over kT}}}

\section*{ Lecture 3}

For a sphere to be a black body, it requires that $f_{abs} = f_{emis} = 1$.
However it can still be Planckean if $0<f_{abs}=f_{emis}\le 1$.

\subsection*{ Local Thermodynamic Equilibrium (LTE)}

Local Thermodynamic Equilibrium means 
$$S_\nu=B_\nu(T)$$
where $T$ is {\it local}.
Take a sphere at surface temp $T$, with some gas (matter) inside it.   Allow
to come to thermal equilibrium.  Inside the sphere, 
you see a Planckean spectrum ($I_\nu = B_\nu$).  Now suppose you take a box of
gas (it is both emissive and absorbent) and shoot rays of photons through
that box suffer some absorption: 
$$dI_\nu = -\alpha _\nu ds I_\nu$$
But
those photons also pick up some intensity: 
$$dI_\nu = +j_\nu ds$$
But since the photons
should not pick up energy going through the box (everything is at the same
temperature), 
$$S_\nu\equiv{j_\nu\over\alpha_\nu}=I_\nu=B_\nu$$

Now suppose you let all the photons out of this sphere.  Those photons will
no longer be in thermal equilibrium with the gas.  Suppose we magically fix
the temperature of the gas at this point.  The source function will remain the
same (it will still be the case that $S_\nu = B_\nu$) because the source 
function {\it is a property of the matter alone} (the absorptive and emissive 
properties of it).  However, $I_\nu\ne B_\nu$.\par

Suppose we look along a column of this gas.  At each point, the source
function will be: $S_\nu=B_\nu(T)$.  Therefore, $I_\nu$ along that column
will be: 
\def\np{{\nu^\prime}}
$$I_\nu=\int_0^{T_\nu}{S_\nu(T_\np)e^{-(T_\np-T_\nu)}dT_\np}$$

\subsection*{ Einstein Coefficients}

Derivation identical to Rybicki's.  We should memorize these.\par

These coefficients govern the interaction of radiation with discrete energy
levels.  Say we have 2 energy levels with a difference $\Delta E=h\nu_0$.
There is some uncertainty associated with $\nu$, but we'll say it's small for
now.\par
There are 3 coefficients:\par
\begin{itemize}
\item  $\ato$ governs decay from 2 to 1, and is the transition 
probability per
unit time.  The probability of spontaneous
de-excitation and release of photon is Poisson-distributed with mean rate
$\ato$. So $\ato^{-1}$ is the mean lifetime of the excited state.  e.g.
For $H_\alpha$ (n=3 to n=2): $\ato\approx 10^9 s^{-1}$.

\item  $\bot$ governs absorptions causing transitions $1\to2$.
The transition probability per unit time is $\bot J_\nu$, where $\bot$ is
the probability constant, and $J_\nu$ is:
$$J_\nu \equiv {\int{I_\nu d\Omega} \over 4\pi}$$ 
It depends on $I_\nu$ (the intensity), but it does not depend on 
direction, so we integrate over all angles.  The $4\pi$ is a normalization
constant which makes $J_\nu$ the mean intensity, instead of the total intensity.
However, we have to remember that there are uncertainties in the energy-level
separations.  $\phi(\nu)\equiv$ is called the line profile function.  It 
describes
some (maybe gaussian) distribution of absorption around $\nu_0$ (the
absorption frequency), and is subject to the requirement that:
$$\int_0^\infty{\phi(\nu)d\nu}=1$$
Say that $\Delta\nu$ is the width
of the distribution around $\nu_0$.  $\Delta \nu $ is affected by many factors:
$\ato$ (the natural, uncertainty-based broadening of at atom in isolation),
$\nu_0{V_T\over c}$ (the thermal, Doppler-based broadening), and 
$n_{coll}\sigma_{coll}v_{rel}$
(collisional broadening, a.k.a. pressure broadening).
So really, the transition probability per unit time is:
$$R_{ex}^{-1}=\bot\int_0^\infty{J_\nu\phi(\nu)d\nu}\approx\bot\_J$$

\item  $\bto$ governs stimulated emission.  In this example, we are in energy
state 2, and an incoming photon causes a transition to energy level 1 and the
emission of 2 photons.  The transition per unit time is $\bto\_J$.
\end{itemize}

\subsection*{ Einstein Relations among coefficients}

Assume we have many atoms with 2 energy states, and $n_1$ is the \# density
in state 1, ditto for $n_2$.  Assume we are in thermal, steady-state 
equilibrium, so:
$$n_1\bot\_J=n_2\ato+n_2\bto\_J$$
This is because as many atoms need to be
going from energy state 1 to 2 as visa versa.
A second relation is:
$\_J = {n_2\ato\over n_1\bot-n_2\bto}$.
Using that ${n_2\over n_1}={g_2\over g_1}\eboltz$:
$$\_J={{\ato\over\bto}\over{g_1\bot\over g_2\bto}\eboltz-1}$$

In thermal equilibrium $J_\nu=B_\nu$:
$$\begin{aligned}\_J&\equiv\int_0^\infty{J_\nu\phi(\nu)d\nu}\\ 
&=\int_0^\infty{B_\nu\phi(\nu)d\nu}\\ 
&\approx B_\nu(\nu_0)\\ 
&={2h\nu_0^3\over c^2(\eboltz-1)}\\ \end{aligned}$$

Combining this with $\_J$ earlier, we get:
$$\boxed{g_1\bot=g_2\bto}$$
\centerline{and}
$${\ato\over\bto}={2h\nu^3\over c^2}$$

\subsection*{ Rewriting $j_\nu, \alpha_\nu $ in terms of Einstein coeffs}

In a small volume $dV$:
$$\begin{aligned}j_\nu&\equiv{dE\over dt\,dV\,d\nu\,d\Omega}\\ 
&={h\nu_0\ato n_2\phi(\nu)\over4\pi}\\ \end{aligned}$$

We can express $\alpha_\nu$ in terms of the Einstein coefficients.  The
excitation probability per time is $n_1B_{12}\_J$, and
the energy lost in crossing the small volume 
$\propto n_1\bot{I_\nu d\Omega\over4\pi}\phi(\nu)d\nu$
(it is the probability per time per volume of going $1\to2$ by absorbing 
$I_\nu$ from
a cone of solid angle $d\Omega$ and frequency range $[\nu,\nu+d\nu]$). Thus,
the energy is given by:
$$\begin{aligned}E&=n_1\bot{I_\nu d\Omega\over4\pi}\phi(\nu)d\nu h\nu dt\,dV\\ 
&=\alpha_\nu I_\nu ds\,dt\,d\Omega\,dA\,d\nu\\ \end{aligned}$$ 
Recognizing that $dV=dA\,ds$:
$$\alpha_\nu={n_1\bot\phi(\nu)\over4\pi}h\nu$$
Correcting for stimulated emission, we get:
$$\boxed{\alpha_\nu={(n_1\bot-n_2\bto)\phi(\nu)h\nu\over4\pi}}$$

\section*{ Lecture 4 }

\subsection*{ Estimating Cross-Section}

The absorption coefficient, written in terms of Einstein constants is:
$$\alpha_\nu={n_1\bot\phi(\nu)\over4\pi}h\nu=n_1\sigot$$
Thus, the cross-section of an atom for absorption of a photon is:
$$\sigot={\bot\phi(\nu)h\nu\over4\pi}$$
To estimate $\bot$, we use the fact that, ignoring g's, $\bot\sim\bto$,
and ${\ato\over\bto}={2h\nu^3\over c^2}$.  Then using the approximation that
that $\phi(\nu)\sim\inv{\Delta\nu}$, we get:
$$\sigot\sim{\ato\over\left({2h\nu^3\over c^2}\right)}
\inv{\Delta\nu}$$
$$\boxed{\sigot\sim{\lambda^2\over8\pi}{\ato\over\Delta\nu}}$$
In a single atom, $\Delta\nu\sim\ato$, so $\sigot\sim{\lambda^2\over8\pi}$.

\subsection*{ Order of Magnitude Interaction of Radiation and Matter}

\subsection*{Energy Levels}
\begin{itemize}
\item Electronic Transitions:\par
We'll start with the Bohr atom.  We begin by quantizing angular momentum:
$$m_ev_ea_0=n\hbar$$
If we balance the force required to keep the $e^-$ in a circular orbit
with the electric force:
$${m_ev_e^2\over a_0}={Ze^2\over a_0^2}$$
$$a_0={\hbar^2\over m_ee^2Z}\approx0.52{\AA\over Z}$$
A {\it Rydberg} is the energy required to ionize an H atom from the ground
state.  It is $\sim13.6eV$.  We can estimate it by integrating the electric
force from $r=a_0$ to $r=\infty$, but in reality, there is another factor of
2:
$$\boxed{Rydberg={Ze^2\over2a_0}={Z^2e^4m_e\over2\hbar^2}=13.6\cdot Z^2eV}$$
\item Fine Structure:\par
Fine Structure comes from the interaction of the magnetic moment of the $e^-$
with the a $\bfield$ caused by the
Lorentz-transformed Coulomb field of the proton (generated by the $e^-$'s 
motion).  The 
energy of a dipole interaction is $E=\vec\mu\cdot\bfield$, so we'd expect
that:
$$\Delta E\sim\mu B$$
To estimate B: $B\sim{Ze\over a_0^2}{v\over c}$, and ${v\over c}\sim
{e^2\over\hbar c}=\alpha=\inv{137}$ ($\alpha$ is the fine structure constant):
$$B\sim{Ze^3\over\hbar ca_0^2}$$
Estimating $\mu$: The $\bfield$ of a dipole goes as $B_{di}\sim{\mu\over r^3}$,
so $\mu\sim B_e\eval{r_e}r_e^3$, where $r_e$ is the classical $e^-$ radius.
Estimating that the rest mass energy of the $e^-$ should be about the 
electrostatic potential energy left in the $e^-$, $m_ec^2\sim{e\over r_e}$:
$$r_e\sim{e^2\over m_ec^2}$$
We can estimate $B_e$ by reverting to Maxwell's equation:
$$\dcb={4\pi J\over c}$$
$${B_e\over2\pi r_e}={4\pi\over c}{I\over4\pi r_e^2}$$
Again, we estimate that $I\sim{e\over t_{spin}}$, and because the electron
spin is quantized, $\hbar=m_er_e^2{2\pi\over t_{spin}}$.
After some algebra, we get that:
$$\mu_e={e\hbar\over2m_ec}$$
\centerline{(Bohr magneton for $e^-$)}
$$\mu_p={Ze\hbar\over2m_pc}$$
\centerline{(Bohr magneton for nucleus)}
Getting back to the energy:
$$\begin{aligned}\Delta E&\sim{e\hbar\over2m_ec}{Ze\over a_0^2}\alpha\\ 
&\sim Z^4\alpha^2\cdot Ryd\\ \end{aligned}$$

\item Hyperfine Structure:\par
Instead of using the Bohr magneton, we use the intrinsic magnetic moment
(spin) of the nucleus:
$$B\eval{p}\sim{\mu_p\over a_0^3}\sim B_{fine}{m_e\over m_p}$$
Thus:
$$\Delta E\sim\Delta E_{fine}{m_e\over m_p}$$

Note that Fine and Hyperfine are magnetic dipole transitions.  $e^-$ level
transitions are {\it electric} dipole transitions.  Magnetic dipole transitions
are generally weaker.

\item  Vibrational Transitions in Molecules:\par
Our general technique with vibrational transitions is to model them as
harmonic oscillators.  Thus, they should have the characteristic harmonic
energy series:
$$E_n=(n+\hf)\hbar\wz$$
For a harmonic oscillator, $\wz=\sqrt{k\over m}$.  We estimate that since
the force for a spring is $k\cdot x$, and that force should be about the 
Coulomb force on $e^-$'s.  If we say that atoms stretch with respect to
each other about a Bohr radius:
$$ka_0\sim{e^2\over a_0^2}$$
$$\Delta E\eval{vib\atop trans}\sim Ryd\cdot\sqrt{m_e\over A\cdot m_p}$$
where A is the atomic mass \# of our atoms.

\item  Rotational Transitions in Molecules:\par
The thing to remember is that angular momentum comes in units of $\hbar$.
\end{itemize}

\subsection*{ Damped Simple Harmonic Oscillator}

We can model $e^-$ and their atoms as a lattice of springs connecting $e^-$
to fixed points.  Say that we send a plane wave at these springs that
looks like:
\def\eikrwt{e^{i(\vec k\vec r-wt)}}
$$\ef=E_0\eikrwt$$
We'll say that the displacement of $e^-\ll{2\pi\over k}$.  The equation of
motion for a single electron is then:
$$\ddot x+\gamma\.x+\wz^2x={eE_0\eikrwt\over m_e}$$
where $\gamma$ is our dampening factor.  Note that $\bfield$ is absent 
here; we're neglecting it because it is small for
reasonable energies. It turns out this
equation has the steady-state solution:
$$x={-e\over m_e}{\left((\wz^2-\omega^2)+i\omega\gamma\right)\over
\left((\wz^2-\omega^2)^2+(\omega\gamma)^2\right)}E_0\eikrwt$$
The limiting cases of this equation explain many phenomena.  For example:
$\gamma=0$ (loss-less propagation).\par
In this case, our solution looks like:
$$x={-e\over m_e}{E_0\eikrwt\over(\wz^2-\omega^2)}$$
From this we can use the dispersion relation to relate $\omega$ (frequency) and
$k$ (phase).

\subsection*{ The Dispersion Relation}

Let's solve Maxwell's equations.  Recall that $\vec J$ is the current density:
$$\begin{aligned}\vec J&=n\cdot e\cdot\.x\\ 
&={ne(-i\omega)(-eE_0\eikrwt)\over m_e(\wz^2-\omega^2)}=\sigma\ef\\ \end{aligned}$$
where $\sigma\equiv conductivity={-ne^2i\omega\over m_e(\wz^2-\omega^2)}$, and
n is the \# density of electrons.
So on to Maxwell's equations:
$$\begin{aligned}\dce&=-{1\over c}{\partial\bfield\over\partial t}\\ 
ikE&={i\omega B\over c}\\ \end{aligned}$$
$$\begin{aligned}\dcb&={4\pi\vec J\over c}+{1\over c}{\partial\ef\over\partial t}\\ 
-ikB&=\left({4\pi\sigma\over c}-{i\omega\over c}\right)E\\ \end{aligned}$$
Combining these two equations we get:
$$\boxed{\left({ck\over\omega}\right)^2=1+{4\pi ne^2\over 
m_e(\wz^2-\omega^2)}}$$
This is our dispersion relation.  The plasma frequency is defined as:
$$\omega_p^2\equiv {4\pi ne^2\over m_e}$$
and the index of refraction is:
$$\eta\equiv{ck\over\omega}$$
Rewritten in these terms, our dispersion relation in plasma is:
$$\boxed{\eta^2=1-{\omega_p^2\over\omega^2}}$$

\section*{ Lecture 5 }

\def\lya{Ly\alpha}
\subsection*{ Einstein A's for $\lya$}

$\ato$ is a measure of the probability of decay per unit time, so
$\ato^{-1}\sim lifetime$.  This should be about
equal to the energy of the electron state divided by the average power radiated
by an electron being accelerated:
$$\ato^{-1}\sim{E\over P}\sim{\hbar\omega_0\over{2\over3}{e^2\ddot x^2
\over c^3}}
\sim{3\hbar\omega_0c^3\over2(e\ddot x)^2}$$
Now $e\cdot\vec x=\vec d$ (the electric dipole moment) and
$\ddot x\sim\omega_0^2x$ for a spring, so:
$$\ato^{-1}\sim{3\hbar\wz c^3\over2d^2\wz^4}$$
$$\boxed{\ato\sim{2d^2\wz^3\over3\hbar c^3}}$$
For H: $d\sim ea_0$ and $\lambda_\lya=1216\AA$ so:
$$\ato\sim5\e8s^{-1}$$

\subsection*{ Magnetic Dipole for $\lya$}

The magnetic dipole of an electron is:
$$\mu_e={e\hbar\over m_ec}$$
Thus we can estimate the ratio of $\ato$ for magnetic dipole transitions to
that of electric dipole transitions:
$${\ato\eval{mag}\over\ato\eval{elec}}\sim\left({\mu_e\over d}\right)^2
\sim\left({e^2\over\hbar c}\right)^2\sim\alpha^2$$
This tells us that the magnetic dipole states (that is, fine and hyperfine
states) are longer lived than electric dipole states by a factor of $\alpha^2$.
$${\ato\eval{21cm}\over\ato\eval{Ly\alpha}}\sim\alpha^2
\left({1216\AA\over21cm}\right)^3$$
$$\ato\eval{21cm}\sim6\e{-15}s^{-1}$$
The actual value is $2.876\e{-15}s^{-1}$.

\def\mfe{\mathfrak{E}}
\subsection*{ Electric Quadrupole}

If one is nearby a rotating quadrupole, one sees the $\mfe$  (electric)
field rotating rigidly.
However, from far away, there are kinks in the field, resulting in a retarded
potential.  The radiation nearby goes as $r_{near}\sim\lambda$.  For a monopole,
the electric field is $\mfe\sim{q\over r^2}$.  For a dipole, it is 
$\mfe={q\over r^2}
{s\over r}$, where s is the charge separation.  For a quadrupole: 
$$\mfe={q\over r^2}\left({s\over r}\right)^2$$
Since $P\propto \mfe^2$, the ratio of the powers emitted by a
quadrupole vs. a dipole should be:
$${P_{quad}\over P_{di}}\sim\left({s\over r}\right)^2
\sim\left({s\over\lambda}\right)^2$$
An acoustic analogy: a kettle whistle is a monopole, a
guitar string is a dipole, and a tuning fork (with its two out-of-phase
prongs) is a quadrupole.\\

Anyway, since $\ato\sim{P\over E}$,
$$\boxed{{\ato\eval{quad}\over\ato\eval{di}}
\sim\left({s\over\lambda}\right)^2}$$
Thus $28\mu m$, the lowest quadrupole rotational transition of
$H_2$, should have an $\ato$ of about:
$$\ato\eval{28\mu m}\sim\ato\eval\lya\left({s\over\lambda_{H_2}}\right)^2
\left({\lambda_\lya\over \lambda_{H_2}}\right)
\sim\ato\eval\lya\left({a_0\over 28\mu m}\right)^2
\left({1216\AA\over 28\mu m}\right)^3
\sim7\e{-11}s^{-1}$$
The actual value is $3\e{-11}s^{-1}$.

\subsection*{ Radio Recombination Lines}

In HI, the $n=110\to n=109$ transition has a wavelength of 6 cm.  We can
estimate its $\ato$:
$$\ato\eval{6cm}\sim\ato\eval{Ly\alpha}
\underbrace{\left({1216\AA\over6cm}\right)^3}_{change\atop in\ \lambda}
\underbrace{\left({a_{110}\over a_0}\right)^2}_{change\atop in\ atom\ size}$$
\def\bra#1{\langle #1|}
\def\ket#1{|#1\rangle}
This presents the question of which dipole moment to use.  It turns out
we must use $\bra{i}\vec k\cdot\vec r\ket{f}$.

\subsection*{ Back to $\sigma$}

$$\sigot\eval{line\atop center}\sim{\lambda^2\over8\pi}{\ato\over\Delta\nu}$$
Now $\Delta\nu\sim\nu$ for doppler broadening, and $\ato\sim\nu^3$, so for
electric and magnetic dipole transitions:
$$\sigot\sim\lambda^0$$
So the cross-section for these transitions does not depend on wavelength.

\subsection*{ Interaction of Radiation with Grains}

In general, we will be talking about a plane wave of wavelength $\lambda$
which is incident upon a particle (grain) of radius a.  The cross-section
for absorption, scattering, and emission will all be proportional to the
physical cross-section of the grain, with some {\it absorption efficiency}
Q:
$$\sigma_i=Q_i\pi a^2$$
Kirchkoff's Law requires that $Q_{emit}=Q_{abs}$.  If $a\gg\lambda$, then
\def\qscat{Q_{scat}}
\def\qabs{Q_{abs}}
we have the geometric optics limit, and $\qscat+\qabs\sim$.  In fact,
Babinet's Principle says:
$$\boxed{\qscat+\qabs=2}$$
The proof of this goes as follows: suppose we have an infinite plane wave
focused by an infinite lens onto a point on a wall.  The power pattern
should be a delta function at that point on the wall.  Now suppose you
place a screen with an aperture of diameter $a$ between the plane wave
and the lens.  We should now see an interference pattern on the wall.  Call
the power incident on a point on the wall $P_1$.  Now put in a new aperture
which is the exact compliment of our previous one (it is a scattering
body of diameter $a$), we should see a new power at our point on the wall: 
$P_2$.
However, the sum of waves incident on the wall under these two apertures
should be the same as the wave from the sum of the apertures, which was
a delta function.  Thus:
$$P_1=P_2$$
The first aperture (the slit) represented the scattering power from 
diffraction alone $(\qscat=1)$, and the second represented the absorbed 
power.  The sum of these two powers, for everywhere but the true focus
point, is actually twice the incident power, and since $P_1=P_2$, we
get:
$$\qscat+\qabs=2$$
Note that if $\qscat>1$, then the object in ``shiny'': more light gets
diffracted and less gets absorbed.  Thus, although $\qscat$ changes,
the sum remains the same.

\section*{ Lecture 6 }

\def\eb{e^{h\nu \over kT}}

\subsection*{ Flat Blackbody disk:}

$$\begin{aligned}\nu F_\nu&\propto\nu^4\int_{r_i}^{r_0}{2\pi r\,dr\over\eb-1}\\ 
&\propto \nu^3\int_{r_i}^{r_0}{r\,dr\over 
e^{h\nu\over kT}({r\over R_*})^{3\over 4}-1}\\ \end{aligned}$$
Define $x \equiv {h\nu\over kT_*}({r\over R_*})^{3/4}$. Then:
\def\hnkt{{h\nu \over kT_*}}
\def\tq{{3\over 4}}
$$\nu F_\nu \propto \nu^4\int{{\hnkt ({r_i\over R_*})^\tq}^{\hnkt ({r_o\over 
R_*})^\tq}{x^3dx\over e^x - 1}}$$
We are free to choose $\nu$, so we'll choose it such that:
$$\nu_0\sim{kT_*\over h}({R_*\over r_0})^\tq\ll\nu\ll{kT_*\over h}
({R_*\over r_i})^\tq\sim\nu_i$$
Then the integral simplifies to:
$$\nu F_\nu \propto \nu^{4\over 3}\int_{x\ll 1}^{x\gg 1}{x^3dx\over e^x-1}$$
This is, on the whole, insensitive to exact values on the bounds of the
integral, so:
$$\nu F_\nu \propto \nu^{4\over 3}$$
Thus, if we are plotting $\nu F_\nu$ vs. $\nu$ (like on the homework), we
see a flattening of the spectrum for the blackbody disk because of the region
where $\nu F_\nu \propto \nu^{4\over 3}$.  

\subsection*{ Plane waves through a lens onto a backdrop}

We are considering a case of sending an infinite plane wave through an
infinite lens with focal length $f$ onto a backdrop which is distance $f$
away.  Next, we consider what happens if we have: (a) an occulting object
at the center of the lens, (b) the exact opposite of that object--an
aperture at that point.  We'll get diffraction patterns on our backdrop in
both of these cases.  What is interesting is that the sum of the apertures
(in our case, an infinite aperture minus a spot plus that spot) should also
sum the diffraction patterns.  The diffraction pattern for an infinite, 
unblocked aperture is just a delta function (everything focused at the 
focal point).  Thus the fringe patterns for each of the partially blocked
apertures must be the same, but $180^\circ$ out of phase.  Of course,
the (infinite aperture minus a spot) contains a delta function in addition
to its diffraction pattern.\par

\subsection*{ Small Grains}

Now let's consider the case that this spot (a small object obstructing a 
plane wave) is small, so that $a \ll \lambda$.  This is called the ``Rayleigh
Limit''.  In this case, $Q_{scat} \ll 1$ and $Q_{em} = Q_{abs} \ll 1$.  If
\def\qabs{Q_{abs}}
\def\qscat{Q_{scat}}
\def\qemis{Q_{emis}}
we were to graph $\qabs$ vs. $\lambda$, we'd get something flat ($= 1$) out to
$2\pi a$, after which, $\qabs$ would decrease as $(\inv{\lambda})^\beta$, where
$\beta\eval{abs} = [1-3]$.  On the other hand, graphing $\qscat$ vs. $\lambda$
, we get that after $2\pi a$, $\qscat$ drops as $(\inv{\lambda})^4$. 
Why does $Q$ decrease as $a\over \lambda$?\par

\begin{itemize}
\item  Let's consider the case of emitting with a dipole antenna.  For a 
region close to the antenna (the near zone, of order $\lambda$), 
$P_{rad}\propto E^2$, and at the edge of the near zone, 
$E\propto{a\over\lambda}$.  Since $P_{rad}\propto\qemis\propto\qabs$,
$$\qabs\propto\left({a\over\lambda}\right)^2$$

\item  Here's another crude way of looking at it: 
if you're in a boat that's small compared
to the size of a wave coming at it, you aren't going to do anything to that
wave (you won't scatter, everything transmits).

\item  Another way of looking at it: 
the damped simple harmonic oscillator.  Recall that:
$$\sigma_{abs} = {4\pi e^2\over m_e c}{w\gamma \over \left[(w_0^2-w^2)^2 +
(w\gamma)^2\right]}$$
If $w \ll w_0$, then:
$$\sigma_{abs} \to {4\pi e^2\over m_e c}{w^2\gamma \over w_0^4} \propto w^2$$
Thus, $\qabs \propto \inv{\lambda^2}$.
\end{itemize}

Finally, there is Mie Theory, which states that if $\eta = n +ik$, where
$\eta$ is the complex index of refraction, then you can express $\qabs$ and
$\qscat$ in terms of the size parameter $x \equiv {2\pi a\over \lambda}$.
For this derivation that Mie did, $n, k$ are called ``optical constants''.  This
is a bad name, because both of these are actually functions of $\lambda$.\par

Now we look at the handout Eugene gave us (Chiang et al, 2001, ApJ, 547, 1077).
Notice on the $\qabs$ vs. $\lambda$ plots, in addition to the decay we
described, there are ``bumps and wiggles''.  These are characteristic of the
resonances of the systems (e.g. they leftrightarrowond to a rotational mode, etc.).
The varying lines on each graph are for different sized particles.  Notice the
characteristic Rayleigh Limit for small particles. \par

\def\qext{Q_{ext}}
There's another graph about how extinction $\qext = \qabs + \qscat$ goes as
the size parameter is changed.  There is a large ``hump'' where $\qext$ goes to
about 4.  It can do this (we'd said it couldn't go above 2) because we are not
yet in the geometric optic limit.  Notice that on this big hump, there is a
region where increased wavelength causes increased extinction.  This is the
process which causes ``blue moons''.  If the size of particles in the atmosphere
is just right, it will pass blue wavelengths while extinguishing red.\par

\subsection*{ Some special results}

\def\sabs{\sigma_{abs}}
\begin{itemize}
\item  Crystalline dielectrics: $\qabs \propto {a\over \lambda^2}$ (for
$\lambda \gg a$.  Then:
$$\alpha_\nu \eval{abs} = n_d\sabs = n_d\pi a^2\qabs
\propto {a^2\over \lambda^2}$$
Now what we measure in the field is $\tau$, the
optical thickness.  However, $\tau_\nu = \sum{K_\nu}$, where $K_\nu = 
{\sabs \over m_{particle}} \propto a^0$.  This is how the mass of 
particles can measured. Then $\qscat \sim ({a\over \lambda})^4$, like
Rayleigh scattering.
\item  In general: $j_\nu\eval{emission} = \alpha_\nu \cdot B_\nu(T)$.
Then:
$$j_\nu\eval{scattering}(\^n) = n_d(\qscat\pi a^2)\int{I_\nu(\^n)
F(\^n-\^n^\prime)d\Omega^\prime)}$$
The term $F(\^n - \^n^\prime)$ is called the ``scattering phase function'', and
is used to add up all of the incoming paths of light, taking into account their
relative phases.  $n_d$ is the \# density of scattering grains.  For small
enough grains ($a\ll \lambda$), F is independent of the properties of the
material the grains are made of:
$$F = {1+\cos^2\theta\over 2}$$
where $\theta$ is the angle from the propagation direction of the incident beam.
This gives us a characteristic ``dog bone'' scattering pattern. Small grains
are (to within a factor of 2) isotropic scatterers.  The factor of 2 is the 
dog bone pattern.  Note that the ``missing half'' at right angles to the
propagation direction is the portion of incident light which had no polarization
component which aligns with the characteristic polarization which must exist
for light detected at a right angle from a scattering body.
\end{itemize}

\section*{ Lecture 7 }

\subsection*{ Increasing Grain Size}

We return to the model in which an infinite plane wave passes through an
aperture, is focused by an infinite lens, and shines on a wall.  In this
model, as the slit aperture widens (a increases), then the diffraction
pattern narrows.  Thus, for larger grain sizes, there is more forward
scattering than in other directions.  The fitting formula for the power
pattern for large grains is:
$$F(\theta)={1-g^2\over1+g^2-2g\cos\theta}$$
where $g\equiv\mean{\cos\theta}={\int{F\cos\theta d\Omega}\over
\int{Fd\Omega}}$. Note that this formula fails for $\theta=100^\circ$.  If
$g=1$, then we have isotropic scattering, and as $g\to1$, $F(\theta)$ peaks
increasingly in the $\theta=0$ direction (it is increasingly 
``forward throwing'').

\subsection*{ Forward Scattering}

Forward scattering can actually increase the intensity of light in some areas
over if there were no scattering at all.
If particles are smaller that the wavelength of light, there is isotropic
scattering.  For larger particles, the power is more concentrated in the 
perfectly forward and backward directions.  This is why, where there is fog,
you dim your headlights.  The larger scattering particles in the fog actually
increases the intensity of light reflected back at you and at oncoming cars.

\subsection*{ Collisional Excitation Cross-sections}

The Einstein analog:
$$\overbrace{A}^{low\ E\ particle} + \overbrace{B_{fast}}^{high\ E\ e^-}
\to \overbrace{A^*}^{excited} + B_{slow}$$
So the Rate of Excitations $R_{ex}$ is given by:

\def\rex{R_{ex}}
\def\vrel{v_{rel}}
$$\rex = n_An_B\sigma_{12}f(\vrel)\vrel$$
Suppose we have some distribution of relative velocities given by $f(v)dv$,
where $f$ is the fraction of collisions occurring with relative velocities
$[\vrel, \vrel +dv]$.  Then:
$$\rex = n_An_B\int_0^{\infty}f(v)dv\sigot(v)v$$
$$=n_An_B \mean{\sigot v}$$
\def\qot{q_{12}}
\def\qto{q_{21}}
$$=n_An_B\qot$$
where $\qot$ is the ``collisional rate coefficient'' $[cm^3s^{-1}]$.  Then the
Rate of de-excitation is given by:
$$R_{deex} = n_An_B\int_0^\infty{f(v)v\,dv\sigma_{21}(v) = n_A^*n_B\qto}$$
We recognize now that $\sigot(v)n_An_Bf(v)dv\,v$ is the rate of excitations of 
A using B moving at relative velocity $v$.  If we have detailed balance,
then this has to be the same as the rate of de-excitation
$n_A^*n_Bf(v^\prime)v^\prime\sigto(v^\prime)$.
$$\overbrace{\hf m_rv^2}^{center\ of\ mass\ E} = hv_{12} + \hf m_rv^{\prime2}$$
Where $m_r$ is the reduced mass $m_Am_B\over m_A+m_B$. However many $B_{slow}$
are created by collisional excitation, the same number are used for the
reverse de-excitation.  This is {\bf detailed balance}.\par
Second, under thermal equilibrium, particles have a {\bf Maxwellian} velocity
distribution:
$$\boxed{f(v)=4\pi\left({m_r\over 2\pi kT}\right)^{3\over 2}v^2
e^{-m_rv^2\over 2kT}}$$
\centerline{(Maxwellian velocity distribution)}
In thermal equilibrium,
$${n_A^*\over n_A} = {g_2\over g_1}e^{-h\nu_{21}\over kT}$$
Now, assuming detailed balance and thermal equilibrium,
$$n_An_Bf(v)dv\,v\sigot=n_A^*n_Bf(v^\prime)dv^\prime v^\prime \sigto$$
\def\emvkt{e^{-m_rv^2\over2kT}}
\def\emvpkt{e^{-m_rv^{\prime2}\over2kT}}
\def\ehvkt{e^{-h\nu_{21}\over2kT}}
$$\sigot v^2\emvkt = {g_2\over t_1}\ehvkt \nu^{\prime2}\emvpkt v^\prime 
dv^\prime\sigto$$
$$\sigot dv\,v^3 \emvkt = {g_2\over t_1}\ehvkt \ehvkt \emvkt v^{\prime3}
dv^\prime \sigto$$
$$\boxed{g_1v^2\sigot(v) = g_2v^{\prime2}\sigto(v^\prime)}$$
This is the ``Einstein analog''.\par
For a specific case, $B=e^-$, $A=$ion with bound electron.
\begin{itemize}
\item Incident electron has kinetic energy $>h\nu_{21}$.
$$\hf m_rv^2 \approx \hf m_ev^2 > h\nu_{21}$$
\item  Coulomb focusing gives $\inv{v^2}$ cross-section.
\end{itemize}
We want to know how far away an electron with $v$ can be aimed
and still hit the $a_0$ radius cloud around the ion.  This is $b$, the
{\bf impact parameter}.  Our collision cross-section $=\pi b^2$.  Our
angular momentum is conserved, so 
$$m_ev\,b=m_ev_fa_0$$
We know that $v_f^2 = v^2 + v_\perp^2$, where $v_\perp$ is the velocity
$\perp$ to the original electron velocity.  This is a result of it falling
toward the ion.  Then:
$$\hf m_ev_\perp^2 \sim {Ze^2\over a_0}$$
$$v_f^2 = v^2 + {Ze^2\over m_ea_0}$$
$$b={a_0v_f\over v}$$
$$\pi b^2 = {\pi a_0^2\over v^2}\left[v^2+{Ze^2\over m_ea_0}\right]$$
$$= \pi a_0^2\left[1+\underbrace{Ze^2\over m_ev^2a_0}_{Coulomb\ focusing\atop 
factor}\right]$$
Generally, the Coulomb focusing factor $>1$ because we want to excite, not
ionize.  $a_0={\hbar^2\over Ze^2m_e}$, so:
$$\pi b^2 = {\pi\hbar^2\over m_ev^2}$$
$$\sigot = {\pi\hbar^2\over m_e^2v^2}
\overbrace{\left(\Omega(1,2)\over 
g_1\right)}^{quantum\ mechanical\atop correction\ factor}$$
$\Omega$ is the ``collisional strength'', and generally is 0 below the
$v$ threshold, goes to 1 at the threshold, and decreases for increasing
$v$, with some occasional spikes.  Generally, it is of order 1, with some
slight temperature dependency.
$$\qot=\mean{\sigot v} \propto \mean{\inv{v}} \propto \inv{\sqrt{T}}$$
\begin{itemize}
\item 2000 K gas.  $v_{term} \sim \sqrt{\gamma kT\over m}$, so 
$v\sim \sqrt{2000\over 100}42\cdot 1{km\over s} \approx 160{km\over s}$. Then
$$\sigot \sim 10^{-14}cm^2\left({\Omega(1,2)\over g_1}\right)$$
$$\sigot\eval{osterbrock} \sim 10^{-15} cm^2$$
\end{itemize}

\section*{ Lecture 8 }

Forgot something for the Einstein Analog.  Recall:
$$g_1v^2\sigma_{12}(v)=g_2v^{\prime2}\sigma_{21}(v^\prime)$$
$$\hf m_4v^2=h\nu_{21}+\hf m_r(v^\prime)^2$$
For the special case of $f(v)$ being Maxwellian, then:
$${q_{12}\over q_{21}}\equiv{\mean{\sigma_{12}v}\over\mean{\sigma_{21}v}}
={g_2\over g_1}e^{-h\nu_{21}\over kT}$$
This has a Boltzmann factor, which makes you thing we're assuming LTE, but
we're not. \par
Last time, we were talking about electron-ion collisional excitation, given by:
$$\qot\equiv\mean{\sigot v}\propto\inv{v}\propto T^{-\hf}$$
We can extend this for neutral-ion collisional excitation:
$$\qot \propto T^\hf T^{-\hf} \propto T^0$$
$$\sigot\equiv{\pi\hbar^2\over m_ev^2}{\Omega(1,2)\over g}$$
where $\Omega(1,2)$ scales as $T^\hf$.  Notice this means that for some
neutral-ion collisional excitation, it is {\it temperature independent}.
For neutral-neutral collisional excitation:
$$\qot=\mean{\sigot v}\propto T^\hf$$
$$\sigot\propto T^0\sim \pi(Fa_0)^2$$
This is all we'll talk about bound-bound transitions.

\subsection*{ Bound-Free Transitions (Photoionization)}

\def\sigbf{\sigma_{bf}}
We'll calculate the cross-section of a bound-free transition $\sigbf$:
$$\sigbf\sim{\lambda^2\over 8\pi}{A_{21}\over \Delta\nu}$$
It turns out that $\Delta\nu$ is about $\nu$. $\lambda\approx 912\AA$, 
so scaling from Lyman-alpha:
$$\sigbf\sim{(912\AA)^3\over c\cdot8\pi}A_{21,Ly\alpha}
\left({1216\AA\over912\AA}\right)^3\sim 10^{-18}cm^2$$
It turns out that the real answer is $\sigbf\sim 6\cdot 10^{-18}cm^2$.  In
general:
$$\sigbf=\sigma\eval{edge}\left({E_{photon\,in}\over E_{edge}}\right)^{-3}$$
That exponent (-3) is actually $-\frac83$ near the edge and goes to $-\frac72$
far from it.  So you see $\sigbf$ spike up as the photon reaches the 
ionization energy, and then decrease exponentially as energy increases.
However, you can see new spikes from ionizing electrons in inner shells.

\subsection*{ Radiative Recombination}

\def\sigfb{\sigma_{fb}}
This is the inverse process of photoionization, so $\sigfb$ is the cross-section
for an ion recapturing its electron and emitting a photon.  We'll relate
$\sigfb$ to $\sigbf$.  This is called the Milne Relation.  In this derivation,
we'll start by assuming complete thermal equilibrium and derive a result which
will end up being independent of thermal equilibrium.  Let's start calculating
the rate of radiative recombinations.  Thermal equilibrium dictates that this
must equal the rate of photoionization.  For radiative recombination:
$$rate\ of\ recombination
=n_+n_e\sigfb(v)v[f(v)dv]={\#\ of\ recombinations\over volume\ time}$$
We'll set this equal to the rate of photoionization. This rate is:
$$rate\ of\ photoionization
={B_\nu4\pi d\nu\over h\nu}n_0\sigbf
\overbrace{\left(1-{g_0\over g_+}{n_+\over n_0}\right)}^{{correction\ for\atop
stimulated}\atop recombination}$$
where $n_0$ is the \# density of neutrals. Note this has units of \# flux.
Note also that $\sigbf$ depends on $\nu$, and $n_+\over n_0$ is evaluated
at the relative velocity $v$ such that $h\nu=\hf m_ev^2+\chi$, where $\chi$
is the threshold ionization energy.  In thermal equilibrium, we know that:
$${n_+\over n_0}={g_+\over g_0}e^{-E\over kT}$$
where $E$ is the energy difference between state 1 (proton + unbound $e^-$)
and state 2 (bound proton/electron pair).  Thus $E$ is given by:
$$E=\hf m_ev^2-(-\chi)=h\nu$$
So we can make our n's and g's go away.  For $f(v)$, we'll use our Maxwellian:
$$f(v)=4\pi\left({m_e\over 2\pi kT}\right)^{3\over 2}v^2e^{-m_ev^2\over 2kT}$$
Finally, Saha tells us in thermal equilibrium:
$${n_+n_e\over n_0}=\left[{2\pi m_ekT\over h^2}\right]^{3\over 2}{2g_+\over g_n}
e^{-\chi\over kT}$$
So now we're essentially done:
$$1={n_+n_ef(v)v\sigfb(v)dv\over{4\pi B_\nu\over h\nu}d\nu(1-\ehvkt)
n_0\sigbf(v)}$$
and plugging in all of our relations we get:
$$\boxed{{\sigfb(v)\over\sigbf(\nu)}=
{g_0\over g_+}\left({h\nu\over m_ecv}\right)^2}$$
\centerline{(Milne Relation)}
Notice how all of the T's vanished.  
This result is independent of thermal
equilibrium.  However, you still have to pay attention to your statistical
weights (g's).

\section*{ Lecture 9 }

Recall that we were deriving the Milne relation:
$${\sigfb(v)\over\sigbf(\nu)}={g_0\over g_+}\left({h\nu\over m_ecv}\right)^2$$
$(\hf m_ev^2+\chi=h\nu)$.  In the following equation, Eugene has replaced
the $g_e$ of Rybicki and Lightman with the number $2$, because that is the
number of spin states of the electron.  This is just to be more clear.  So
the {\bf Saha} equation is:
$${n_+n_e\over n_0}=\left[{2\pi m_ekT\over h^2}\right]^{3\over2}{2g_+\over
g_0}e^{-\chi\over kT}$$
\centerline{(Saha Equation)}
Note that the number of internal degrees of freedom for the proton $g_+=2$, and
the number for neutral hydrogen:
$$g_0=\overbrace{n^2\cdot2}^{deg\ of\ freedom\ bound\ e^-}\overbrace{\cdot2}
^{proton}$$
$$g_0=4n^2$$
Beware that Shu says $g_0=2n^2$ and $g_+=2$.  We are now going to write the
``Recombination Coefficient'' $\alpha$:
$$\begin{aligned}\alpha&=\sum{\int_0^\infty{\sigfb(n,v)v\,f(v)dv}}\\ 
&=\sum_n{\mean{\sigfb(n,v)v}}\\ \end{aligned}$$
$$\vartheta(\alpha)\simeq\vartheta\left(\sigbf({h\nu\over m_ecv})^2v\right)$$
We'll estimate that for hydrogen, $\sigbf\sim10^{-18} cm^2$, $h\nu\approx
13.6eV$, and $v\sim\sqrt{2kT\over m_e}$, where $T\sim10^4K$.  Thus
$$\vartheta(\alpha)\simeq10^{-13}{cm^3\over s}\left({10^4K\over T}\right)^\hf$$
You can look this up in Osterbrock: at $T=10^4K$, the sum over all bound
states (``recombination''):
$$\alpha_A=4\e{-13}{cm^3\over s}$$
and the sum over all bound states except $n=1$ (we might be interested in
omitting the free-to-ground transition because it is likely to ionize
a nearby atom):
$$\alpha_B=2\e{-13}{cm^3\over s}$$

\subsection*{ Derivation of Saha}

Saha makes use of LTE.  This equation tells us what the \# density ratio is
between states 1 and 2.  We'll say that $n_{0,1}$ is the \# density of neutral
atoms with an $e^-$ in energy level 1.  $n_{+,1}$ is the \# density of ionized
atoms which still have an $e^-$ in energy level 1.  Saha says
collisional ionizations match the rate of collisional recombination:
$${n_+n_e\over n_0}=\left[{2\pi m_ekT\over h^2}\right]^{3\over2}{2g_+\over
g_0}e^{-\chi\over kT}$$
The Boltzmann relation says:
\def\npo{n_{+,1}}
\def\noo{n_{0,1}}
\def\gpo{g_{+,1}}
\def\goo{g_{0,1}}
$${\npo\over\noo}={\gpo\over\goo}e^{-E\over kT}$$
where $E=E_{state\ 2}-E_{state\ 1}=\hf m_ev^2-(-\chi)=\hf m_ev^2+\chi$.  Now
we need to figure out our numbers of internal degrees of freedom:
$$\gpo =\gpo\eval{internal}\cdot g_e\eval{internal}\cdot 
g_e\eval{translational\ motion}\cdot \gpo\eval{translational\ motion}$$
$g_e$ is easy: $g_e=2$.  $g_e\eval{translational}$ is harder:
$$g_e\eval{translational} = {(vol\ in\ conf\ space)(vol\ in\ momentum\ space)
\over h^3}$$
The volume in configuration space is $\inv{n_e}$.  The volume in momentum
space is $4\pi p^2dp=4\pi m_e^3v^2dv$. Now on to $\goo$:
$$\goo=\goo\eval{internal}\cdot\goo\eval{translational}$$
$${\npo\over\noo}=\int_0^\infty{\gpo\eval{internal}\cdot\gpo\eval{trans}
\over\goo\eval{internal}\cdot\goo\eval{trans}}{2\over n_e}
{4\pi m_ev^2dv\over h^3}e^{-(\hf m_ev^2+\chi_1)\over kT}$$
$${\npo n_e\over \noo}=\left[{2\pi m_ekT\over h^2}\right]^{3\over2}
{2\gpo\eval{int}\over\goo\eval{int}}e^{-\chi_1\over kT}$$
Now what if we were talking about going from neutral bound state in energy
level 2 (instead of 1) to an ionized atom with electron in energy state 1.
It turns out, we just need to replace $\goo$ with $g_{0,2}$ and
\def\got{g_{0,2}}
\def\nzt{n_{0,2}}
$\chi_1$ with $\chi_2=\chi_1-E_{0,12}$.  Thus:
$${\npo n_e\over\nzt}=\left[{2\pi m_ekT\over h^2}\right]^{3\over2}{2\gpo
e^{-\chi_2\over kT}\over\got e^{E_{0,12}\over kT}}$$
\def\tpmekth{\left[{2\pi m_ekT\over h^2}\right]^{3\over2}}
Where the rightmost, bottom factor used to be $\goo$.  In general,
$${n_{0,j}\over t_je^{-E_{0,1j}\over kT}}={\npo n_e\over\tpmekth 2\gpo
e^{-\chi_1\over kT}}$$
We'll define the right-hand side above to be $R$.  Then:
$${R\over n_0}=\inv{U_0(T)}={\npo n_e\over\tpmekth 2\gpo
e^{-\chi_1\over kT}n_0}$$
$${\npo n_e\over n_0}=\tpmekth{2\gpo\over U_0(T)}
e^{-\chi_1\over kT}$$
$$\boxed{{n_pn_e\over n_o}=\tpmekth{2U_+(T)\over U_0(T)}
e^{-\chi_1\over kT}}$$
This is the full Saha equation.  Remember that Saha assumes that we are in LTE:
the rate of collisional ionizations must equal the rate of collisional 
recombination.  This predicts why we had to wait until the
universe got to 3000K until recombination occurred.  We would have naively 
expected
that as soon as the temperature of the universe dropped below the ionizing 
energy of ground-state hydrogen, we would have recombination.  However, 
Remember that we can't follow Saha out too far.  Soon, collisions stop happening
often enough to maintain LTE, and Saha becomes invalid.  For example, Saha
would say that after recombination we would continue to lose free electrons
at the same (logarithmic) rate.  In truth, the number of free electrons
asymptotically approaches $10^{-3}n_B$ (where $n_B$ is the number of baryons).
We can estimate why this is.  The time for a proton to find an electron is
given by:
$$t=\inv{n_e\mean{\sigfb v}}\ll{a\over \.a}$$
That is, collisional ionizing equilibrium just starts to fail when:
$$\inv{n_e\mean{\sigfb v}}\sim{a\over \.a}$$
Since we know the Hubble time ($2\e5 yrs$), we can actually estimate
the number of free electrons in the universe at the time of recombination:
$$n_e\sim\inv{(2\e{-13})(2\e5)(\pi\e7)}\sim1$$

\section*{ Lecture 10 }

\subsection*{ Applications of Saha}

Recall that we had calculated the time for a proton to radiatively recombine
with an $e^-$ as:
$$t_{recomb}\sim\inv{n_e\mean{\sigfb v}}$$
Saha tells us that:
$${\npo\over\noo}=\underbrace{\left({2\pi m_ekT\over 
h^2n_e^{2\over3}}\right)^{3\over2}}_{{translational\atop phase\ space}\atop factor}
{2\gpo\over\goo}e^{-\chi_1\over kT}$$
where T is the temperature of matter.

\begin{itemize}
\item  T and z of recombination:\par
COBE has measured the temperature of the CMB to be:
$$T=3K(1+z)$$
Let's assume that just prior to recombination, photons and matter were
in thermal equilibrium, so:
$$T_\gamma\eval{recomb}=T_m\eval{recomb}$$
and also, the rate of photoionization of H by the photon gas should equal
the rate of radiative recombination.  These assumptions allow us to use the
Saha Equation.  Since ${\rho_0\over m_H}\sim{H^2\Omega_b\over Gm_H}$,
$$n=n_0(1+z)^3$$
Therefore, if we set ${n_+\over n_0}=\hf$ (we define this at recombination
time), then we can use Saha to solve for T, and then z.  After recombination,
there is no more Compton scattering to exchange energy between photons and
matter, so $T_\gamma\ne T_m$.  Since fractional ionization is determined by the
ratio of the rates of ionization and recombination (of which radiative
recombination is largest):
$$\inv{n_e\mean{\sigfb v}}\eval{recomb}\sim{a\over\.a}\eval{recomb}$$
That is, the timescale of radiative recombination is of order $2\e5yrs$.

\item  Kramer's Opacity (used for stellar interiors and accretion disk
interiors.\par
This is the opacity due to the photoionization of metals.  It is useful
in a gas at thermal equilibrium which is as sufficiently high temperatures
that H,He are nearly ionized, but metals retain their last few $e^-$.  This
formula is an approximate fitting formula (it doesn't account for the ionization
edges of metals).  It assumes the cosmic abundance of materials, and it
uses Saha.  It also uses $v\sim{kT\over h}$:
$$\boxed{K_{Kramer}={\rho_{cgs}\over T^{3.5}\cdot3\e{23}{cm^2\over g}}}$$
where $\rho_{cgs}$ is the density of all gas (including metals). Thus:
$$\rho K_{bf}=n_{metal}\sigbf$$
where $\rho,K_{bf}$ are for everything, $n_{metal}$ is the number of
extinguishers, and $\sigbf$ is the cross-section for extinguishers.
\def\nmet{n_{metals}}
$$\begin{aligned}R_{photoion}&\propto\nmet n_{photons}\sigbf\\ 
&\propto\nmet\underbrace{\left({aT^4\over h\nu}\right)}_{whole\ field\over
energy}\sigbf\\ 
&\propto\nmet{T^4\over T}\sigbf\\ \end{aligned}$$
Note that we used $h\nu=kT$.  Similarly, if $\alpha\sim T^{-\hf}$, then:
$$\begin{aligned}R_{rad-recomb}&\propto n_e\nmet+\alpha\\ 
&\propto n_e\nmet+T^{-\hf}\\ \end{aligned}$$
Setting these two results equal we get:
$$\nmet\sigbf\propto n_e\nmet+T^{-3.5}$$
$$K_{bf}\propto{n_e\nmet\over\rho} T^{-3.5}$$
And since $n_e\sim\rho$ and $\nmet\propto\rho Z$:
$$K_{bf}\propto\rho T^{-3.5}$$
\end{itemize}

\subsection*{ Instability Strip}

There is a strip on an HR diagram where stars will pulsate.  This is the
result of the exponential sensitivity of the ionization temperature
($e^{-\chi\over kT}$).  The following describes the ``K mechanism'' for
star pulsation.\par
In a normal star, if the outer shell loses pressure, the work done by
the collapse of the shell heats the star, increasing density and pressure.
$K_{bf}$ affects how fast radiation escapes, and when energy finally
does  escape, equilibrium is restored and the star maintains its new
size.  However, in zones of partial ionization, the adiabatic increase
in E goes into ionizing material.  This liberates $e^-$ without raising
the temperature.  Since temperature is constant and $\rho$ increases,
so $K_{bf}$ increases, and photons can't get out.  This continues until
He is fully ionized, and then the outer shell puffs up, and the cycle
continues.

\subsection*{ Collisional Ionization Cross-Section}

For H:
$$\sigma_{coll,bf}\sim\pi a_0^2\left({\chi\over E}\right)$$
for an $e^-$ hitting an H atom.  Note that $\chi$ is the ionization
energy, and E is the electron kinetic energy.  The cross-sections
for many atoms are uncalculated and unmeasured.  For more information,
take a look at (Bely and van Regemorter 1970 ARAA), (Bates 1962 Atomic
and Molecular Processes), and (Jefferies 1968 Spectral Line Formation).

\section*{ Lecture 11 }

\subsection*{ Bremsstrahlung (braking radiation)}

Bremsstrahlung is the continuum of emission from a plasma caused by
the deflection of charged particles off of one another.  The most
important deflection which we will talk about is of an $e^-$ by
a positive nucleus.  Recall that the power radiated by an accelerated
$e^-$ is:
$$P={2\over3}{e^2a^2\over c^3}$$
In this case, the acceleration is from the electrical force of a
nucleus, so $a\sim{Ze^2\over b^2m_e}$, where b is the distance of
closest approach between the $e^-$ and the nucleus.  Thus:
$$P={2\over3}{Z^2e^6\over b^4m_e^2c^3}$$
The energy released by this encounter is given by $E\sim P\Delta t$, where
$\Delta t$ is about how long the $e^-$ is within order b of the nucleus.  Thus
$\Delta t\sim{2b\over v}$, where v is the velocity of the electron.  Therefore:
$$E\sim{4\over3}{Z^2e^6\over m_e^2c^3b^4}{b\over v}$$
From the point of view of the $e^-$, the force of the $\ef$ of the nucleus
starts out pulling the $e^-$ almost directly forward, and ends up pulling
the $e^-$ nearly backward.  If we graph the portion of this force which
pulls the $e^-$ sideways, we have something that looks a lot like the
upper half of a sine wave of period $2\Delta t$.  
This enables us to relate $d\nu$ to
$db$, which will be useful to us in a minute:
$$d\nu\sim{v\over4b}db$$
Now let's consider $e^-$'s going in a ring around a nucleus at radius
b.  The ``fraction'' of an electron in a section db of that ring is 
$2\pi b\,db\,n_ev$, where $n_e$ is the number of electrons in the ring.
Thus, the power radiated by that section of ring is:
$$dP\sim E\cdot2\pi b\,db\,n_ev$$
Using our relation between $d\nu$ and db, and our expression for E:
$${dP\over d\nu}\sim{32\pi\over3}{Z^2e^6n_e\over m_e^2c^3v}$$
So the power per frequency interval is independent of distance.\par
Now let's assume a Maxwellian
velocity distribution.  We'll define a $\nu_{fix}$ such that
$h\nu_{fix}\sim\hf mv_{min}^2$, where $v_{min}$ is the minimum velocity
required to keep an $e^-$ moving unbound around a nucleus.  Then
the average total power released over all frequencies is:
$$\begin{aligned}\mean{P\over\nu}&=\int_{v_{min}}^\infty{{dP\over d\nu}4\pi
\left({m_e\over2\pi kT}\right)^{3\over2}e^{-m_ev^2\over2kT}v^2dv}\\ 
&={64\sqrt{\pi}\over3\sqrt{2}}{Z^2e^6n_e\over m_e^{3\over2}c^3(kT)^\hf}
e^{-h\nu\over kT}\\ \end{aligned}$$
\def\jnff{j_{\nu,ff}}
We now define $\jnff$ to be the volume emissivity for free-free interactions.
That is, $\jnff$ measures the power radiated by plasma, per volume, into
a solid angle $d\Omega$.  We can calculate the ``per $d\Omega$'' because this
radiation is isotropic:
$$\boxed{\jnff={16\over3\sqrt{2\pi}}{Z^2e^6\over m_e^{3\over2}c^3(kT)^\hf}
n_en_pe^{-h\nu\over kT}}$$
where $n_p$ is the \# density of ions.  This is the expression for
Thermal Bremsstrahlung.  Note that the definition of $\jnff$ in Rybicki \&
Lightman has an additional factor of ${\pi\over\sqrt{3}}\_g_{ff}(v,T)$, which
is a quantum mechanics correction factor of order unity.  It's called the
``Gaunt factor''.  Compare $\nu\jnff$ to the power emitted by a blackbody:
they both peak at $4kT\over h$, and for small $\nu$, they both go as
$\nu^3$.

\subsection*{ Inverse Bremsstrahlung}

An $e^-$ can also absorb a photon and become more energetic and ``free''.  We 
define the coefficient for thermal free-free absorption as:
\def\anff{\alpha_{\nu,ff}}
$$\boxed{\anff\equiv{\jnff\over B_\nu}}$$
We can express the opacity as:
\def\knff{K_{\nu,ff}}
$$\boxed{\knff={\anff\over\rho}\propto{n_en_p\nu^{-3}(e^{-h\nu\over kT}-1)\over
\rho\sqrt{T}e^{h\nu\over kT}}}$$
where $\rho$ is the total density.  Since most photons have $h\nu\sim kT$,
$$\begin{aligned}\knff&\propto{n_en_p\over\rho}{T^{-3}\over\sqrt{T}}\\ 
&\propto\rho T^{-3.5}\\ \end{aligned}$$
See how we got back to Kramer's opacity!\par
It is important to remember the assumptions we made to get here:
\begin{itemize}
\item Maxwellian velocity distribution, which should be valid since the
collision time scale is very small, so the system should relax to a 
Maxwellian distribution quickly.
\item Non-relativistic, so $T\le{m_ec^2\over k}\sim7\e9K$, which
is an okay assumption for most plasmas.\par
Some examples of Thermal Bremsstrahlung are HII regions, and the diffuse
IGM (which contains nuclei and H).
\end{itemize}

\section*{ Lecture 12 }

\subsection*{ Synchrotron (Magneto-Bremsstrahlung)}

Synchrotron radiation is radiation caused by a magnetic field accelerating
a charge.  It is generally relativistic (the Lorentz factor $\gamma\gg1$).
We'll discuss this later, but first will do Cyclotrons, where $\gamma\approx1$.

\subsection*{ Cyclotron}

If we have an electron moving in circles in a magnetic field B, then the
frequency of its ``orbit'' is:
$$\omega_{cyc}={eB\over m_ec}$$
We can compute the power pattern radiated by this electron in a direction 
$\theta$ (measured from the vector toward the center of the orbit) by the 
acceleration of the charge:
$${dP\over d\Omega}={e^2a^2\over c^3}{\sin^2\theta\over4\pi}$$
where $a$ is the acceleration of the charge, given by:
$$a={evB\over cm_e}$$

\subsection*{ Derivation of Power Pattern}

Say that we have an $e^-$ moving in a straight line along the x axis, and
it gets accelerated, starting at $x=0$, for a duration $\Delta t$.  Let
$\theta$ measure the angle of a field line emitted by the $e^-$, from the
acceleration axis.  The 
acceleration of $e^-$ causes a ``jog'' in the electric field compared to where
it would have been (this jog propagates outward at $c$).  Say that $E_{tr}$
is the electric field carried in the transverse component 
($\perp$ to $\theta$), and
$E_r$ measures the component in the radial component of the jog.  Then:
$$\begin{aligned}{E_{tr}\over E_r}&={(a\Delta t)t\sin\theta\over c\Delta t}\\ 
&={at\sin\theta\over c}\\ \end{aligned}$$
Just to get that factor of $t$ out of there, say that $r=ct$:
$${E_{tr}\over E_r}={ar\sin\theta\over c^2}$$
We know that $E_r={e\over r^2}$, so:
$$\boxed{E_{tr}={ea\sin\theta\over c^2r}}$$
Note how the transverse field goes off as $r^{-1}$, so when
you go far enough away, it always wins out over the plain electric field.\par
What's the transverse magnetic field?  Maxwell's equations tell us:
$$\begin{aligned}\dce=-{1\over c}{\partial B\over\partial t}\\ 
&=\vmatrix\^x&\^y&\^r\\
{\partial\over\partial x}&{\partial\over\partial y}&{\partial\over\partial r}\\
E_{tr}&0&E_r\endvmatrix\\ 
&=\^x({\partial E_r\over\partial y})+\^y({\partial E_{tr}\over\partial r}
-{\partial E_r\over\partial y})+\^r({\partial E_{tr}\over\partial y})\\ 
&\approx\^y{\partial E_{tr}\over\partial r}\\ \end{aligned}$$
Say we define $f$ such that:
$$E_{tr}=f(r-ct)$$
then $\vec B=f(r-ct)\^y$, so:
$$\begin{aligned}-{1\over c}{\partial f(r-ct)\^y\over \partial t}
&={-\^y\over c}{\partial f(r-ct)\over \partial(r-ct)}{\partial(r-ct)\over
\partial t}\\ 
&=\^y{\partial f(r-ct)\over\partial(r-ct)}\\ \end{aligned}$$
Also:
$$\begin{aligned}\dce&={\partial E_{tr}\over\partial r}\\ 
&=\^y{\partial f(r-ct)\over\partial(r-ct)}{\partial(r-ct)\over\partial r}\\ 
&=\^y{\partial f(r-ct)\over\partial(r-ct)}\\ \end{aligned}$$
Thus, $E_{tr}$ and $B_{tr}$ have the same magnitude.\par
To cut to the chase, the power radiated from the entire sphere is:
$$\begin{aligned}P_{sphere}&={e^2a^2\over4\pi r^2c^3}\int_0^\pi{\sin^2\theta 
2\pi\sin\theta r^2d\theta}\\ 
&={2\over3}{e^2a^2\over c^3}\\ \end{aligned}$$
This is the Larmor power formula.  We can also calculate the Poynting flux:
$${\vec E\times\vec B\over4\pi}c=e^2a^2{\sin^2\theta\over4\pi r^2}$$
Incidentally, you get the same result if you accelerate the charge perpendicular
to the direction it was moving.\par
The {\it axis of polarization} of this radiation is:
$$(\vec a\times\vec r)\times\vec r$$
Remember that $\vec a$ is the acceleration axis, and $\theta$ is always measured
from that axis.
\begin{itemize}
\item Returning to the cyclotron example, if we observe from above
an $e^-$ going in circles, we should see a constant flux (our angle with respect
to the $e^-$ isn't changing), and it will be {\it circularly polarized}, 
because the axis of polarization goes in circles as the $e^-$ goes in circles.
If we were to observe this circling edge-on, then we'd receive linearly
polarized radiation, and the flux from the accelerating $e^-$ oscillates
with a frequency twice that of the $e^-$'s orbit.
\end{itemize}

Looking at the electron cyclotron emission from solar wind particles, we see
that the flux density of the aurora from most planets drops after a frequency
of about 1MHz, as dictated by the cyclotron frequency:
$$\omega_{cyc}={eB\over m_ec}$$
This tells us that the magnetic field for most planets is about the same.
Jupiter is different because it has a very strong magnetic field.

\section*{ Lecture 13 }

\subsection*{ Clarification on Milne Relation}

Recall that we had an expression for the rate of photoionization:
$$R_{ion}={B_\nu4\pi d\nu\over h\nu}n_0\sigbf
\overbrace{\left(1-{g_0\over g_+}{n_+\over n_0}\right)}^{{correction\ for\atop 
stimulated}\atop recombination}$$
It looks as though, for a highly ionized gas, the correction for stimulated 
emission (which we may alternately write as $1-e^{-h\nu\over kT}$) could 
become negative (which is bad).
Defining G to be the ratio of the rate stimulated 
recombination to the rate of spontaneous recombination, so that:
$$R_{ion}={4\pi B_\nu\over h\nu}d\nu\sigbf(\nu)\inv{1+G}$$
Then by analogy with line emission:
$$G={n_2\bto\_J\over n_2\ato}={\_J\over {2h\nu^3\over c^2}}$$
Recall that $\_J=\int_0^\infty{J_\nu\phi(\nu)d\nu}$, and in thermal
equilibrium $J_\nu=\inv{4\pi}\int{I_\nu d\Omega}$, so assuming that
$\phi(\nu)$ is about a delta function, 
$\_J\approx B_\nu$.  Thus for G, we get:
$$G={B_\nu\over{2h\nu^3\over c^2}}$$
We know by conservation of energy that for
stimulated emission:
$$\begin{aligned}h\nu+\hf m_ev_e^2&=2h\nu-\chi\\ 
h\nu&=\hf m_ev_e^2+\chi>0\\ \end{aligned}$$
so it turns out that the stimulated emission term cannot be negative.  In 
general,
if you're unsure about whether to include the correction for stimulated
recombination, compute G, and it will tell you.

\subsection*{ Synchrotron Radiation}

The Synchrotron is a relativistic cyclotron, so the Lorentz factor 
$\gamma\gg1$.  As a result, the angular power pattern of an $e^-$ circling in a B
field will take a new form.  Instead of having a ``donut'' of power emitted
from the accelerated charge, much more of the field is going to be thrown in
the forward direction, and much less in the backward direction.  To see this,
let's define the {\it primed frame} to be the instantaneous rest frame of an
$e^-$ which, in our frame, is moving in the $\^x$ direction and being 
accelerated in the $\^y$ direction.
In the primed frame, the acceleration of the electron ($\ap$) is still pointing
in the $\^y$ direction.  We can relate $\ap$ to $a$ by
noting that if $\vec\ap=\ap_y\^y^\prime+\ap_x\^x^\prime$:
$$\begin{aligned}a_x&={\ap_x\over\gamma^3(1+{vu_x^\prime\over c^2})}\\ 
a_y&={\ap_y\over\gamma^2(1+{vu_x^\prime\over c^2})}\\ \end{aligned}$$
where $\upx$ is the observed velocity in the primed frame: 
$\upx\equiv{d\xp\over d\tp}$.
\begin{itemize}
\item Proof: The Lorentz transformation tells us:
$$\begin{aligned}\xp&=\gamma(x-vt)\\ 
\yp&=y\\ 
\zp&=z\\ 
\tp&=\gamma(t-{vx\over c^2})\\ \end{aligned}$$
Taking derivatives of these equations:
$$\begin{aligned}d\xp&=\gamma(dx-vdt)\\ 
d\yp&=dy\\ 
d\zp&=dz\\ 
d\tp&=\gamma(dt-{vdx\over c^2})\\ \end{aligned}$$
Thus:
$$\begin{aligned}{dx\over dt}&={d\xp+vd\tp\over d\tp+{vdxp\over c^2}}\\ 
&={\upx+v\over1+{vu_x\over c^2}}\\ 
{dy\over dt}&={\upy\over\gamma(1+{vu_x\over c^2})}\\ \end{aligned}$$
Defining $\theta$ to be the angle of a vector from the $\^x$ direction (and
similarly, $\theta^\prime$ from $\^x^\prime$), then:
$$\begin{aligned}\tan\theta&={u_y\over u_x}={\upy\over\gamma(\upx+v)}\\ 
&={\tan\theta^\prime\over\gamma(1+{v\over\upx})}\\ \end{aligned}$$
Since $\upx=c\cos\theta^\prime$:
$$\begin{aligned}\tan\theta&={\tan\theta^\prime\over\gamma(1+{v\over 
c\cos\theta^\prime})}\\ 
&={\sin\theta^\prime\over\gamma(\cos\theta^\prime+\beta)}\\ \end{aligned}$$
\end{itemize}

Here's a chart relating $\theta$ and $\theta^\prime$ for $\beta\approx1$,
$\gamma\gg1$:
$$\begin{matrix}
\theta^\prime&\theta\\
0^\circ&0^\circ\\
45^\circ&{\sqrt{2}\over2+\sqrt{2}}{1\over\gamma}\ll1\\
90^\circ&\inv{\gamma}\\
135^\circ&{\sqrt{2}\over2-\sqrt{2}}\inv{\gamma}\\
180^\circ&180^\circ\\
\end{matrix}$$
As is evident, photons emitted various angles in the $e^-$'s rest frame
end up being beamed forward in the lab frame.
This has applications in gamma ray bursts.  When a star goes supernova, 
it's electrons are accelerated to relativistic velocities.  Because of
relativistic beaming, we can only see the few electrons whose beams
point at us.  As the $e^-$'s slow down (because they are radiating power),
they stop being relativistic, and we get to see radiation from a larger
angle.  Thus, the flux curve is ``flattened'' for times shortly after
a star goes supernova.

\subsection*{ Synchrotron Characteristic Frequency}

Suppose you are observing an $e^-$ emitting synchrotron radiation.  Since
radiation is just being emitted forward by the $e^-$, you won't see radiation
from the electron very often.  In fact, you'll just see it once per revolution.
Each time you do see it, you will see a spike in power.  We'd like to figure
out long in time these pulses are separated.  To do this, we have the 
following equations for relativistic motion:
$$\begin{aligned}\ddt\vec P&={e\over c}\vec v\times\vec B\\ 
\ddt(\gamma m\vec v)&={e\over c}\vec v\times\vec B+e\vec E\\ \end{aligned}$$
$$\ddt(Energy)=\ddt(\gamma mc^2)=e\vec v\cdot\vec E$$
Now $\vec B=\vec B_{external}=B\^z$ and $\vec E=\vec E_{external}=0$.  There
are also contributions of the self-interaction of the electron's field with
the electron, but we'll neglect these as being a minor perturbation.  Then:
$$\boxed{\gamma m{d\vec v\over dt}=e\vec v\times\vec B}$$
Let's define $\alpha$ to be the ``pitch angle'' between $\vec B$ and $\vec v$
(that is, the angle which makes the $e^-$ travel in a helix instead of a 
circle).
Then:
$$\gamma m{v_\perp^2\over r_p}={evB\sin\alpha\over c}$$
where $r_p$ is the projected radius of orbit, looking down on $\vec B$. Thus:
$$\gamma m{(v\sin\alpha)^2\over r_p}={evB\sin\alpha\over c}$$
$$r_p={\gamma mc\over eB}v\sin\alpha$$
The time to make an orbit is ${2\pi\over\omega_B}$, neglecting radiation
reactions, so:
$$\omega_B={v_\perp\over r_p}={v\sin\alpha\over r_p}={eB\over\gamma mc}
={\omega_{cyc}\over\gamma}$$

\subsection*{ Photons Chasing Photons}

The photons emitted by the $e^-$ are being squished together by the fact that
the $e^-$ is itself moving close to the speed of light.  If some $e^-$ is
spiraling around, there is only a tiny arc over which the $e^-$ emits
photons that we can see.  We would like to calculate this arc in order to
figure the width of the pulse of radiation an observer sees from synchrotron
radiation.  The width of the arc over which the $e^-$ emits radiation that
we see (as the electron sweeps its beam past us) is just the width of the
beam that the $e^-$ emits, which is $2\over\gamma$.  The time interval
over which this arc is swept out is determined by the time it takes the
$e^-$ to travel an angle of $2\over\gamma$ around the circle.  However, we can
also calculate this by noting that the change in v over the interval must
come from the acceleration of the $\vec B$ field:
$$\gamma{\Delta v\over\Delta t_{21}}={evB\sin\alpha\over c}$$
$${\Delta v\over\Delta t_{21}}=\omega_Bv\sin\alpha$$
$$\Delta t={\Delta v\over \omega_B v\sin\alpha}$$
Now $\Delta v\sim v{2\over\gamma}$, so:
$$\boxed{\Delta t_{21}\sim{2\over\gamma\omega_B\sin\alpha}}$$

\section*{ Lecture 14 }

\subsection*{ Finishing Synchrotron Characteristic Frequency}

In computing the time width of the pulses received from an $e^-$ emitting 
synchrotron radiation, we first measure the time during which the 
electron emits radiation toward the observer (starting at some initial
point 1 and finishing at point 2):
$$\Delta t_{21}={2\over\gamma\omega_B\sin\alpha}$$
However, during the time the electron was emitting this radiation, it was
also moving toward us, so we receive these photons in a shorter burst than
they were emitted at.  The difference in the actual physical length of
\def\tto{{t_{21}}}
the emission goes from being $c\Delta\tto$ to $(c-v)\Delta\tto$ (we assume
that for the duration of the emission, the $e^-$ is going approximately
straight at you).  Thus we can get the actual time width of the pulse:
$$\Delta t={(c-v)\Delta\tto\over c}=(1-\beta)\Delta\tto$$
For $\gamma\gg1$, $\beta\to1$, so:
$$\Delta t\approx\inv{\omega_{cyc}\gamma^2\sin\alpha}$$
This factor of $\inv{\gamma^2}$ comes from the following:
$${\gamma_{rel\atop mass}\over
\gamma_{{phot\atop chase}\atop phot}^2\gamma_{beam}}$$
Using $\Delta t$, we can deduce the synchrotron frequency.  We've just done 
an approximate computation here.  Rybicki \& Lightman do it
for real and get
$$\boxed{\omega_{sync}={3\over2}\gamma^2\omega_{cyc}\sin\alpha}$$

\subsection*{ Synchrotron Power}

Notice that the $\gamma^2$ makes synchrotron radiation ``harder'' than
cyclotron radiation.  In a cyclotron, the power radiated into all solid
angles is:
$$P={2\over3}{e^2a^2\over c^3}$$
Let's derive this for the synchrotron.  In the electron frame:
$$P^\prime={2\over3}{e^2a^{\prime 2}\over c^3}$$
It turns out that power is a relativistic invariant.  To see this, note
$P^\prime={dU^\prime\over dt^\prime}$, and we know the following:
$$\begin{matrix}
\begin{aligned} U^\prime&=\gamma(U-vp_x)\\ 
t^\prime&=\gamma(t-{vx\over c^2})\\ \end{aligned}&
\begin{aligned} U&=\gamma(u^\prime+vp_x^\prime)\\ 
t&=\gamma(t^\prime+{v\xp\over c^2})\\ \end{aligned}
\end{matrix}$$
and in the prime frame  ($s^\prime$), $d\xp=dp_x^\prime=0$.\par
So we've shown that once we calculate $P^\prime$, we know $P$ in all frames.
What we need to do now is calculate $a^\prime$.  We'll do the following:
\begin{itemize}
\item  Get $a^\prime(a)$, the Lorentz transform of acceleration.
\item  Get $a^\prime$ due to $E^\prime$.
\end{itemize}
As an exercise in special relativity, we'll derive what $\ef^\prime$ is by
\def\sigo{\sigma_0}
investigating two parallel plates--the bottom one having charge density $\sigo$ 
and the top having $-\sigo$ We
know in a motionless frame that the field between the plates is:
$$\begin{aligned}E_{y_0}&=4\pi\sigma_0\\ 
B_0&=0\\ \end{aligned}$$
In a frame ($s$) where we are moving at $v=-v_0\^x$, these become:
$$\begin{aligned}E_y&=4\pi\sigma=4\pi\gamma_0\sigo=\gamma_0E_{y_o}\\ 
B_z&={4\pi J\over c}={4\pi\over c}\sigma v_0={4\pi\over c}\gamma_0\sigo v_0=
E_{y_0}{v_0\over c}\gamma_0\\ \end{aligned}$$
Jumping into one more frame ($s^\prime$) where $v=v^\prime\^x$ relative to 
frame $s$, we have:
$$\begin{aligned}E_y^\prime&=4\pi\sigma^\prime\\ 
B_z^\prime={4\pi\over c}\sigma^\prime v^\prime\\ \end{aligned}$$
We need to figure $\sigma^\prime$, and our first instinct might be to say
$\sigma^\prime=\gamma^\prime\sigma$, but that is wrong.  We have to reference
it from $\sigo$:
$$\sigma^\prime=\gamma^\prime\sigo$$
Now we do some algebra:
\def\gamp{{\gamma^\prime}}
\def\gamo{{\gamma_0}}
$$\begin{aligned}E_y^\prime&=4\pi\gamp\sigo\\ 
&=4\pi\gamma\gamo(1-{vv_0\over c^2})\sigo\\ 
&=\underbrace{4\pi\gamo\sigo}_{E_y}\gamma-\underbrace{4\pi\gamo v_0\sigo\over
c}_{B_z}{\gamma v\over c}\\ \end{aligned}$$
Thus we have:
$$\begin{matrix}
\boxed{E_y^\prime=\gamma E_y-{\gamma v\over c}B_z}&
\boxed{E_z^\prime=\gamma E_z+{\gamma v\over c}B_y}&
\boxed{B_y^\prime=\gamma B_y+{\gamma v\over c}E_z}
\end{matrix}$$
And, of course, $E_x^\prime=E_x$.  The only thing we need to get now is
$B_x^\prime$.  For this we'll talk about a solenoid aligned with the $\^x$
direction with $n$ turns per unit length.  The field of this solenoid
in the rest frame is:
$$B_x={4\pi\over c}nI$$
Jumping to a frame where $\vec v=v\^x$, $B_x^\prime={4\pi n^\prime I^\prime\over
c}$. Using that:
$$\begin{aligned}n^\prime&=\gamma n\\ 
I^\prime&={dQ^\prime\over dt^\prime}={dQ\over\gamma dt}={I\over\gamma}\\ \end{aligned}$$
We find that:
$$\boxed{B_x^\prime=B_x}$$
Recall that we've derived all of this for boosts in the $\^x$ direction.
To be completely general, we'll write them for any direction:
$$\begin{matrix}\begin{aligned}E_\|^\prime&=E_\|\\ 
B_\|^\prime&=B_\|\\ \end{aligned}&
\begin{aligned}\vec E_\perp^\prime
&=\gamma(\ef_\perp+{\vec v\over c}\times\bfield)\\ 
\vec B_\perp^\prime&=\gamma(\bfield_\perp-{\vec v\over c}\times\ef)\\ 
\end{aligned}
\end{matrix}$$
So finally, back to the synchrotron.  Since $e^-$ is at rest in the primed
frame:
$$\vec a^\prime={e\ef^\prime\over m_e}$$
$$\ef_\|^\prime=\ef_\|=0$$
$$\ef_\perp^\prime=\gamma{\vec v\times\bfield\over c}$$
$$|E_\perp|={\gamma vB\over c}\sin\alpha$$
Therefore, the magnitude of the acceleration is:
$$|a^\prime|={e\gamma vB\over m_ec}\sin\alpha$$
And so the power radiated is:
$$\boxed{P^\prime={2\over3}{e^4\gamma^2v^2B^2\over m_e^2c^5}\sin^2\alpha}$$
Note that as $v\to c$,
$$P\to{2\over3}{e^4\gamma^2B^2\over m_e^2c^3}\sin^2\alpha$$
Thus we get {\it way} more power ($\gamma_{v\to c}^2\over\beta_{v\ll c}^2$) 
out of the synchrotron.  How long can an $e^-$ hold up radiating this kind
of power?
$$t_{life}\sim{\gamma m_ec^2\over\left({e^4\gamma^2B^2\over m_e^2c^3}\right)}$$
The time it takes an $e^-$ to go in the circle is just:
$$t_{orb}\sim{\gamma m_ec\over eB}$$
Taking the ratio of these, we find that the critical $B$ required to make
these timescales comparable is:
$$\gamma^2B\sim{c^4m_e^2\over e^3}\approx{3^4\e{40}\e{-54}\over
5^3\e{-30}}\approx10^{16}cgs$$
Getting back to P, there is a prettier way of writing it:
$$P=\overbrace{2\sigma_Tc\,U_B}^{E\ intercepted\atop by\ e^-}
\sin^2\alpha\overbrace{\gamma^2}^{relativistic\atop enhancement}$$
where $\sigma_T={8\pi\over3}r_0^2$ is the Thomson cross-section ($r_0$ being
defined by ${e^2\over r_0}=m_ec^2$), and $U_B$ is the energy stored in the
magnetic field $U_B={B^2\over8\pi}$.

\section*{ Lecture 15 }

\subsection*{ Synchrotron Cooling Time}

To estimate the Synchrotron cooling time, we'll set up our standard expression
of self-energy over power radiated:
$$t_{cool}\sim{\gamma m_ec^2\over u_Bc\sigma_T\gamma^22\cdot\sin^2\alpha}$$
Instead of doing anything fancy with $\sin^2\alpha$, we'll just use that
$\mean{\sin^2\alpha}={2\over3}$.  $U_B$ we can estimate as $\mean{B^2}\over
8\pi$, giving us:
$$\begin{aligned}t_{cool}&\sim{m_ec^2\over{4\over3}\sigma_Tc}\inv{\gamma U_B}\\ 
&\sim16yr\left({1\,Gauss\over B}\right)^2\left(\inv{\gamma}\right)\\ \end{aligned}$$
\begin{itemize}
\item Let's examine the cooling time for radio jets, 
where $B\sim1mGauss$, 
and $\gamma\sim10^3$.  Plugging this in, we get $t_{cool}\sim10^4yr$.  Compare
this to $t_{dyn}\sim{l\over c}\sim{1kpc\over c}\sim10^4yr$.

\item We'll also estimate the cooling time of the Crab Nebula.  To set
an upper bound on $t_{cool}$, we'll use the most energetic X-rays.  We can
do this because $\gamma$ uniquely determines the electron energy and as a 
result, it uniquely determines a photon energy.  For the Crab Nebula, we'll
pick a photon energy: $E=4keV$.  Then:
$$\begin{aligned}\omega&\sim\omega_{crit}\sim\gamma^2\omega_{cyc}\\ 
&\sim{E\over\hbar}\sim\gamma^2{eB\over m_ec}\\ \end{aligned}$$
$B\sim mG$ gives us that $t_{cool}\sim2yrs$.  Clearly, since the Crab Nebula
was created some 1000 years ago, there must be a source of fresh electrons.
This source is the pulsar, sitting in the middle of the nebula.
\end{itemize}

\subsection*{ Polarization}

Returning to the non-relativistic cyclotron, recall that when we observe 
an $e^-$ in a circular orbit from the plane of that orbit, we see linearly
polarized photons, and when we observe it from above the plane of the
orbit, we see circularly polarized photons.  In the synchrotron, 
the total emission of the $e^-$ (both the linearly polarized photons
in the plane of the orbit and the circularly polarized photons perpendicular
to the orbit) are swept into the forward direction.  Thus, for a single
electron, we'd see the top and bottom fringes of the beam are circularly
polarized, and the center of the beam is linearly polarized.  If we have
several electrons, the bottom (circularly polarized) fringe of
its beam will overlap the top (oppositely circularly polarized) fringe
of the next electrons beam.  These will tend to cancel, and we find that
synchrotron radiation is generally dominated by linearly polarized photons.
The upper bound for how much of the radiation is linearly polarized is set by
a perfectly uniform $B$ field, where 75\% is linearly polarized.

\subsection*{ Spectra of Synchrotron Radiation}

The power spectrum of a single $e^-$ undergoing synchrotron radiation peaks at 
$\omega_{crit}\sim\gamma^2\omega_{cyc}$.  For small $\omega$, $P$ goes as
$\omega^{1\over3}$, and for $\omega\gg\omega_{crit}$, $P$ goes as
$\omega^\hf e^{-\omega}$.  In general, recall that $P\propto B^2\gamma^2$.
We would like to calculate the power spectrum of an ensemble of $e^-$.  To do
this, we need to describe how many electrons there are per energy.  We'll
assume a power law distribution of $e^-$ energies (${dN\over dE}$ goes as
$E^p$, where $p$ is the {\it differential energy spectrum index}).  We make this
assumption simply because this coincides with our observations (see Nilsen and
Zager).  Let's consider the power radiated by electrons with energies between
$E$ and $E+dE$:
$$\begin{aligned}dP&=dN\times P\eval{single\ e^-}\\ 
&\propto\left({dN\over dE}dE\right)\gamma^2B^2\propto(E^pdE)E^2B^2\\ 
&\propto E^{2+p}B^2dE\\ \end{aligned}$$
Thus we have:
$${dP\over dE}\propto E^{2+p}B^2$$
Now let's say that we have a one-to-one relationship between $E$ and $\omega$.
That is, electrons of a specific energy $E$ are solely responsible for
generating photons of frequency $\omega\sim\gamma^2\omega_{cyc}$.  Note also
that $\omega\propto\gamma^2B\propto E^2B$, so $E\propto\left({\omega\over B}
\right)^\hf$.  Therefore:
$$\begin{aligned}{dP\over d\omega}&={dP\over dE}{dE\over d\omega}\\ 
&\propto(E^{2+p}B^2)B^{-\hf}\omega^{-\hf}\\ 
&\propto\left({\omega\over B}\right)^{2+p\over2}B^{3\over2}\omega^{-\hf}\\ \end{aligned}$$
$$\boxed{{dP\over d\omega}\propto B^{1-p\over2}\omega^{1+p\over2}}$$
Usually, $p<0$.  Note that our assumption of one-to-one correspondence was
not necessary to get this power law dependence on $\omega$.  If hadn't made
that assumption, we would have found that the sum of the contributions of
electrons in nearby energies would have yielded the same result we got, except
near the edges of $\omega$.  Also note that this expression relies on an
influx of $e^-$ to replace old ones which cooled down.  If we cut off this
influx, we'll see that since $t_{cool}\sim\inv{\gamma B^2}$, the $e^-$ emitting
higher $\omega$ photons decay first, and so we see turn-offs from a power law
distribution for increasingly low $\omega$ as time goes by.\par
For measuring $p$, let's define $\alpha\equiv{1+p\over2}$.
Observations of extended radio sources have measured $\alpha\approx-0.7
\imply p=-2.4$.  In general, we find that $-0.75\le\alpha\le-0.5$, or
$-3\le p\le -2$.  
Now we've made an assumption of a constant magnetic field.
Suppose we have an optically thin synchrotron emitting gas with a power law
emissivity
$j_{\nu_1}>j_{\nu_2}$ for $\nu_1<\nu_2$.  We will use observations through
this thin gas to infer a ``minimum pressure'' or ``minimum energy density''.  
The electron pressure (or electric energy density) is given by:
$$\begin{aligned}P_e&\propto\int_{E_1\leftrightarrow\nu_1}^{E_2\leftrightarrow\nu_2}
{{dN \over dE}dE\,E}\\ 
&\propto E^{2+p}\eval{E_1}^{E_2}\\ \end{aligned}$$
For $p\le-2$,
$$P_e\propto E_1^{2+p}$$
In reality, spectra might cut off outside of the range of our observations.
To correct for this, we'll carry around a correction factor of $\left({E_{min}
\over E_1}\right)^{2+p}$.  Measuring $j_\nu$, we get:
$$\begin{aligned}j_\nu(\nu_1\leftrightarrow E_1)&\propto\left({dN\over dE}dE\right){P
\eval{single\ e^-}\over\nu_1}\\ 
&\propto\left({dN\over dE}dE\right){B^2\gamma^2\over\gamma^2 B}
\propto{dN\over dE}dE{E_1^2B\over E_1^2}\\ 
&\propto\overbrace{{dN\over dE}dE\,E_1}^{P_e}{E_1B\over E_1^2}\\ 
&\propto P_e{B\over E_1}\propto P_eB\left({B\over\nu_1}\right)^\hf\\ \end{aligned}$$
$$j_\nu\propto P_eP_{mag}^{3\over4}\nu_1^{-\hf}$$
where $P_{mag}$ is the magnetic energy density.  
Now since $P_{tot}=P_e+P_{mag}$:
$$\boxed{P_{tot}={C\,j_\nu\over P_{mag}^{3\over4}\nu^{-\hf}}+P_{mag}}$$
This expression has a minimum for a unique $P_{mag}$.  This $P_{mag}$ tells
us the way energy is partitioned in the system between $P_e$ and $P_{mag}$.
The minimum should occur when $P_e=P_{mag}$.

\section*{ Lecture 16 }

Recall that last time we derived for an optically thin synchrotron gas that:
\def\pmag{{P_{mag}}}
$$j_\nu(\nu_1)\propto P_eP_{mag}^{3\over4}\nu_1^{-\hf}$$
$$\begin{aligned}P_{tot}&=P_e+P_{mag}\\ 
&={Cj_\nu\over\pmag^{3\over4}\nu_1^{-\hf}}+\pmag\\ \end{aligned}$$
Thus, the minimum total power occurs near the equipartition point between
$\pmag$ and $P_e$.  This gives us:
\def\jn{j_\nu}
$${C\jn\over\pmag^{3\over4}\nu_1^{-\hf}}\sim\pmag$$
$${B^2\over8\pi}=\pmag\sim(C\jn\nu_1^\hf)^{4\over7}$$

\subsection*{ Synchrotron Self-Absorption}

To discuss synchrotron self-absorption, we need to discuss what the spectrum
of an optically {\it thick} medium of relativistic electrons looks like.
If we take a bunch of $e^-$ spiraling around magnetic field lines with
$\_E=\gamma m_ec^2$, then the energy of the photons emitted by these
electrons is $h\nu\sim h\nu_{crit}\sim h\nu_{cyc}\gamma^2$.  If this gas
were optically thin, we'd just see a sharply peaked spectrum around $\nu_{crit}$
with $\nu\ll\nu_{crit}$ going as $\nu^{1\over3}$ and $\nu\gg\nu_{crit}$ going
as $\nu^\hf e^{-{\nu\over\nu_{crit}}}$.  Now let's add more electrons to this
gas.  For a while, the more electrons we add, the more emission we see.
At some point, though, self-absorption starts making a difference, and we
get less emission per electron added.\par
Let's examine the 
peak amplitude of emission (that is, $I_\nu$ at $\nu=\nu_{crit}$).  
To do this,
imagine we have an optically thick ball of blackbody (perfectly absorbing 
and emitting) particles
with a temperature carefully chosen so that $kT\sim\gamma m_e c^2$. Then:
\def\nucrit{{\nu_{crit}}}
$$I_\nu(\nu=\nucrit)=B_\nu(T\sim{\gamma m_ec^2\over k},\nu=\nucrit)$$
Now we'd like to argue that this blackbody particle system is emissively 
identical to
the optically thick synchrotron gas.  To see why this is true, imagine we
overlaid our sphere of blackbody particles on our sphere of synchrotron gas.
Since all particles in it have the same energy, then energy cannot be
transfered between the two system by collisions.  The only option for exchanging
energy is through photons of energy $h\nucrit$.  However, this cannot create
a net flow of energy because (I don't know).\par
If $h\nucrit\ll\gamma m_ec^2$, then:
$$\begin{aligned}I_\nu(\nucrit)^{\e,thick}
&=B_\nu(T\sim{\gamma m_ec^2\over k},\nu=\nucrit)\\ 
&={2kT\over\lambda_{crit}^2}={2\gamma m_ec^2\over\lambda_{crit}^2}\\ 
&=2\gamma m_e\nucrit^2\\ \end{aligned}$$
Since $\nucrit\sim\nu_{cyc}\gamma^2\propto B\gamma^2$, we have that
$\gamma\sim\left({\nucrit\over B}\right)^\hf$.  Thus, for an optically
thick gas:
$$I_\nu\propto\nucrit^{5\over2}B^{-\hf}\propto\nu^{5\over2}B^{-\hf}$$
It is important to remember that each $\gamma$ has a unique corresponding
$\nucrit$.  Also note that optical thickness depends inversely on frequency,
so if we plot $I_\nu$ vs. $\nu$, we get a power law $\nu^{(1+p)\over2}$ for
high frequencies (low optical thickness), and a $\nu^{5\over2}$ dependence
for low frequencies (high optical thickness).  There is also some $\nu_m$ where
$I_\nu$ is maximal.

\subsection*{ Compton Scattering}

Compton scattering is the scattering of a photon off of an electron.  If
the photon {\it loses} energy, this is called {\bf Compton Scattering}.  If
the photon gains energy, this is called {\bf Inverse Compton Scattering}, or
``Compton Up-Scattering''.  The ``comptonization'' of an electron gas is the
gain in energy of an electron gas as the result of a photon gas.\par
Some examples of applications of Compton scattering are:
\begin{itemize}
\item  Compton exchange keeps electrons in thermal equilibrium with
photons at redshifts $z\ge10^3$.
\item  The spectra of AGN and xray binaries are altered by Compton
Scattering (e.g. radio emission to optical wavelengths).
\item  CMB photons get upscattered by galaxy cluster plasma.  This is
called the Sunyaev-Zeldovich effect.
\end{itemize}
The classic model for Compton scattering is where a photon of energy $E=h\nu$
going in the $\^n$ direction scatters off of a stationary $e^-$.  After
scattering, the photon leaves with a new energy $E_1$ in a new direction
$\^n_1$ (at an angle $\phi$ to its original direction $\^n$), and the $e^-$
now moves in direction $\^p$ with energy $E$ and momentum $p$.  Energy
conservation requires that:
$$E+m_ec^2=E_1+E$$
and momentum conservation requires that:
$${E\over c}\^n={E_1\over c}\^n+p\^p$$
Squaring these two equations and subtracting them, we find that:
$$\lambda_1-\lambda={h\over m_ec}(1-\cos\phi)\leftrightarrow
E_1={E\over1+{E\over m_ec^2}(1-\cos\phi)}$$
Comments:
\begin{itemize}
\item $\lambda_1-\lambda>0$: the shift here is tiny.  The maximum possible
value is $\lambda_1-\lambda={2h\over m_ec}=0.04\AA$.  The momentum
tends to be shared between the electron and photon, but no so much the energy.
\item  Remember this is {\it scattering}, not absorption.  Photon \# is
conserved.
\item  Note that for $h\nu\ll m_ec^2$, $\sigma=\sigma_T$.  For 
$h\nu\gg m_ec^2$, $\sigma\sim\sigma_T\left({m_ec^2\over h\nu}\right)$.  This
additional term is called the Klein-Nishna correction.
\item  The mean scattering angle is $\phi={\pi\over 2}$. 
${d\sigma_T\over d\Omega}\propto1+\cos^2\phi$.  Beware that:
$${d\sigma\over d\Omega}\eval{Klein-Nishna}\ne{d\sigma_T\over d\Omega}$$
\end{itemize}
What we've done so far was for a stationary electron.  For a moving electron,
we need to consider the dependence on the angle at which the photon
is coming in with respect to the direction of velocity ($\theta$).  To make
this situation similar to the one we just considered, we need to be in the
frame of the electron.  In this frame, the photons has a new energy as a
result of time dilation:
$$E^\prime=\gamma E(1-{v\over c}\cos\theta)$$
The $v\over c$ term is just the classic Doppler shift of the photon.  Now
suppose that the photon (which entered at angle $\theta^\prime$ in the 
electron frame) rebounds at an angle $\theta_1^\prime$.  To relate this to
our previous derivation, we want to find $\phi$.  So note that 
$\theta_1^\prime-\phi^\prime=\theta^\prime$.  Thus, in the electron's frame:
$$E_1^\prime={E^\prime\over1+{E^\prime\over m_ec^2}(1-\cos\phi^\prime)}$$
Transforming this back into the lab frame:
$$E_1=E_1^\prime\gamma(1+{v\over c}\cos\theta_1^\prime)$$
This generally follows Rybicki \& Lightman.  However, it might help to know
that in R\&L, 7.7b follows from 7.8a, which follows from 7.7a.\par
Comments:
\begin{itemize}
\item  If $E^\prime\ll m_ec^2$, then:
$$\begin{aligned}E_1&\approx E^\prime\gamma(1+{v\over c}\cos\theta_1^\prime)\\ 
&\approx\gamma^2E(1+{v\over c}\cos\theta_1^\prime)(1-{v\over c}\cos\theta)\\ \end{aligned}$$
In a ``typical'' collision, $\theta\sim\theta_1^\prime\sim{\pi\over2}$, so
$E_1\sim\gamma^2E$.
\item  If $E^\prime\gg m_ec^2$, then:
$$\begin{aligned}E_1^\prime&={E^\prime\over1+{E^\prime\over m_ec^2}
(1-\cos\phi^\prime)}\approx m_ec^2\\ 
E_1&=E_1^\prime\gamma(1+{v\over c}\cos\theta_1^\prime)\\ 
&\approx m_ec^2\gamma(1+{v\over c}\cos\theta_1^\prime)\approx\gamma m_ec^2\\ \end{aligned}$$
This final term defines the maximum rebound of the photon.
\end{itemize}

\section*{ Lecture 17 }

\subsection*{ Inverse Compton Scattering}

Recall the following rules for photons bouncing off of relativistic electrons:
$$\begin{aligned}E_1&\sim\gamma^2E\\ 
max(E_1)&\sim\gamma m_ec^2\\ \end{aligned}$$
We'd like next to discuss the behavior of a single $e^-$ swimming through a sea
of photons.  We'd like to know how much power this $e^-$ is going to scatter
by Compton-upscattering these photons.  To order of magnitude, the power
scattered by a single relativistic ($\gamma\gg1$) electron should depend on the 
cross-section for scattering, the speed of the electron ($c$), the \# density 
of photons having various energies, and the energy these photons take from
scattering off of the electron:
$$\begin{aligned}P&\sim\sigma_Tc\int{\eta(E)dE\,E_1(E)}\\ 
&\sim\sigma_Tc\int{\eta(E)dE\,\gamma^2E}\\ 
&\sim\gamma^2c\sigma_T U_{ph}\\ \end{aligned}$$
where $U_{ph}$ is the energy density of the photon field.  This looks very
similar to the power radiated by the synchrotron magnetic field, which is
$P_{synch}\sim U_Bc\sigma_T\gamma^2$.\par
This was an order of magnitude derivation, but we can do better--we just need
to be more careful about how much the photons end up with, versus how
much the electron actually gave to the photons (they had energy to begin with).
We made the assumption that the initial energy of the photons was negligible
because the electron was relativistic.  However, what follows will be true
for any $\gamma$.\par
First, note that the energy bequeathed to the photon bath is given by 
$P_{net}=P_{scat}-P_{incident}$:
$$\begin{aligned}P_{inc}&=c\sigma_T\int{\eta(E)E\,dE}\\ 
&=c\sigma_TU_{ph}\\ \end{aligned}$$
The power scattered {\it out} by the electron should be equal to the 
power radiated by the $e^-$ in its own rest frame, due to accelerations caused
by $\ef^\prime$-fields of photons seen in its rest frame (recall that power
is Lorentz-invariant quantity).  In the $e^-$'s rest frame, the $\bfield$'s of
the photons don't produce accelerations (there's no velocity), so all we
need to consider are the Lorentz-transformed $\ef$'s of the photons:
$$P^\prime={2\over3}{e^2(a^\prime)^2\over c^3}
={2\over3}{e^2\over c^3}\left({e\ef^\prime\over m_e}\right)^2
={2\over3}{e^4\over m_e^2c^3}(\ef^\prime)^2$$
Recall that the Lorentz-transformed $\ef$'s look like:
$$\begin{aligned}E_x^\prime&=E_x\\ 
E_y^\prime&=\gamma E_y-{\gamma v\over c}B_z\\ 
E_z^\prime&=\gamma E_z+{\gamma v\over c}B_y\\ \end{aligned}$$
for an electron moving in the $\^x$ direction.  Substituting these values
into our equation for $P^\prime$:
$$\begin{aligned}\mean{P^\prime}&={2\over3}{e^4\over m_e^2c^3}\left(\mean{E_x^2}+
\gamma^2\mean{E_y^2}+\gamma^2\beta^2\mean{B_z^2}-\underbrace{2\gamma\beta
\mean{E_yB_z}}_{=0}+\gamma^2\mean{E_z^2}+\gamma^2\beta^2\mean{B_y^2}
+\underbrace{2\gamma^2\beta\mean{E_zB_y}}_{=0}\right)\\ 
&={2\over3}{e^4\mean{E_x^2}\over m_e^2c^3}\left(1+2\gamma^2+2\gamma^2\beta^2
\right)\\ 
&=\underbrace{{2\over3}{e^4\over m_e^2c^4}}_{=\sigma_T\over4\pi}
{c\over3}\underbrace{3\mean{E_x^2}\over4\pi}_{=U_{ph}}
4\pi(1+2\gamma^2+2\gamma^2\beta^2)\\ 
&=\sigma_TcU_{ph}{1\over3}(2\gamma^2+2(\gamma^2-1)+1)\\ \end{aligned}$$
Recall that $\gamma^2\beta^2=\gamma^2-1$, so $\mean{P^\prime}=\sigma_T
cU_{ph}{1\over3}(4\gamma^2-1)$, and our net power scattered is:
$$\boxed{P_{net}={4\over3}\sigma_TU_{ph}c\beta^2\gamma^2}$$
This is an exact expression.  Note that if $\gamma\gg1$ then the average photon
energy of Compton-upscattered radiation is:
$$\begin{aligned}\mean{h\nu}_1&={P\over{U_{ph}\over\mean{h\nu}}\sigma_Tc}\\ 
&=\mean{h\nu}{4\over3}\gamma^2\\ \end{aligned}$$
Another note: Compare our expression for the Compton-upscattering radiation
to the power radiated by a single electron undergoing synchrotron radiation:
$$P_{sync}=2\sigma_TcU_B\beta^2\gamma^2\mean{\sin^2\alpha}$$
The only difference (ignoring the $\sin^2\alpha$), is the exchange of
$U_B$ for $U_{ph}$.\par

\subsection*{ Synchrotron Self-Compton (SSC)}

Electrons undergoing synchrotron radiation create a photon bath which
other electrons will then interact with via inverse Compton scattering.  Recall
that for original (unprocessed) synchrotron radiation, that $F_\nu$, between
some minimum and maximum frequency cut-off, goes as $K\nu^\alpha$, and that
the number of photons per $\gamma$ is ${dN\over d\gamma}=N_0\gamma^s$, where
$\alpha={1+s\over2}$.  These frequency cut-offs were set by $\gamma_{min}^2
\nu_{cyc}$ and $\gamma_{max}^2\nu_{cyc}$.  After this radiation is processed
by SSC, approximately every photon is upscattered to a new energy
${4\over3}\gamma^2\nu$.  We are assuming that the relationship between
an incoming photon frequency and it's final frequency are related via a
delta function.  Thus:
\def\tn{{\tilde\nu}}
$$F_{\nu,SSC}(\nu)=\tau\int_{\tn}{K\tn^\alpha d\tn\delta\left(\tn-
{\nu\over\gamma^2}\right)\int_\gamma{N_0\gamma^sd\gamma}}$$
Keep in mind that $N_0$ is normalized to so the integral comes out to 1
(it just accounts for the 
``shape'' of the energy distribution function). $\tau$ is what contains the actual
\# density of $e^-$'s.  It is the fraction scattered,
and is generally $\ll1$. $\nu\sim\tn\gamma^2$.\par
For a fixed $\nu\sim\tn\gamma^2$, we find that $\gamma\sim\left({\nu\over
\tn}\right)^\hf\propto\tn^{-\hf}$.

\section*{ Lecture 18 }

\def\numin{{\nu_{min}}}
\def\numax{{\nu_{max}}}
\def\gamin{\gamma_{min}}
\def\gamax{\gamma_{max}}
\def\gul{{min(\sqrt{\nu\over\numin},\gamax)}}
\def\gll{{max(\sqrt{\nu\over\numax},\gamin)}}
Recall in deriving the interaction of synchrotron radiation with
synchrotron electrons, we derived the following formula for flux:
$$F_{\nu,SSC}(\nu)=\tau\int_{\tn}{K\tn^\alpha d\tn\delta\left(\tn-
{\nu\over\gamma^2}\right)\int_\gamma{N_0\gamma^sd\gamma}}$$
Now the integral over $\gamma$ is along ``slant paths'' through
the rectangle ($\gamin\to\gamax$, $\numin\to\numax$).
Some of these slant paths will not stretch all the way to $\gamax$
or $\gamin$ because of the boundaries imposed by $\numin$ and
$\numax$.  So we need to be a little more precise about the bounds
on the $\gamma$ integral:
$$\begin{aligned}F_{\nu,SSC}(\nu)
&\sim\tau\int_\numin^\numax{K\tn^\alpha\delta(\tn-{\nu\over\gamma^2})d\tn
\int_\gll^\gul{N_0\gamma^sd\gamma}}\\ 
&\sim\tau K\left({\nu\over\gamma^2}\right)^\alpha
\int_\gll^\gul{N_0\gamma^sd\gamma}\\ 
&\sim\tau K\nu^\alpha N_0
\int_\gll^\gul{{\gamma^s\over\gamma^{2\alpha}}d\gamma}\\ \end{aligned}$$
But since $\alpha={1+s\over2}$, the integrand is simply $\inv{\gamma}$,
giving us:
$$\boxed{F_{\nu,SSC}(\nu)\sim\tau k\nu^\alpha N_0\ln\left({\gul\over\gll}\right)}$$
Recall that $N_0$ was normalized so that $\int{N_0\gamma^sd\gamma}=1$, so
saying that $\gamax$ is just some multiple of $\gamin$, it must be that
$$N_0\sim\gamin^{-1-s}$$
Note for $\nu\sim\gamin^2\numin$, $F_\nu$ looks like (using the above 
relationship):
\def\gllogul{\left({\gul\over\gll}\right)}
$$\begin{aligned}F_{\nu,SSC}&=\tau K\gamin^{2\alpha}\numin^\alpha N_0\ln\gllogul\\ 
&=\tau K\numin^\alpha\gamin^{2\alpha}\gamin^{-1-s}\ln\gllogul\\ 
&=\tau K \numin^\alpha\ln\gllogul\\ \end{aligned}$$

\subsection*{ Compton Catastrophe}

If you keep scattering the same electrons, as in Synchrotron Self-Compton,
there is a danger, if things are dense enough, of a runaway amplification
of radiation energy density, or a ``Compton Cooling Catastrophe''.  However,
we've never seen anything with a brightness temperature of $10^{12}K$.  
What sets this ``inverse Compton limit'' at this temperature?  Comparing,
for a single electron,
the luminosity of inverse Compton scattering to synchrotron scattering:
$${L_{IC}\over L_{sync}}={{4\over3}\beta^2\gamma^2\sigma_TcU_{ph}\over
{4\over3}\beta^2\gamma^2\sigma_TcU_B}={U_{ph}\over U_B}
\begin{cases} >1&catastrophe\\ <1 &no\ catastrophe\end{cases}$$
Now we're going to make an approximation that we are on the Rayleigh-Jeans
side of the blackbody curve, so that:
$$\begin{aligned}U_{ph}=U_{ph,sync}&\propto\nu_mI_\nu(\nu_m)\\ 
&\propto\nu_m{2kT_B\over\lambda_m^2}\\ 
&\propto\nu_m^3T_B\\ \end{aligned}$$
where $\nu_m$ is the frequency of peak of synchrotron emission. 
Now $U_B\propto B^2$ is pretty obvious:  
$$\nu_m\sim\gamma_m^2\nu_{cyc}\propto\gamma_m^2B$$
where this $\gamma_m$ is not $\gamax$.  Making the approximation that we
are in the optically thick synchrotron spectrum, so that $\gamma m_ec^2\sim
kT$, then we get $\nu_m\sim T_B^2B$.  We can say that the kinetic temperature
is the brightness temperature because we are talking about the average kinetic
energy of the electrons generating the synchrotron radiation with a particular
brightness temperature (i.e. another frequency of synchrotron radiation will
have another brightness temperature, and another set of electrons moving
with a different amount of kinetic energy). Thus,
$${U_{ph}\over U_B}=C{\nu_m^3T_B\over\nu_m^2}T_B^4
=\left({\nu_m\over10^9Hz}\right)\left({T_B\over10^{12}K}\right)^5=1$$
A way of think about this is that, in order to avoid having infinite energy
in this gas of electrons, there has to be a limit on the brightness 
temperature (which is determined by the density of electrons).  This is a
self-regulating process--if the brightness temperature goes too high, an
infinite energy demand is set up, knocking it back down.

\subsection*{ The Sunyaev-Zeldovich Effect: Compton Y-Parameter}

Consider a non-relativistic thermal bath of electrons at temperature $T_e$.
The average kinetic energy of these electrons is ${3\over2}kT_e$.  Now
suppose that there is also a bath of photons which all have
energy $E=h\nu\ll kT_e$.  Putting this cold bath of photons in with
the hot (but non-relativistic) electrons, we find that they get up-scattered.
After first scattering, the mean shift in energy is:
$$\begin{aligned}\Delta\_E&={P_{K,single\ e^-}\over n_{photons}\sigma_Tc}=
{{4\over3}U_{ph}\gamma^2\beta^2\sigma_Tc\over n_{photons}\sigma_Tc}=
{4\over3}\gamma^2\beta^2E\\ 
{\Delta\_E\over E}&={4\over3}\gamma^2\beta^2={4\over3}{v^2\over c^2}=
{4\over3}{3kT_e\over m_ec^2}={4kT_e\over m_ec^2}\\ \end{aligned}$$
which is much the mean fractional change in photon energy after 1 scattering.

\section*{ Lecture 19 }

\subsection*{ More Sunyaev-Zeldovich Effect}

Last time, we'd written down the mean energy of the upscattered photons:
$$\Delta\_E={P_{Compton,single\ e^-}\over n_{photons}\sigma_Tc}$$
The bottom term $n_{photons}\sigma_Tc$ is just the collision rate for
photons. We used this to calculate:
$${\Delta\_E\over E}={4kT_e\over m_ec^2}\ll1$$
Thus, after a single scattering:
$$\_E_i=E\left(1+{4kT_e\over m_ec^2}\right)$$
So after $N$ scatterings:
$$\_E_N=E\left(1+{4kT_e\over m_ec^2}\right)^N$$
In the limit of $N\to\infty$:
$$E_N=Ee^{4kT_eN\over m_ec^2}\equiv Ee^{4y}$$
This is the definition of the Compton $y$ parameter: $y\equiv {kT_e\over m_e
c^2}N$.\par
Suppose we have a cloud of electrons that has radius $l$, and the optical
depth out of the cloud $\tau_e\gg1$.  The mean number of collisions a photon 
will undergo in getting out of the cloud is $N\sim\tau_e^2$.  This is because
each time step between collisions is ${\lambda_{mfp}\over c}$, and the number
of steps to get out is:
$${Time\ to\ get\ out\over Each\ time\ step}\sim{{l^2\over c\lambda_{mfp}}
\over {\lambda_{mfp}\over c}}\sim\left({l\over\lambda_{mfp}}\right)^2
\sim\tau_e^2$$
\def\ppt#1{{\partial #1\over\partial t}}
The reason why the time to get out is $l^2\over c\lambda_{mfp}$, is because
the diffusion equation gives us $\ppt{Q}=D\nabla^2Q$, which by dimensional
analysis, says:
$$\inv{T}\sim{D\over L^2}$$
$D$ is the diffusivity (viscosity) of the medium.\par
If $\tau_e\ll1$, $N_s\sim\tau$ because the intensity goes as $e^{-\tau}\approx
1-\tau$.  Generally:
$$\begin{aligned}N_s&\sim max(\tau_e,\tau_e^2)\\ 
&\sim\tau_e(1+\tau_e)\\ \end{aligned}$$
Thus, the Compton $y$ parameter is:
$$y\approx {kT_e\over m_ec^2}\tau_e(1+\tau_e)$$
\begin{itemize}
\item We'll apply this to CMB photons traveling through an intra-cluster
gas.  We'll say: $n_e\sim3\e{-3}cm^{-3}$, $\sigma_T\sim10^{-24}cm^2$, and 
$l\sim10^6\cdot3\e{18}cm$.  This gives us $\tau_e\sim10^{-2}$.  Now 
$kT_e$ should be a few $keV$, so $${kT_e\over m_ec^2}\sim10^{-2}$$
The us the energy post-traversal of the cluster is related to the initial
energy of the photons by:
$$E_{post}=E_{init}e^{4y}=E_{init}(1+4\e{-4})$$
Thus, we expect to see a shift in the frequency of peak flux when looking
through a cluster vs. around one.  We can calculate the change in brightness
temperature this causes by using:
$$\begin{aligned}F_{\nu,before}&={2kT_{CMB}\over\lambda^2}\\ 
F_{\nu,after}\left(\lambda={\lambda_0\over1+4y}\right)
&={2kT_{CMB}\over\lambda^2}\\ 
&={2kT_{CMB}^\prime\over\lambda_0^2}\\ \end{aligned}$$
Solving for $T_{CMB}^\prime$:
$$\begin{aligned}T_{CMB}^\prime&=T_{CMB}\left({\lambda\over\lambda_0}\right)^2\\ 
&=T_{CMB}\left({1\over1+4y}\right)^2\approx T_{CMB}(1-8y)\\ \end{aligned}$$
So the change in brightness temperature is $\Delta T_{CMB}\sim8yT_{CMB}\sim
10mK$.  This is hard to detect from ground-based telescopes, but this is
a much greater effect than the inherent anisotropies of the CMB (which we've
detected), so space-based telescopes like WMAP, if they have the angular
resolution, should have already detected this.  Yet they haven't.\par
Now we've been a little sloppy.  We've said that every photon from the CMB
underwent a small shift in energy, but for an optically thin cloud like the
one we've been discussing, only a few photons are ever scattered.  We'll examine
this on the Rayleigh-Jeans tail.\par
For any $F_\nu$ incident upon our cloud, the $F_{\nu,after}$ which results
from unscattered photons should be less (there are fewer photons).  Then the
flux from the upscattered photons gets added back in.  We'll choose some
frequency $\nu_0$ which scatters into $\tn=\nu_0\left(1+{4kT_e\over m_ec^2}
\right)$, so that:
$$\begin{aligned}F_{\nu,after}(\tn)
&={2kT_{CMB}\over c^2}\tn^2-{2kT_{CMB}\over c^2}\tn^2\tau_e
+{2kT_{CMB}\over c^2}\nu_0^2\tau_e\\ 
&={2kT_{CMB}\over c^2}
{\tn^2\over\left(1+{4kT_e\tau_e\over m_ec^2}\right)^2}\\ \end{aligned}$$
This matches what we estimated just by shifting, so we're okay.\par

\item We'll examine another situation where we can use the SZ effect
to measure $H_0$ in an x-ray cluster.  We'll say that $\theta_c$ is the 
observed angular size of the cluster, and $F_x$ is the observed x-ray flux.
We'll say there's some observed redshift (using optical data), which is 
related to the recession velocity $v$.  The x-ray luminosity $L_x=
F_x\cdot4\pi d^2$, and the recessional velocity is $v=Hd$.  Thus:
$$L_x\propto F_xv^2H^{-2}$$
We also know, from bremsstrahlung that $L_x\propto n_e^2r_c^3F(T_e)$, where
$F(T_e)$ is some function of the temperature of the electrons, and $r_c$ is 
the radius of the cluster.  Now we can express $r_c$ as:
$$r_c\sim\theta_cd\sim\theta_c{v\over H}$$
Setting our two expressions for the luminosity equal:
$$\begin{aligned}F_xv^2H^{-2}&\propto n_e^2r_c^3F(T_e)\\ 
&\propto n_e^2\theta_c^3{v^3\over H^3}F(T_e)\\ \end{aligned}$$
which gives us that $n_e\propto H^\hf$.  Now SZ tells us that $y\propto T_en_e
r_c\propto n_er_c\propto H^\hf\theta_c\cdot d{v\over H}$.  Thus:
$$y\propto \inv{H^\hf}$$
\end{itemize}

\subsection*{ Zeeman Effect}

The Zeeman Effect concerns the splitting of electronic levels in a magnetic
field.  We've already talked a little bit about this in hyperfine 
splitting, which was caused by magnetic fields intrinsic to an atom.  We find
that a single line can split into $\sim3-27$ components, depending on the 
number of combinations of $\vec L$ and $\vec S$ there are in the atom.  The
strengths of these various lines depend on viewing geometry, and can be
polarized.  The change in energy between previously degenerate states set up
by an external $B$ field is:
$$\begin{aligned}\Delta E&\sim\mu B\sim{e\hbar\over m_ec}B\\ 
&\sim{hc\over\lambda_1}-{hc\over\lambda_2}\\ 
&\sim{hc\over\lambda}{\Delta\lambda\over\lambda}\sim{e\hbar B\over m_ec}\\ \end{aligned}$$
Thus the fractional change in wavelength is:
$$\boxed{{\Delta\lambda\over\lambda}\sim\lambda{eB\over2\pi m_ec^2}}$$
In practice, we find that the various split components of the original
absorption lines are hard to resolve, and we see the effect expressed mostly
as a broadening of the original line.

\section*{ Lecture 20 }

\subsection*{ Faraday Rotation}

Faraday Rotation is the rotation of the axis of polarization of radiation
while propagating through a magnetized plasma.  We'll derive this by examining
the equation of motion of an $e^-$ in plasma.  A long time ago, we wrote the
equation of motion of electrons in plasma when subjected to a plane $\ef$ wave:
\def\vv{{\vec v}}
$$m\vv=-e\ef-{e\over c}\vv\times\bfield$$
We had thrown away the $\bfield$ dependency because we hadn't imposed 
and $\bfield$ on the plasma, but this time, we'll examine what happens if we
don't throw it away, but instead say that $\bfield=\bfield_0+\bfield_{rad}$,
where $\bfield_0$ is an externally imposed field.  Then
$$m\.\vv=-e\ef-{e\over c}\vv\times\bfield_0$$
We'll just guess the solution:
$$\ef=Re\left(E_0(\^x\mp i\^y)e^{i(k\mp z-\omega t)}\right)$$
where the '-' of $\mp$ corresponds to a right-circularly-polarized wave, and
'+' corresponds to left-circular-polarization.  For the magnetic field, if
$\bfield_0=B_0\^z$, then we need to solve the equation:
$$m\.\vv=-eE_0(\^x\mp i\^y)e^{i(k\mp z-\omega t)}
-{eB_0\over c}(-v_x\^y+v_y\^x)$$
This leaves us with equations for $\.v_x$ and $\.v_y$.  Instead of doing all
the algebra, we'll just trust that the following is a solution:
$$\boxed{\vv={-ie\ef\over m(\omega\pm\omega_{cyc})}}$$
Recall that $\omega_{cyc}={eB_0\over mc}$.  Now we'll solve for the current
in the plasma:
\def\ocyc{\omega_{cyc}}
$$\vec j=-n_ee\vv={-n_ee(-ie)\over m(\omega\pm\ocyc)}\ef\equiv\sigma\ef$$
where $\sigma={in_ee^2\over m(\omega\pm\ocyc)}$ is the conductivity.  Now let's
solve Maxwell's equations for a perturbing wave through the plasma 
(this is very similar to what we did months ago...).  Note that in the following
equation, $\bfield=\bfield_0+\bfield_{rad}$, but $\ppt{\bfield_0}=0$:
\def\erad{{\ef_{rad}}}
\def\brad{{\bfield_{rad}}}
\def\jv{{\vec j}}
\def\div{{\vec\nabla}}
$$\begin{aligned}\div\times\erad&={-1\over c}\ppt{\bfield}\\ 
\div\times\bfield&={4\pi\over c}\jv+\inv{c}\ppt{\ef}\\ \end{aligned}$$
Note also that $\div\times\bfield_0=0$.  Now we can solve for the index of
refraction:
$$\begin{aligned}\eta^2=\left({ck_\mp\over\omega}\right)^2
&=1-{4\pi\sigma\over i\omega}\\ 
&=1-{4\pi in_ee^2\over i\omega m(\omega\pm\ocyc)}\\ 
&=1-{\omega_p^2\over\omega(\omega\pm\ocyc)}\\ \end{aligned}$$
Thus, $k_-\eval{\omega}\ne k_+\eval{\omega}$.  If we put a linearly polarized
wave through a magnetized plasma, the left-circularly-polarized component will
evolve more quickly than the right-circularly-polarized component, by a
factor of $\Delta\phi$, so when the wave emerges from the plasma it will be
fixed into a new phase where $\Delta\theta={\Delta\phi\over2}$.  If we wanted
to solve for this in terms of a distance through the plasma, we'd write:
$$\Delta\theta={\Delta\phi\over2}=\hf\int_0^d{(k_--k_+)ds}$$
where $\left({ck_\mp\over\omega}\right)^2=1-{\omega_p^2\over\omega(\omega\pm
\ocyc)}$.  We'll do two approximations of this equation:
\begin{itemize}
\item $\ocyc\ll\omega$.  Then $\ocyc=3MHz\left({B_0\over ??}\right)$.
\item $\omega\gg\omega_p$.  Then:
$$\begin{aligned}\left({ck_\mp\over\omega}\right)^2
&=1-{\omega_p^2\over\omega^2}(1\mp{\ocyc \over\omega})\\ 
k_\mp&={\omega\over c}
\sqrt{1-{\omega_p^2\over\omega^2}(1\mp{\ocyc\over\omega})}\\ 
&={\omega\over c}
\left(1-{\omega_p\over2\omega^2}(1\mp{\ocyc\over\omega})\right)\\ 
k_--k_+&={\omega\over c}{\omega_p^2\ocyc\over\omega^3}\\ 
&={\omega_p^2\ocyc\over c\omega^2}\\ \end{aligned}$$
\end{itemize}
Getting back to $\Delta\theta$:
$$\begin{aligned}\Delta\theta&=\hf\int_0^d{{\omega_p^2\ocyc\over c\omega^2}ds}\\ 
&={2\pi e^3\over c^2m_e^2\omega^2}\underbrace{\int_0^d{n_eB_\|ds}}_{rotation\ 
measure}\\ \end{aligned}$$
In practice, if we want to figure out the effect of a magnetized plasma on
radiation, we need to look at the phase change for two different frequencies 
from a linearly polarized source:
$$\Delta\theta\eval{\omega_1}-\Delta\theta\eval{\omega_2}
={2\pi e^3\over c^2m_e^2}RM\left(\inv{\omega_1^2}-\inv{\omega_2^2}\right)$$
where $RM$ is the rotation measure.
Some sources of linearly polarized radiation are pulsars, AGN, and the
galactic radio synchrotron from cosmic rays.  A magnetized plasma, if it
has different rotation measures over different paths from a source to us, can
depolarize a source which was originally polarized.  This is {\it Faraday
De-Polarization}.

\subsection*{ Evolution of 2-Stream Propagation}

Rybicki \& Lightman, CH1, gives us the following 3-dimensional 
integro-differential equation for the evolution of specific intensity through
an adsorptive, scattering medium:
\def\nabs{{\nu,abs}}
\def\nscat{{\nu,scat}}
\def\nemis{{\nu,emis}}
$${dI_\nu\over ds}=-(K_\nabs+K_\nscat)\rho I_\nu+j_\nu$$
where $j_\nu=j_\nscat+j_\nemis$, and $j_\nscat(\^s)$ is given by:
$$j_\nscat(\^s)=\rho K_\nscat\times\int_{4\pi}{I_\nu(\^n)p(\^s,\^n)d\Omega}$$
where $p(\^s,\^n)$ is proportional to the probability that $\^n$-photons get
redirected to $\^s$-photons, and $\int_{4\pi}{p(\^s,\^n)d\Omega}=1$.  Finally,
we have:
$$j_\nemis\equiv S_\nemis(\^s)\rho K_\nabs$$
So we have ourselves a 3-D Integro-differential equation.  If we are going to
be able to actually solve this, we're going to have to make some assumptions.
We'll make the following 4 idealizations:
\begin{itemize}
\item{(A)} We have isotropic scattering.  Then $p(\^s,\^n)=\inv{4\pi}$, and
the mean intensity is given by:
$$J_\nu\equiv\inv{4\pi}\int{I_\nu d\Omega}$$
which gives us:
$${dI_\nu\over\rho(K_\nabs+K_\nemis)ds}=-I_\nu+\~\omega_\nu J_\nu+(1-
\~\omega_\nu)S_\nu(\^s)$$
where $\~\omega_\nu\equiv{K_\nscat\over K_\nscat+K_\nabs}$ is the {\it
single-scattering albedo}, and $0<\~\omega_\nu<1$.
\item{(B)} $S_\nu(\^s)=B_\nu(T)$ (i.e. Local Thermal Equilibrium).
\item{(C)} We have a plane-parallel atmosphere.  We'll say that parallel
surfaces of constant temperature.  If $\^z$ is the direction straight up
through the atmosphere, and $\theta$ is the angle of a vector from the $\^z$
direction, the we define $\mu\equiv\cos\theta$.  Then $ds={dz\over\cos\theta}
={dz\over\mu}$.  Note that the change in optical depth along $\^z$ is:
$$d\tau_\nu=-(K_\nabs+K_\nemis)\rho\,dz$$
\item{(D)} No scattering.  That is, $\~\omega_\nu=0$.  We'll relax this
eventually (otherwise it would have been silly to assume A).  \par
\end{itemize}
So the final form of our equation is:
$$\boxed{\mu{dI_\nu\over d\tau_\nu}=I_\nu-B_\nu(T(z))}$$
We'll solve this by applying the {\it Eddington 2-stream Formalism}.
We begin by taking a moment $\int{equation\,d\Omega}$ of 
the above:
\def\ddtau#1{{d#1\over d\tau_\nu}}
$$\ddtau{}\int{\mu I_\nu(\mu)d\Omega}=\int{I_\nu d\Omega}-\int{B_\nu d\Omega}$$
Using that $d\Omega=-2\pi d\mu$:
$$\ddtau{}\int_1^{-1}{\mu I_\nu(-2\pi d\mu)}=4\pi J_\nu-4\pi B_\nu$$
Using that $2\pi\int_{-1}^1{\mu I_\nu d\mu\equiv F_\nu}$
(Rybicki \& Lightman 1.3b), we have:
$$\ddtau{F_\nu}=4\pi(J_\nu-B_\nu)$$
Now Eddington's 2-stream approximation was to say that for a point in our
atmosphere, the only important components of the specific intensity are
those from above and those from below.  That is, we need only consider
the components:
$$\begin{aligned}I_\nu^+(\tau_\nu)&=I_\nu(\tau_\nu,0<\mu<1)\\ 
I_\nu^-(\tau_\nu)&=I_\nu(\tau_\nu,-1<\mu<0)\\ \end{aligned}$$

\section*{ Lecture 21 }

\subsection*{ Monochromatic Radiative Equilibrium}

Last time we had the integro-differential equation:
$$\mu{dI_\nu\over d\tau_\nu}=I_\nu-B_\nu(T(z))$$
We got rid of the ``integro'' part, but then we did an integral over $d\Omega$
to get the angle-averaged $I_\nu$, so we brought back our ``integro'' when we
said:
\def\inp{{I_\nu^+}}
\def\inm{{I_\nu^-}}
$${dF_\nu\over d\tau_\nu}=4\pi(J_\nu-B_\nu)$$
Now recall our definitions of $\inp, \inm$:
$$\begin{aligned}\inp(\tau_\nu)&=I_\nu(\tau_\nu,0<\mu<1)\\ 
\inm(\tau_\nu)&=I_\nu(\tau_\nu,-1<\mu<0)\\ \end{aligned}$$
Recall that $\mu\equiv\cos\theta$, but keep in mind that $\theta$ measures the
angle of the {\it incoming flux intensity}, which is generally the opposite 
direction
from which we are looking.  Thus, $\inp$ is flux intensity coming from 
{\it below}.\par Now the angle-averaged flux $J_\nu$ is given by:
$$J_\nu=\int{I_\nu d\Omega}=\hf(\inp+\inm)$$
Likewise, our flux is:
$$F_\nu=\int{I_\nu\cos\theta d\Omega}
=2\pi\int_0^{\pi\over2}{\inp\cos\theta\sin\theta d\theta} +
2\pi\int_{\pi\over2}^{\pi}{\inm\cos\theta\sin\theta d\theta}
=\pi(\inp-\inm)$$
Now let's calculate our second moment, $\int{equation\cdot\mu d\Omega}$:
$$\begin{aligned}LHS&=\ddtau{}\int{\mu^2I_\nu d\Omega}
=\ddtau{}\left[2\pi\int_{-1}^0{\mu^2\inp d\mu}+2\pi\int_0^1{\mu^2\inm d\mu}
\right]
={2\pi\over3}\ddtau{}(\inp+\inm)\\ 
&={4\pi\over2}\ddtau{}J_\nu\\ \end{aligned}$$
The right-hand side of the equation is:
$$RHS
=\int{\mu I_\nu d\Omega}-\underbrace{\int{\mu B_\nu d\Omega}}_{0}
=\int{\mu I_\nu d\Omega}
=F_\nu$$
where the last step was taken with the aid of Rybicki \& Lightman Eq. 1.3b. 
So our full second moment equation is:
$${4\pi\over3}\ddtau{}J_\nu=F_\nu$$
Recall that our first moment equation was:
$$\ddtau{F_\nu}=4\pi(J_\nu-B_\nu)$$
We may rewrite this as:
$$J_\nu=\inv{4\pi}\ddtau{F_\nu}+B_\nu$$
So our first and second moment equations together give us:
$$\boxed{{4\pi\over3}\ddtau{}
\left[\inv{4\pi}\ddtau{F_\nu}+B_\nu\right]=F_\nu}$$
Our third moment equation, because we are in {\bf Monochromatic Radiative
Equilibrium} is:
$$\ddtau{F_\nu}=0$$
Thus, $F_\nu$ is constant.  Now $\ddtau{B_\nu}=F_\nu=constant$, so $B(T(z))$
scales linearly with $\tau$.  Now $\tau=0$ corresponds the the top
of the atmosphere, and we'd expect that the temperature there should be
$0$, but this is incorrect.  Likewise $\tau_{max}$ corresponds to the
full atmospheric depth, i.e. the ground.  We'd expect that the temperature
there should just be the temperature of the ground $T_g$.  This is also
incorrect.  This is because there is an interface of conductivity at the
surface of the ground.  We'll call $T_1$ the temperature of the air at the
base of the atmosphere, and try to see if get $B_\nu(T_1)$ in terms of 
$B_\nu(T_g)$.  
First, we need to get an expression for $J_\nu$ in terms of 
and $F_\nu$.  Using that $J_\nu=\hf(\inp+\inm)$ and $F_\nu=\pi(\inp-\inm)$
$$\boxed{J_\nu=\inm+{F_\nu\over2\pi}=\inp-{F_\nu\over2\pi}}$$
Now since $\ddtau{F_\nu}=0=4\pi(J_\nu-B_\nu)$, $J_\nu=B_\nu$ everywhere.
Specifically, in the atmosphere right above the ground, $J_\nu=B_\nu(T_1)$, so:
$$J_\nu=\inp\eval{surface}-{F_\nu\over2\pi}\eval{surface}$$
Now $\inp\eval{surface}=B_\nu(T_g)$, so:
$$B_\nu(T_1)=B_\nu(T_g)-{F_\nu\over2\pi}$$
So we have shown why $B_\nu(T_1)\ne B_\nu(T_g)$.  We should also discuss why,
at the top of the atmosphere, $T_0\ne 0$.  For this, we'll write:
$$\begin{aligned}J_\nu=B_\nu(T_0)
&=\underbrace{\inm\eval{altitude}}_{0}+{F_\nu\over2\pi}\\ 
B_\nu(T_0)&={F_\nu\over2\pi}\\ \end{aligned}$$
Thus, $T_0\ne0$.  So we could solve for $F_\nu$ and find that it is a function
of $\tau$, but solving it for the fairly contrived case of MRE (Monochromatic
Radiative Equilibrium) doesn't give us anything very meaningful.  Instead we'll
solve it for BRE (Bolometric Radiative Equilibrium).

\subsection*{ Bolometric Radiative Equilibrium}

Let's start again with our equation:
$$\mu\ddtau{I_\nu}=I_\nu-B_\nu$$
We can rewrite this in terms of atmospheric height $z$ (recall that as
$z\to\infty$, $\tau\to0$, so:
$$\mu{dI_\nu\over\rho\kappa_\nu dz}=-I_\nu+B_\nu$$
Then the first moment of this equation is:
$${\mu\over\rho}{d\over dz}\int{{I_\nu\over\kappa_\nu}d\nu}=-I+B$$
where $I\equiv\int{I_\nu d\nu}$ and $B\equiv\int{B_\nu d\nu}$.  Now we'll make
the (dubious) approximation that $\kappa_\nu=\kappa$ does not depend on
frequency (``Grey atmosphere'').  This gives us:
$${\mu\over\rho\kappa}{dI\over dz}=-I+B=-{dI\over d\tau}$$
Now we make the assumption of Bolometric Radiative Equilibrium, so that:
$${4\pi\over3}{dB\over d\tau}=F=constant$$
Recall that $B=\int{B_\nu d\nu}={\sigma T^4\over\pi}$, so BRE is telling us:
$${4\pi\over3}\ddtau{}\left({\sigma T^4\over\pi}\right)=F$$
Or rewriting this:
$$\boxed{\sigma T^4=\sigma T_b^4+{3F\over 4}\tau}$$
At infinite altitude, $J\equiv\int{J_\nu d\nu}=B$, so using that:
$$\begin{aligned}J&=I^-+{F\over2\pi}\\ 
&=I^+-{F\over2\pi}\\ \end{aligned}$$
we have:
$$J=B=\underbrace{I^-\eval{altitude}}_{0}+{F\over2\pi}$$
Therefore:
$$F=2\pi B\eval{altitude}=2\pi{\sigma T_0^4\over\pi}=2\sigma T_0^4$$
Thus, our BRE equation gives us:
$$\boxed{\sigma T^4=\sigma T_0^4\left(1+{3\over2}\tau\right)}$$
Saying $F=2\sigma T_0^4\equiv\sigma T_e^4$, where $T_e$ is an ``effective
temperature'', then:
$$T_0=\inv{\root 4\of{2}}T_e$$
This is apparently a classical result.  Let's do an example by calculating
the effective temperature of the Earth. $F\eval{earth}$ is given by:
$$4\pi R_\oplus^2F=(1-\~\omega_{eff}){L_\odot\over4\pi d^2}\pi R^2$$
$\~\omega_{eff}$ is a measure of how much of the sun's energy we get.  We'll
say that, since it's cloudy about a third of the time, $\~\omega\sim0.3$.
Plugging in the numbers, we find that $T_e=258K$, so $T_0=217K$.  This is
indeed about the mid-latitude temperature of air in the troposphere.  Now
temperature scales with optical depth by:
$$\begin{aligned}T^4(\tau)&=T_0^4\left(1+{3\over2}\tau\right)\\ 
T&=T_0\tau^{1\over4}\\ \end{aligned}$$
This is an expression of the greenhouse effect.

\section*{ Lecture 22 }

\subsection*{ Radiative Diffusion}

Recall that we had, using Bolometric Radiative Equilibrium, an equation which
described the greenhouse effect:
$$\sigma T^4=\sigma T_0^4\left[1+{3\over2}\tau\right]$$
Now we want to talk about the effects of the diffusion of photons.  For this,
we have the general diffusion equation:
$$F=-D\nabla n$$
For photons, $F$ is the energy flux, $D$ is $\lambda_{mfp}\cdot c$, 
and $n\sim{\sigma\over c}T^4$ is the number density of photons.  Then:
$$F\sim\underbrace{\lambda_{mfp}\over L}_{1\over\tau}c{\sigma\over c}T^4
\sim{\sigma T^4\over\tau}$$
Recall that $F\equiv\sigma T_e^4$, so:
$$T^4\sim T_e^4\tau$$
This says that as we go deeper into the atmosphere, the temperature increases,
but slowly (as the fourth root).  

\subsection*{ Reintroducing the Scattering Term}

A while ago, we had the equation:
$$\mu{dI_\nu\over d\tau_\nu}=I_\nu-\~\omega_\nu J_\nu-(1-\~\omega_\nu)B_\nu$$
and we decided to simplify our lives, we had set the scattering term
$\~\omega_\nu=0$.  Now we'd like to solve the compliment problem: we'll set
$B_\nu=0$ and look at the equation for pure scattering:
$$\mu{dI_\nu\over d\tau_\nu}=I_\nu-\~\omega_\nu J_\nu$$
The first thing we'll do to simplify our integro-differential equation is to
say that scattering preserves frequency (which it does, except for Compton
scattering).  Then we have:
\def\wt{{\~\omega}}
$$\mu\ddtau{I_T}=I_T-{\wt\over4\pi}\int{I_Td\Omega}$$
where we say the total intensity $I_T(\tau,\mu)=I_*(\tau,\mu)+I(\tau,\mu)$.
$I_*$ is the incident (never scattered) intensity, and $I$ is the diffuse
(scattered at least once) intensity.  Our equation for the incident component
is:
$$\mu\ddtau{I_*}=I_*$$
This is a familiar equation with a familiar solution:
$$I_*=I_*(\tau=0)e^{\tau\over\mu}$$
You'd expect that $I_*$ should decrease due to scattering, so why is the
exponent positive in this equation?  Remember by our convention, that
$\mu$ points toward the direction of incoming flux, so a positive $\mu$ 
corresponds to looking upward in the atmosphere, in which direction the
intensity of unscattered light should increase.  To make our equation look
like the solution in Chamberlain, we define:
$$I_*(\tau=0)\equiv\pi\mathfrak{F}\delta(\mu+\mu_0)\delta(\phi-\phi_0)$$
$$\boxed{I_*=\pi\mathfrak{F}\delta(\mu+\mu_0)\delta(\phi-\phi_0)}$$
where $\mu_0$ is the angle the incident intensity makes with the plane of
the atmosphere, and $\phi$ measures ``right ascension''.\par
Now the whole diffusion equation looks like:
$$\begin{aligned}\mu\ddtau{I}&=I-{\wt\over4\pi}\int{I_*+I)d\Omega}\\ 
&=I-{\wt\over4\pi}\int{Id\Omega}-{\wt\over4\pi}\int{I_*d\Omega}\\ 
&=I-{\wt\over4\pi}\int{I\overbrace{d\Omega}^{d\phi d\mu}}-{\wt\over4\pi}\pi
\mathfrak{F} e^{-\tau\over\mu_0}\\ 
&=I-{\wt\over4\pi}2\pi\int_{-1}^1{Id\mu}-{\wt\over4}\mathfrak{F} e^{-\tau\over\mu_0}\\ 
\end{aligned}$$
The key to solving this equation is to get rid of that integral.  To do this,
we'll do a math trick by turning the integral into a sum:
$$\int_{-1}^1{Id\mu}=\sum_{j=1}^n{a_jI(\mu_j)}$$
In general, for some polynomial of order $(2m-1)$, $f(x)=c_0+c_1x+\dots+c_{
2m-1}x^{2m-1}$:
$$\int_{-1}^1{f(x)dx}=\sum_{j=1}^n{a_jf(x_j)}$$
If we want to get an exact answer, each individual term in the polynomial had
better give exact equality:
$$\int_{-1}^1{c_\ell x^\ell dx}=a_1c_\ell x_1^\ell+a_2c_\ell x_2^\ell+\dots
+a_nc_\ell x_n^\ell$$
which is an equation in $2n$ unknowns.  Overall, we have $2m$ equations, so
to get perfect equality for a $(2m-1)^{th}$ order polynomial, we must
demand that we have $n=m$ sample points.  For a given $n$, we then need
the coefficients $a_i[i=1,\dots,n]$, $x_i[i=1,\dots,n]$.  Luckily, values for
these are tabulated in books, and in Numerical Recipes.  We'll work out a 
simple case for $n=2$:
\def\intmoo#1{\int_{-1}^1{#1}}
$$\begin{aligned}\int_{-1}^1{Id\mu}&=a_1I(\mu_1)+a_2I(\mu_2)\\ 
\intmoo{\mu^0d\mu}&=2=a_1\mu_1^0+a_2\mu_2^0\\ 
\intmoo{\mu^1d\mu}&=0=a_1\mu_1^1+a_2\mu_2^1\\ 
\intmoo{\mu^2d\mu}&={2\over3}=a_1\mu_1^2+a_2\mu_2^2\\ 
\intmoo{\mu^3d\mu}&=0=a_1\mu_1^3+a_2\mu_2^3\\ \end{aligned}$$
If we stare at these equations for a while, maybe we'll see that the answer
is $a_1=a_2=1$, and:
$$\begin{matrix}\mu_1=+{1\over\sqrt{3}},&\mu_2=-{1\over\sqrt{3}}\end{matrix}$$
Getting back to our original equation, we now have:
\def\aiaio{\left[a_1I(\mu_1)+a_2I(\mu_2)\right]}
$$\begin{aligned}\mu\ddtau{I}
&=I-{\wt\over2}\intmoo{Id\mu}-{\wt\mathfrak{F}\over4}e^{-\tau\over\mu_0}\\ 
(n=2)\,\imply\,\mu\ddtau{I(\mu)}&=I-{\wt\over2}\aiaio
-{\wt\over4}\mathfrak{F} e^{-\tau\over\mu_0}\\ \end{aligned}$$
where $\mu$ is the only remaining parameter we need to solve for:
$$\begin{aligned}\mu_1\ddtau{I(\mu_1)}&=I(\mu_1)-{\wt\over2}\aiaio-{\wt\over4}\mathfrak{F}
e^{-\tau\over\mu_0}\\ 
\mu_2\ddtau{I(\mu_2)}&=I(\mu_2)-{\wt\over2}\aiaio-{\wt\over4}\mathfrak{F}
e^{-\tau\over\mu_0}\\ \end{aligned}$$
So we have two coupled, linear, differential equations for $I(\mu_1),I(\mu_2)$.
To cut to the chase, these have homogeneous solutions:
$$\begin{aligned}I_{1,hmg}&=
{\mathfrak{L}\over1-\mu_1k}e^{k\tau}+{\mathfrak{L}\over1+\mu_1k}e^{-k\tau}\\ 
I_{2,hmg}&={\mathfrak{L}\over1+\mu_1k}e^{k\tau}+{\mathfrak{L}\over1-\mu_1k}e^{-k\tau}\\ \end{aligned}$$
where $k=\sqrt{1-\wt\over\mu_1^2}$, and $\mathfrak{L}$ is just a number.  Compare this
to 4.1.17 in the handout, where the only difference is that they pull out 
$\wt\mathfrak{F}\over4$ in
the coefficient.  There is also an inhomogeneous solution, which we can figure
out by guessing that it looks like the final terms in the coupled differential
equations, with some coefficients we need to solve for:
$$\begin{matrix} I_{1,part}={\wt\mathfrak{F}\over4}e^{-\tau\over\mu_0}h_1,&
I_{2,part}={\wt\mathfrak{F}\over4}e^{-\tau\over\mu_0}h_2\end{matrix}$$
Solving for $h_1, h_2$:
$$\begin{matrix} h_1={1-{\mu_1\over\mu_0}\over1-\wt-{\mu_1^2\over\mu_0^2}},&
h_2={1+{\mu_1\over\mu_0}\over1-\wt-{\mu_1^2\over\mu_0^2}}\end{matrix}$$

\section*{ Lecture 23 }

Recall that we were solving the following equation for $I$:
$$\mu\ddtau{I(\mu,\tau)}=I(\mu,\tau)-{\wt\over2}\left[I_1(\mu_1,\tau)+
I_2(\mu_2,\tau)\right]-{\wt\over4}\mathfrak{F}e^{-{\tau\over\mu_0}}$$
We then broke $I$ into two components measuring the upward and downward directed
components of the specific intensity:
$$\begin{aligned}\mu_1\ddtau{I(\mu_1)}&=I(\mu_1)-{\wt\over2}\aiaio-{\wt\over4}\mathfrak{F}
e^{-\tau\over\mu_0}\\ 
\mu_2\ddtau{I(\mu_2)}&=I(\mu_2)-{\wt\over2}\aiaio-{\wt\over4}\mathfrak{F}
e^{-\tau\over\mu_0}\\ \end{aligned}$$
The final solution is a sum of exponentials:
$$\begin{aligned}I(\mu_1,\tau)&\equiv I_1=I(\mu_1=\inv{\sqrt{3}},\tau)\\ 
&=I_{1,hmg}+I_{1,part}\\ 
I(\mu_2,\tau)&\equiv I_2=I(\mu_2=1\inv{\sqrt{3}},\tau)\\ 
&=I_{2,hmg}+I_{2,part}\\ \end{aligned}$$
and last time we solved for the homogeneous component.  All that remains to
be done at this point is to fit the boundary conditions.

\subsection*{ Fitting Boundary Conditions }

The first condition is that $I_2$, which measures the intensity directed toward
the ground, should vanish at $\tau=0=$the surface of the atmosphere (we are
only considering reflected light here):
$$I_2(\mu_2,\tau=0)=0$$
The second condition is that at the ground the intensity directed upward out
of the ground should be equal to whatever is reflected from the downward
directed intensity.  Thus:
$$I_1(\tau=\tau_{max},\mu_1>0)=I_2(\tau-\tau_{max},\mu_2<0)\Lambda$$
where $\Lambda$ is the ground albedo.  There is another component we should
examine.  First, let's define:
$$I_*(\tau,\mu)=\pi\mathfrak{F}\delta(\mu+\mu_0)\delta(\phi-\phi_0)e^{\tau\over\mu}$$
Then we'll examine the total flux (specific intensity integrated over solid
angle, weighted by $\mu$) directed into the ground:
\def\taum{{\tau_{max}}}
$$\begin{aligned}F_{in,*}
&=\int{\pi\mathfrak{F}\delta(\mu+\mu_0)\delta(\phi-\phi_0)e^{\taum\over\mu}\mu
\,\underbrace{d\mu d\phi}_{d\Omega}}\\ 
&=-\pi\mathfrak{F}e^{-{\taum\over\mu}}\mu_0\\ \end{aligned}$$
Since the flux out of the ground is just the reflected portion of the incoming
flux:
$$\begin{aligned}F_{out,*}&=\Lambda F_{in,*}\\ 
F_{out,*}&\equiv\int{I_{1,*}\mu d\Omega}\\ 
&=I_{1,*}\int{\mu d\Omega}=I_{1,*}\pi\\ \end{aligned}$$
$$\boxed{I_{1,*}=\lambda\mathfrak{F}\mu_0e^{-{\taum\over\mu_0}}}$$
Therefore, the full second boundary condition reads:
$$\boxed{I_1(\tau=\taum,\mu_1>0)=\Lambda I_2(\tau=\taum,\mu_2<0)
+\Lambda\mathfrak{F}e^{-{\taum\over\mu_0}}\mu_0}$$
These two boundary conditions together give us answers to the two right-hand 
components of our original equation:
$$\mu\ddtau{I(\mu,\tau)}=I(\mu,\tau)-{\wt\over2}\left[I_1(\mu_1,\tau)+
I_2(\mu_2,\tau)\right]-{\wt\over4}\mathfrak{F}e^{-{\tau\over\mu_0}}$$
So the equation had just become the equation of radiative transfer:
$$\mu\ddtau{I}=I-S$$
and now we know the source function $S$.

\subsection*{ Effective Albedo of the Atmosphere }

Using 4.1.29 in the handout, we have that an atmosphere of optical depth
$\tau=\taum$, with ground albedo $\Lambda$ is identical to an infinite 
atmosphere ($\tau\to\infty$) if $\xi=0$.  In order for this to be the case:
$$\xi={\sqrt{1-\wt}(1+\Lambda)-(1-\Lambda)
\over\sqrt{1-\wt}(1+\Lambda)+(1-\Lambda)}=0$$
$$\boxed{\Lambda={1-\sqrt{1-\wt}\over1+\sqrt{1-\wt}}}$$
where $\Lambda$ here is the effective, macroscopic albedo of a semi-infinite
atmosphere.\par
We can get this same result by solving the diffusion equation, this time with
an absorption modifier:
$$\ppt{n}=\underbrace{D\nabla^2n}_{scattering\atop term}
-\underbrace{nn_d\sigma_ac}_{absorption\atop modifier}$$
where $n$ is the \# density of photons (which don't change in frequency), 
$\sigma_a$ is the cross-section for
absorption, $n_d$ is the \# density of the ``dust'' (any absorbing/scattering
particle), and $D$ is the diffusivity, $D=\lambda_{mfp}c$.  We have to be
careful that $\lambda_{mfp}$ contain both aspects of absorption and scattering:
$$\lambda_{mfp}=\inv{n_d(\sigma_a+\sigma_s)}$$
We have a steady-state solution to this diffusion equation.  In 1-D, this 
equation looks like:
$$\ppt{n}=0=D{\partial^2n\over\partial z^2}-n(n_d\sigma_ac)$$
We'll guess that $n=Ae^{z\over z_0}$ is a solution.  Doing some algebra, we
can show that:
$$z_0=\inv{n_d[\sigma_a(\sigma_a+\sigma_s)]^\hf}$$
where we are assuming $z=0$ is the top of the atmosphere, where $\tau=0$, and
that $z<0$ as we move down through the atmosphere.  Now we'll break our
solution into two components, evaluated at the surface of the atmosphere:
$$\begin{aligned}n&=Ae^{z\over z_0}\\ 
n(0)&=(n_i+n_e)e^{0\over z_0}\\ \end{aligned}$$
where $n_i$ are the incident photons from space, and $n_e$ are the escaping
photons from below.  Then the diffusion equation (this time written in terms
of flux) tells us:
$$\begin{aligned}F&=-D\nabla n=c(n_e-n_i)\\ 
{-DAe^{z\over z_0}\over z_0}\eval{z\sim0}&=c(n_e-n_i)\\ \end{aligned}$$
We are saying $z\sim0$, and not $z=0$, because we have to be careful about
using the diffusion equation at low optical depth, which we have at the surface
of the atmosphere.  Ignoring the subtleties inherent in the assumption, we'll
say that $e^{z\over z_0}\sim1$, giving us:
$$\boxed{{-D\over z_0}(n_i+n_e)}$$
Now we use the definition $\Lambda\equiv{n_e\over n_i}\eval{z=0}$ to get our
previous equation.  First, note that:
$$n_e={\left(\sqrt{\sigma_a+\sigma_s\over\sigma_a}-1\right)\over
\left(\sqrt{\sigma_a+\sigma_s\over\sigma_a}+1\right)}n_i
={\left(\sqrt{\inv{1-\wt}}-1\right)\over\left(\sqrt{\inv{1-\wt}}+1\right)}n_i$$
where we used that $1-\wt={\sigma_a\over\sigma_a+\sigma+s}$.  Using this
result, we can solve for $\Lambda$:
$$\Lambda={n_e\over n_i}={1-\sqrt{1-\wt}\over1+\sqrt{1-\wt}}$$

\subsection*{ Quick Reality Check }

We can use the fact that 
$$F_{out}\eval{surface}=\int{I_1\eval{surface}\mu\,d\Omega}$$
to independently find the macroscopic albedo:
$$\Lambda={F_{out}\eval{surface}\over F_{in}\eval{surface}}$$
Doing this by hand, using the test case of looking at reflected flux directed
straight up ($\mu_0=1$), from perfectly reflective particles $\wt=1$, with
$\mu_1=\hf$ (an approximation of a quadrature gaussian integral), we get:
$$\begin{aligned}F_{out}\eval{surface}&={3\pi\over2}\ln2\mathfrak{F}\\ 
F_{in}\eval{surface}&=\pi\mathfrak{F}\\ \end{aligned}$$
Numerically, we find $\Lambda=1.03$.  We were, of course, shooting for an
answer of $1$, so this hand-calculation gives us an indication that all of
the various assumptions we made have introduced a couple \% error in our
final calculations.

\section*{ Lecture 24 }

\subsection*{ Cosmic Masers }

$OH$ and $H_20$ masers can occur in dusty, star-forming regions which are cold
enough for these molecules to form.  The the dust's black-body radiation in
the infrared band creates is absorbed by these molecules and a population
inversion is established.  When maser emission is caused by via stimulated
emission, these clouds can get very bright (brightness temperatures $\sim
10^{14}K$).  Temperatures this high cannot be thermal, so we know a maser
when we see one. \par
Generally, masers are useful for tracing the galaxy's magnetic field (emission
lines are Zeeman split), and for following disks of gas and dust around stars
in star-forming regions.  In order to detect them, we need clouds which are
moving uniformly together, and have velocity coherence both $\perp, \|$ to
our line-of-sight through a disk of rotation gas.  
In general, the intensity we observe depends on the
path length through the masing cloud, so we like long path lengths with the
same velocity.

\subsection*{ How masers Work }

Consider a molecule with two rotational energy levels.  Observing a homogeneous
slab of this molecule, the intensity we receive is given by the familiar:
$$I_\nu=S_\nu(1-e^{-\tau_\nu})$$
where $S_\nu={j_\nu\over\alpha_\nu}$.  $j_\nu$ and $\alpha_\nu$ are given by:
$$\begin{aligned}j_\nu&=\overbrace{n_2\ato}^{per time}\overbrace{h\nu}^{per E}
\overbrace{\phi(\nu)}^{per\ Hz}\overbrace{\inv{4\pi}}^{per steradian}\\ 
\alpha_\nu&={h\nu\over4\pi}\phi(\nu)[\overbrace{n_1\bot}^{abs}-\overbrace{
n_2\bto}^{stim\atop emis}]\\ \end{aligned}$$
Thus, our source function looks like:
$$S_\nu={n_2\ato\phi(\nu)h\nu\inv{4\pi}\over{h\nu\over4\pi}\phi(\nu)[n_1\bot-
n2\bto]}$$
Then since $g_1\bot=g_2\bto$ and ${\ato\over\bto}={2h\nu^3\over c^2}$, we have:
$$S_\nu={2h\nu^3\over c^2}\inv{{n_1g_2\over g_1n_2}-1}$$
A population is said to be inverted when $n_1g_2<g_1n_2$ (not $n_1<n_2$).
$n_1\over g_1$ is an expression for the population per degenerate sub-level in
energy level 1.  If ${n_1\over g_1}<{n_2\over g_2}$, then $S_\nu<0$, and we
have a maser.  Expressed in terms of the excitation temperature (${n_2\over n_1}
={g_2\over g_1}e^{-{h\nu\over kT_{ex}}}$), we have:
$$S_\nu={2h\nu^3\over c^2}\inv{e^{h\nu\over kT_{ex}}-1}$$
which is less than 0 when $T_{ex}<0$.  We can express the optical depth of
this slab to maser radiation as:
$$\tau_\nu=\alpha_\mu L
={h\nu\over4\pi}\phi(\nu)\bto\left[{n_1g_2\over g_1}-n_2\right]
={h\nu\over4\pi}\phi(\nu)\bto n_2\left[{n_1g_2\over g_1n_2}-1\right]$$
If ${n_2\over g_2}>{n_1\over g_1}$, then $\tau_\nu<0$.  Now you might think
we're talking nonsense with a negative source function and a negative optical
depth, but we're not.  Recall that the intensity is:
$$I_\nu=S_\nu(1-e^{-\tau})$$
If $\tau\to-\infty$, then $I_\nu\to-S_\nu e^{\tau_\nu}$, so $I_\nu\gg1$. 
On the other hand, if $\tau<0$ and $|\tau|\ll1$, then $I_\nu\to S_\nu\tau_\nu$,
which is the product of two negatives = positive.  This should be convincing
you that $I_\nu$ is always positive, and therefore, actually manifested.

\subsection*{ Maser Species }

$OH$ mases at around 18 cm, and is found around AGB (asymptotic-giant-branch)
stars in star-forming regions and around the galactic nucleus.  AGB's are
important because they have lots of dust.  The two masing transitions are
from $1667\to1612MHz$ and $1720\to1665MHz$.\par
$H_2O$ mases at $1.35 cm$ in a transition to its ground rotational state.  Note
that an order-of-magnitude calculation of $\Delta E={\hbar\over 2I}$ for $H_2O$
gives us an estimate of $\sim 1 mm$, which is incorrect.  The correct 
transitional energy is caused by a slight degeneracy in water molecules.\par
$SiO$ mases at $3.4 mm$ in its ground vibrational state, and is typically
found in star-forming regions and around AGB stars.\par
Other molecules mase, and there is even potential for detecting atomic lasers 
around massive stars ($L\sim10^5 L_\odot$).  Hydrogen transitions from $10\to9$
have been observed (at $55\mu m$), and Stielnitski 1996 claims to have observed
a population inversion in atomic H.

\section*{ Lecture 25 }

\subsection*{ Saturated vs. Unsaturated masers }

There are two modes of operation for masers.  For unsaturated masers, the gain
is exponential with the path length, and for saturated ones, the gain only
grows linearly with path length.  The masers we find in the cosmos are typically
saturated.  The following is working toward understanding why there are two
modes in masers.  We begin with the expression for specific intensity from
emissivity:
$$\begin{aligned}{dI_\nu\over dz}&=j_\nu-\alpha_\nu I_\nu\\ 
&={h\nu\over4\pi}\phi(\nu)n_2A_{21}-{h\nu\over4\pi}
\phi(\nu)[n_1\bot-n_2\bto]I_\nu\\ \end{aligned}$$
In principle, the line-profile function governing spontaneous emission and the
line-profile governing stimulated emission might not have to be the same, but
evidence seems to suggest they are.  Let's assume they are, and rewrite this:
$$\begin{aligned}{dI_\nu\over dz}&={h\nu\over4\pi}\phi(\nu){n_2\ato\over g_2}\cdot g_2
-{h\nu\over4\pi}\phi(\nu)g_2\bto\left[{n_1\over g_1}-{n_2\over g_2}\right]
\end{aligned}$$
Defining $N_1\equiv{n_1\over g_1}$, $N_2\equiv{n_2\over g_2}$, $A\equiv
\ato g_2$, and $B=\bto g_2$, our equation looks like:
$${dI_\nu\over dz}={h\nu\over4\pi}\phi(\nu)[N_2A+B(N_2-N_1)I_\nu]$$
Now we'll integrate over frequency.  $\phi(\nu)$ is a sharply peaked function,
so we'll treat it as a $\delta$-function.  Then substituting
$I\equiv\int{\phi(\nu)I_\nu d\nu}$, we have:
$${\int{I_\nu d\nu}\over\Delta\nu}\equiv\int{I_\nu\phi(\nu)d\nu}$$
which is true by definition of $\Delta\nu$: $\int{I_\nu d\nu}=I\Delta\nu$.
So now we have:
$${dI\over dz}={h\nu\over4\pi\Delta\nu}[(N_2-N_1)BI+N_2A]$$
This is an equation with three unknowns.  To close this system, we'll use the
equations of statistical equilibrium, which say:
$${dN_2\over dt}=-N_2BJ+N_1BJ-N_2A+R_2(N-N_1-N_2)-\Gamma_2N_2$$
where $J\equiv\int{I_\nu\phi(\nu)d\nu}$, and $J_\nu=\inv{4\pi}\int{I_\nu
d\Omega}$.  $R_2$ is the ``pumping rate''.  It describes the rate at which
molecules not in state 1 or 2 (counted by $N-N_1-N_2$) are radiatively or 
collisionally
knocked into state 2. $\Gamma_2$ is the loss rate of molecules in state 2 into
any state other than state 1.  In the above equation, we've neglected two
terms: collisional excitation $1\to2$ and $2\to1$.  These are crucial terms,
being tightly related to local thermal equilibrium.  However, we will neglect
them to simplify analysis.  We also have a similar equation for state 1:
$${dN_1\over dt}=N_2BJ-N_1BJ+N_2A+R_1(N-N_1-N_2)-\Gamma_1N_1$$
We'll further simply matters by setting the loss-rates for the two populations
equal to each other ($\Gamma_1=\Gamma_2=\Gamma$).  Now let's solve for the 
steady-state solution (${dN\over dt}=0$).
$${d(N_2-N_1)\over dt}=-(N_2-N_1)2BJ-2N_2A+(R_2-R_1)(N-N_{12})-\Gamma(N_2-N_1)$$
where $N_{12}=N_1+N_2$.
Two terms in this equation are acting to reduce the population inversion of
state 2 with respect to state 1: $(N_1-N_1)2BJ$ and $\Gamma(N_2-N_1)$.  In the
{\it unsaturated regime} where $BJ\ll\Gamma$, stimulated emission is a minor
perturbation to the inverse.  In this case we can solve the system of equations:
$$\begin{aligned}{d\Delta N\over dt}={d(N_2-N_1)\over dt}&=-2N_2A+(R_2-R_1)(N-N_{12})
-\Gamma\Delta N=0\\ 
{dN_{12}\over dt}&=(R_2+R_1)(N-N_{12})-\Gamma(N_1+N_2)=0\\ \end{aligned}$$
Then solving for $N-N_{12}$:
$$N-N_{12}={\Gamma(N_1+N_2)\over R_1+R_2}={\Gamma N_{12}\over R_1+R_2}$$
Substituting this into the difference equation, which reads:
$$-2N_2A={R_2-R_1\over R_1+R_2}\Gamma N_{12}-\Gamma\Delta N=0$$
the using $N_2={N_{12}+\Delta N\over2}$ (this is just an identity), we get:
$$\underbrace{\left(1+{\Gamma\over A}\right)}_{\equiv2\beta}\Delta N
=N_{12}\left[\underbrace{{R_2-R_1\over R_1+R_2}{\Gamma\over A}}_{\equiv
\inv{\alpha}}-1\right]$$
Thus:
$$\Delta N={N_{12}\over2\beta}\left[\inv{\alpha}-1\right]={N_{12}(1-\alpha)\over
1\alpha\beta}$$
Now we'll define $S\equiv{\alpha\beta\over1-\alpha}$.  $S$ is meant to connote
the source function here.  The reason we might expect $S$ to be related to
the source function is that $S_\nu\propto{N_2\over N_2-N_1}\propto{N_2\over
\Delta N}$, so $\Delta N\propto {N_{12}\over2S}$.  This is why we use an $S$
here.  Getting back to our original equation, we have:
$${4\pi\Delta\nu\over h\nu A}{dI\over dz}=\Delta N\underbrace{BI\over A}_{
\equiv\mathfrak{I}}+N_2$$
Using our newly defined $S$, this becomes:
$${4\pi\Delta\nu\over h\nu B}{d\mathfrak{I}\over dz}={N_{12}\over2S}\mathfrak{I}+
\hf(N_{12}+\Delta N)$$
Finally, defining an unsaturated gain length $L\equiv{4\pi\Delta\nu\over Bh\nu}
{2S\over N_{12}}$, we have:
$$L=\mathfrak{I}+S+\hf$$
Then using $ds={dz\over L}$, we have:
$${d\mathfrak{I}\over ds}=\mathfrak{I}+\hf+S$$
To order of magnitude:
$$\begin{aligned}\mathfrak{I}&\equiv{B\over A}I={B\over A}B_\nu(T_{bright})\\ 
&={c^2\over2h\nu^3}{2h\nu^3\over c^2}\inv{e^{h\nu\over kT_{bright}}-1}\\ 
&=\inv{e^{h\nu\over kT_{bright}}-1}\\ \end{aligned}$$
Now since $1\ll{kT_{bright}\over h\nu}$ (typical maser wavelengths are in
mm and $T_{bright}$ is typically several K), we can throw away our $\hf$ in
our equation for ${d\mathfrak{I}\over ds}$:
$${d\mathfrak{I}\over ds}=\mathfrak{I}+S\imply
\int{d\mathfrak{I}\over\mathfrak{I}+S}=\int{ds}\imply
\mathfrak{I}+S=De^s$$
Choosing $\mathfrak{I}=0, s=0$ (we're assuming there is no background source), then
$D=S$, so:
$$\boxed{\mathfrak{I}=S(e^s-1)}$$
If $s\ll1$, we have $\mathfrak{I}=Ss$ from spontaneous emission, and this is the 
saturated case.  If $s\gg1$, the $\mathfrak{I}=Se^s$ from stimulated emission, and
this is the unsaturated case.  Earlier we threw away the $BJ$ term, but we
could have done all of this including that term, and the algebra would have
been the same.  If we do this, we ge the answer:
$$\boxed{{d\mathfrak{I}\over ds}={\beta(\mathfrak{I}+\hf)\over\beta+\mathfrak{J}}+S}$$
where $\mathfrak{J}$ is the non-dimensionalized, integrated flux, 
$\mathfrak{J}={BJ\over A}$.

\section*{ Lecture 26 }

\subsection*{ The Saturated Mode }

Recall our masing equation:
$${d\mathfrak{I}\over ds}={\beta(\mathfrak{I}+\hf)\over\beta+\mathfrak{J}}+S$$
In the unsaturated case, ${\beta\over\beta+\mathfrak{J}}\to1\iff\beta\gg\mathfrak{J}$.
This is equivalent to saying:
$$\hf\left(1+{\Gamma\over A}\right)\gg{BJ\over A}$$
And if $\Gamma\gg A$, then:
$$A+\Gamma\gg BJ\iff\Gamma\gg BJ$$
In the saturated case, then $\beta\ll\mathfrak{J}$.  In this case:
$${d\mathfrak{I}\over ds}={\beta\mathfrak{I}\over\mathfrak{J}}+S={4\pi\beta\mathfrak{I}\over
\int{\mathfrak{I}ds}}+S$$
where we used that $\mathfrak{J}\sim\inv{4\pi}\int{\mathfrak{I}d\Omega}$.  If we consider
a single beam of photons through a cloud, then $\int{\mathfrak{I}d\Omega}\approx
\mathfrak{I}\Delta\Omega$, so:
$$\boxed{
\begin{aligned}{d\mathfrak{I}\over ds}&={4\pi\beta\mathfrak{I}\over\mathfrak{I}\Delta\Omega}+S\\ 
&={4\pi\beta\over\Delta\Omega}+S\\ \end{aligned}}$$

\subsection*{ How We Get Population Inversions }

There is another dichotomy in masers: ones which are excited radiatively and
those which are excited collisionally.  We'll discuss a cloud of molecules
which have 3 energy states which are populated by {\it simple collisional
pumping}.  To simplify our lives, we'll say that energy levels 3 and 2 can
talk to each other via photon emission/absorption, as can 2 and 1, but we'll
say that 1 and 3 cannot talk radiatively (say, because of parity violation, and
note that they can still talk collisionally).  2 and 1 will be
our masing levels.  Then the rate of change of the population of energy state
1 is given by (sources - sinks):
$${dn_1\over dt}=n_2\ato+n_2C_{21}+n_3C_{31}+n_2\bto J_{21}-n_1\bot J_{12}
-n_1C_{12}-n_1C_{13}$$
Then $J_{12}=J_{21}$, and because of local collisional thermal equilibrium,
$C_{12}=C_{21}{g_2\over g_1}e^{-{E_{21}\over kT}}$.  Similarly,
$C_{13}=C_{31}{g_3\over g_1}e^{-{E_{21}\over kT}}$.  So dividing by $g$, and
defining $N_1\equiv{n_1\over g_1}$, we have:
$$\begin{aligned}{dN_1\over dt}=&\underbrace{n_2\over g_2}_{N_2}\underbrace{
{\ato\over g_1}g_2}_{\ato}+\underbrace{n_2\over g_2}_{N_2}\underbrace{
{C_{21}\over g_1}g_2}_{\equiv C}+\underbrace{n_2\over g_3}_{N_3}\underbrace{
{C_{31}\over g_1}g_3}_{\equiv C}+\underbrace{n_2\over g_2}_{N_2}\underbrace{
{\bto\over g_1}g_2}_{\bto}J_{21}\\
&-N_1\underbrace{\bto}_{\bto{g_2\over g_1}\equiv
\bto}-N_1\underbrace{C_{21}{g_2\over g_1}}_{\equiv C}e^{-{E_{21}\over kt}}-
N_1\underbrace{C_{31}{g_3\over g_1}}_{\equiv C}e^{-{E_{31}\over kT}}\\
\end{aligned}$$
Phew.  Notice that we set $C_{21}=C_{31}$.  This is just to make our lives
easier.  We can do the same for $dN_2\over dt$, but omitting $B_{23}$, because
we're deciding not to have absorptions from $2\to3$ and no stimulated emission
from $3\to2$.  Then our total population is $N=N_1+N_2+N_3$.  Without being
careful, instinct tells us that in steady state, we'll have a population
inversion ${N_2\over N_1}>1$ if $C\ll A_{32}$.  This
instinct is correct, but let's do this carefully.  First we'll make some
assumptions:
\begin{itemize}\item $\ato\ll C$.  For $H_2O$: 
$$\ato\sim10^8s^{-1}\left({1216\AA\over1.35cm}\right)^3
\sim10^8\e{-15}s^{-1}
\sim10^{-7}s^{-1}\left({\mu\over ea_0}\right)^2$$
where $\mu$ is our way of accommodating the fact that the dipole moment for
$H_2O$ might not be the same as for the fine structure of hydrogen.  It turns
out the answer is $\ato\sim2\e{-9}s^{-1}$.\par
Estimating C:
$$C\sim n_{H_2}\sigma v_{rel}\sim n_{H_2}\e{-15}\cdot(3{km\over s})
\sim n_{H_2}\cdot3\e{-10}$$
which is $\gg10^{-7}s^{-1}\left({\mu\over ea_0}\right)^2$ when
$n_{H_2}\gg10^3cm^{-3}\left({\mu\over ea_0}\right)^2$.  
\item Next we'll assume $E_{21}\ll kT$.  Define ${E_{21}\over kT}\equiv\delta
\ll1$.\end{itemize}
Now we have a 2-step:
\begin{itemize}\item Step 1: Since $1\to3$ are not linked by radiation,
$${N_3\over N_1}\approx e^{-{E_{31}\over kT}}\equiv\theta\le1$$
\item Step 2: Radiative decays from $3\to2$.  To get an inversion, we'll
argue that the sources into 2 are larger than the sinks out of 2 (this is 
a little weird because we're solving our steady-state equations, but whatever):
$$N_3A_{32}+N_3C+N_1Ce^{-{E_{21}\over kT}}>N_2Ce^{-{E_{32}\over kT}}+N_2C$$
We can rewrite this as:
$$\begin{aligned}
\underbrace{N_2\over N_1}_{=\theta}{A_{32}\over C}&>\left[{N_2\over N_1}
\underbrace{e^{-{E_{32}\over kT}}}_{E_{32}=E_{32}-E_{12}}-{N_3\over N_2}\right]
+\left[{N_2\over N_1}-e^{-{E_{21}\over kT}}\right]\\ 
{A_{32}\theta\over C}&>\left[\underbrace{N_2\over N_1}_{1}\theta(1+\delta)-
\theta\right]+\left[\underbrace{N_2\over N_1}_{1}-(1-\delta)\right]\\ 
&=\theta\delta+\delta=\delta(\theta+1)\\ \end{aligned}$$
Thus, the $\delta$ really helps get masing started.  
\end{itemize}
Now one last thing: we'd
chosen to ignore stimulated radiative transfer between energy states 3 and 2.
In general, this process
will tend to reduce the population inversion.  However, for optically thick
clouds $\tau\gg1$, photons only have a $P\sim\inv{\tau}$ probability of escaping,
so we can describe this by ``diluting'' the $A_{32}$ term by $\inv{\tau}$.

\end{document}
