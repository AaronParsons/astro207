\documentclass[11pt]{article}
\usepackage[top=1in, bottom=1in, left=1in, right=1in]{geometry}

\usepackage{titlesec}
\titleformat{\subsection}[runin]{\normalfont\large\bfseries}{\thesubsection}{1em}{}

\begin{document}
\pagestyle{empty}
\parindent=0pt

\section*{\centering Problem Set 3 (part 1)}

\section{Blackbody Flux}

Derive the blackbody flux formula
\begin{equation}
F=\sigma T^4
\end{equation}
from the Planck function for specific intensity,
\begin{equation}
B_\nu=\frac{2h\nu^3}{c^2}\frac{1}{e^\frac{h\nu}{kT} - 1}
\end{equation}
You may use the integral $\int_0^\infty{\frac{dx}{x^5(e^{1/x}-1)}}=\pi^4/15$.
Those who know what they have to do can stop reading here. Those who need a bit of help should read on.

Imagine a perfectly flat, blackbody patch at the center of a sphere of radius
$R$. The patch has tiny surface area $dA$ and is lying flat at $r = 0$ in the $\theta=\pi/2$
plane (in spherical coordinates, where $r$, $\theta$, $\phi$ are the radius, polar angle, and
azimuth). Only one side of the patch is warm; the other side is at absolute
zero. Assume that the warm side radiates as a perfect thermal blackbody.

A thermal blackbody emits radiation isotropically --- i.e., with no preference for
direction. In other words, imagine each point on the patch beaming out a tiny,
uniform, hemispherical dome of rays (like an umbrella). It is a hemispherical
dome and not a full sphere because only one side of the patch is warm. Note
that the Planck blackbody function has no variable inside it that specifies
direction of emission; the direction of emission doesn't matter for purely
thermal emission. The only variables that need to be specified for the Planck
function are $T$ and $\nu$.

Despite the isotropic nature of the emission, because the patch is
geometrically flat, has definite area, and emits only on one side of itself, a
detector glued to the inside of the sphere at $r = R$ detects different amounts
of radiation depending on where it is placed. For example, if the detector is
glued on the half of the sphere that can’t see the bright side of the patch,
the detector detects nothing. If the detector is glued to the sphere directly
above the patch, the detector picks up the most number of photons, because it
sees the full face-on area of the patch (namely, $dA$). If the detector is glued
at an angle to the pole, then it sees less than area dA. If it is glued in the
plane of the patch, it sees nothing but a infinitesimally thin line segment.

This problem asks you to place detectors everywhere on the inside of the sphere
and sum up the radiation collected; this operation is equivalent to
``integrating the Planck function over all solid angles into which the radiation
is beamed."

Calculate the total luminosity (in units of energy/time) emitted by the patch
of area dA. You must integrate the Planck function over the entire emitting
area (which is maximally $dA$ but in general will be less, depending on the
viewing geometry), over all solid angles into which the radiation is beamed,
and over all frequencies. Think about placing tiny detectors all over the
inside of the sphere and asking how much energy/time each of the detectors
receives.

Then divide the luminosity by $dA$ to calculate the total flux (in units of
energy/time/area) emitted by the patch. You should recover the usual blackbody
flux formula, $\sigma T^4$. By definition, $\sigma T^4$ is the total amount of
energy radiated per time per unit area of a blackbody surface, radiated into
all solid angles and over all frequencies.

\section{Flat Disks}

This problem forms the foundation for understanding the spectral energy distributions of
circumstellar disks, e.g., those surrounding pre-main-sequence stars and AGN.

Consider a perfectly flat, blackbody disk encircling a blackbody star. The star
has radius $R_*$ and effective temperature $T_*$. The disk begins at a stellocentric
radius far from the star, i.e. the inner disk radisu $r_i\gg R_*$.  
The disk extends to infinity.

Calculate the temperature of the disk, $T(r)$, as a function of radius, $r$. Make
whatever approximations you deem necessary in light of the fact that $r_i\gg R_*$. Be
sure you get the scaling of $T$ with $r$ correctly. It is less important that you
get the numerical coefficient correctly.

This disk is PERFECTLY flat. Do not consider individual particles in the disk.

Also, neglect radial transport of energy. Each annulus is independent of
neighboring annuli. This is a fine approximation for thin disks since they
transport more radiation vertically than radially (the temperature gradient is
steeper vertically than radially).

\section*{\centering Problem Set 3 (part 2)}

\section{Practice with $j_\nu$, $\alpha_\nu$, $S_\nu$, $I_\nu$}

\subsection{}

A plane-parallel slab of uniformly dense gas is known to be in LTE (local
thermodynamic equilibrium) at a uniform temperature $T$. Its thickness normal
to its surface is $s$. Its absorption coefficient is $\alpha_{\nu,{\rm gas}}$. 
Write down the
specific intensity, $I_\nu$, viewed normal to the slab, in terms of the variables
given.

\subsection{}

The same slab is now filled uniformly with non-emissive dust having absorption
coefficient $\alpha_{\nu,{\rm dust}}$. The dust is non-emissive, so its emissivity 
$j_{\nu,{\rm dust}}=0$.
Write down $I_\nu$ viewed normal to the slab, in terms of all variables given so
far.

\subsection{}

The slab of gas and dust is further mixed with a third component: an emissive,
non-absorptive uniform medium having emissivity $j_{\nu,{\rm med}}$ and
absorption coefficient $\alpha_{\nu,{\rm med}}=0$. Write down $I_\nu$ viewed
normal to the slab, in terms of all variables given.

A physical realization of this problem might be an HII region surrounding an
ionizing O star. The material in LTE would be the fully ionized plasma,
emitting thermal bremsstrahlung radiation. The dust would be dust. The
emissive, non-absorptive medium would be the same ionized plasma emitting
recombination (line) radiation. For the assumptions stated in the problem to be
valid, we would have to evaluate $\nu$ at, say, an optical recombination line
like H$\alpha$.

\section{Detailed Balance}

Suppose an atom has 2 energy states $E_1$ and $E_2$.  This atom occasionally
collides with an e$^-$, whereupon it may change states.  Let us say that
during a collision, there is a probability of $P_{12}$ that the atom transitions
from state 1 to state 2, and $P_{21}$ for the reverse (from 2 to 1).

\subsection{}
If these are the only transitions, what do you expect the relative populations
to be of atoms in state 1 versus state 2 after many collisions?  Write some
code that verifies this by simulating it (and please submit your code with your homework,
along with the output of the code).  Pick $P_{12}$ and $P_{21}$ to be something
illustrative.  What are the relative populations in each state?
Does this system satisfy detailed balance?  Does it matter if $P_{12}=P_{21}$?
Why/why not?

\subsection{}
Modify your code above to add a third energy state, $E_3$, along with the relevant
transition probabilities to/from all other states.  Try two cases: one
where you assign symmetric jump probabilities for excitation/de-excitation,
and one where assign asymmetric jump probabilities.
What are the relative
populations in each state?
Does this system satisfy detailed balance?  Why/why not?

\end{document}
