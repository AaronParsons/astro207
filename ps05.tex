\documentclass[11pt]{article}
\usepackage[top=1in, bottom=1in, left=1in, right=1in]{geometry}
\usepackage{amsmath}
\usepackage{titlesec}
\titleformat{\subsection}[runin]{\normalfont\large\bfseries}{\thesubsection}{1em}{}

\begin{document}
\pagestyle{empty}
\parindent=0pt

\section*{\centering Problem Set 5 (part 1)}

\section{Multiple Multipoles}

\subsection{}
Write down the electric field a distance $r$ away from a monopole of charge $q$.


\subsection{}
Someone moves another monopole of charge $−q$ next to the original monopole.
The two charges are separated by a distance $b$. Derive, to order-of-magnitude,
the factor by which this dipole electric field is reduced from the monopole
field.

\subsection{}
Someone moves two more charges, $−q$ and $q$, into position to
form a square.  The edge of the square has length $b$. Going around the square,
the charges are $−q$, $q$, $−q$, and $q$. Derive, to order-of-magnitude, the
factor by which this quadrupole electric field is reduced from the dipole
field.

\subsection{}
Draw a picture of a pure electric octopole, that is, a charge distribution
for which the electric field decreases as $1/r^5$ (and no less gradually). For
bonus points, draw a picture of an electric hexadecapole (electric field dies
no less gradually than $1/r^6$).

\subsection{}
Now imagine the dipole and quadrupole configurations 
rotating about their centers-of-charge with frequency $\nu$. We have a rotating
barbell and a rotating square, respectively. (Think CO and H$_2$).

To order-of-magnitude, what is the maximum distance from each object inside of
which the electric fields are nearly perfectly in phase with the rotation?
This is the boundary of the near zone, inside of which the electric field
geometry rotates with frequency $\nu$ as if it were a rigid body.

\subsection{}
Electromagnetic waves are emitted at the boundary of the near zone, into
the far (radiation) zone.

Write down (one line of argument suffices) the factor by which the power
carried by waves emitted by the rotating quadrupole is smaller than the power
carried by waves emitted by the rotating dipole.

\section{Rotating Magnetic Dipole}

Based on Rybicki \& Lightman Problem 3.1.

A pulsar is convetionally believed to be a rotating neutron star.  Such
a astar is likely to have a strong magnetic field, $B_0$, since it traps
lines of force during its collapse.  If the magnetic axis of the neutron
star does not line up with the rotation axis, there will be magnetic dipole
radiation from the time-changing magnetic dipole, $m(t)$.  Assume that the mass
and radius of the neutron star ar $M$ and $R$, respectively; that the angle
between the magnetic and rotation axes is $\alpha$; and that the rotational
angular velocity is $\omega$.

\subsection{}
Find an expression for the radiated power $P$ in terms of $\omega$,
$R$, $B_0$, and $\alpha$.

\subsection{}
Assuming that the rotational energy of the pulsar is the ultimate source
of the radiated power, find an expression for the slow-down timescale,
$\tau$, of the pulsar.

\subsection{}
For $M=1 M_\odot$, $R=10^6$ cm, $B_0=10^{12}$ gauss, $\alpha=90^\circ$,
find $P$ and $\tau$ for $\omega=10^4$ s$^{-1}$, $10^3$ s$^{-1}$,
$10^2$ s$^{-1}$.  The highest rate $\omega=10^4$ s$^{-1}$ is believed
to be typical of newly formed pulsars.

Hint: The Larmor power formula gives the power radiated by an accelerating
electric dipole moment. This problem is similar except that it concerns an
accelerating magnetic dipole moment.

\section*{\centering Problem Set 5 (part 2)}

\end{document}
