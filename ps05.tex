\documentclass[11pt]{article}
\usepackage[top=1in, bottom=1in, left=1in, right=1in]{geometry}
\usepackage{amsmath}
\usepackage{amssymb}
\usepackage{titlesec}
\titleformat{\subsection}[runin]{\normalfont\large\bfseries}{\thesubsection}{1em}{}

\begin{document}
\pagestyle{empty}
\parindent=0pt

\section*{\centering Problem Set 5 (part 1)}

\section{Multiple Multipoles}

\subsection{}
Write down the electric field a distance $r$ away from a monopole of charge $q$.


\subsection{}
Someone moves another monopole of charge $−q$ next to the original monopole.
The two charges are separated by a distance $b$. Derive, to order-of-magnitude,
the factor by which this dipole electric field is reduced from the monopole
field.

\subsection{}
Someone moves two more charges, $-q$ and $q$, into position to
form a square.  The edge of the square has length $b$. Going around the square,
the charges are $-q$, $q$, $-q$, and $q$. Derive, to order-of-magnitude, the
factor by which this quadrupole electric field is reduced from the dipole
field.

\subsection{}
Draw a picture of a pure electric octopole, that is, a charge distribution
for which the electric field decreases as $1/r^5$ (and no less gradually). For
bonus points, draw a picture of an electric hexadecapole (electric field dies
no less gradually than $1/r^6$).

\subsection{}
Now imagine the dipole and quadrupole configurations 
rotating about their centers-of-charge with frequency $\nu$. We have a rotating
barbell and a rotating square, respectively. (Think CO and H$_2$).

To order-of-magnitude, what is the maximum distance from each object inside of
which the electric fields are nearly perfectly in phase with the rotation?
This is the boundary of the near zone, inside of which the electric field
geometry rotates with frequency $\nu$ as if it were a rigid body.

\subsection{}
Electromagnetic waves are emitted at the boundary of the near zone, into
the far (radiation) zone.

Write down (one line of argument suffices) the factor by which the power
carried by waves emitted by the rotating quadrupole is smaller than the power
carried by waves emitted by the rotating dipole.

\section{Rotating Magnetic Dipole}

Based on Rybicki \& Lightman Problem 3.1.

A pulsar is convetionally believed to be a rotating neutron star.  Such
a astar is likely to have a strong magnetic field, $B_0$, since it traps
lines of force during its collapse.  If the magnetic axis of the neutron
star does not line up with the rotation axis, there will be magnetic dipole
radiation from the time-changing magnetic dipole, $m(t)$.  Assume that the mass
and radius of the neutron star ar $M$ and $R$, respectively; that the angle
between the magnetic and rotation axes is $\alpha$; and that the rotational
angular velocity is $\omega$.

\subsection{}
Find an expression for the radiated power $P$ in terms of $\omega$,
$R$, $B_0$, and $\alpha$.

\subsection{}
Assuming that the rotational energy of the pulsar is the ultimate source
of the radiated power, find an expression for the slow-down timescale,
$\tau$, of the pulsar.

\subsection{}
For $M=1 M_\odot$, $R=10^6$ cm, $B_0=10^{12}$ gauss, $\alpha=90^\circ$,
find $P$ and $\tau$ for $\omega=10^4$ s$^{-1}$, $10^3$ s$^{-1}$,
$10^2$ s$^{-1}$.  The highest rate $\omega=10^4$ s$^{-1}$ is believed
to be typical of newly formed pulsars.

Hint: The Larmor power formula gives the power radiated by an accelerating
electric dipole moment. This problem is similar except that it concerns an
accelerating magnetic dipole moment.

\section*{\centering Problem Set 5 (part 2)}

\def\Tex{{T_{\rm ex}}}
\def\Jbar{{\bar J}}
\def\qot{{q_{12}}}
\def\qto{{q_{21}}}
\def\Ato{{A_{21}}}
\section{21 (Flavors of) Temperatures for 21cm Radiation}

This problem revisits the famous hyperfine transition in neutral hydrogen. Here
we try to understand what sets the excitation (spin) temperature, $\Tex$. (In a
previous problem set, we merely contented ourselves with the statement that 
$\Tex\gg T_*$.  While we're at it, we learn more astronomer's jargon.

In general, the excitation temperature of a transition is influenced by two
factors: (1) the radiation field, and (2) collisions with surrounding species.

The radiation field can either excite the atom through photon absorption, or
de-excite through stimulated emission. Measure the strength of the ambient
radiation field at 21 cm by $\Jbar$, the mean (i.e., angle-averaged) intensity
integrated over the hyperfine line $\nu$ profile (recall Rybicki \& Lightman chapter
1).

As for collisions, consider here exciting and de-exciting collisions with
fellow neutral hydrogen atoms; Purcell and Field (1956, hereafter PF) conclude
that collisions between a given electronic-ground-state H atom and other
electronic-ground-state H atoms are most important in the predominantly neutral
HI clouds of the ISM. (Electrons are $\sim$42 times faster and tend to dominate
the excitation dynamics in other situations, but assume here that there are too
few of them in these cold clouds.) Denote the collisional excitation rate
coefficient by $\qot$, and the collision de-excitation rate coefficient by
$\qto$ Assume for this problem that both hyperfine-excited and hyperfine-
ground atoms can excite or de-excite the hyperfine level in an atom.
\footnote{ In truth, excitations and de-excitations proceed, as PF describe, by
``spin-exchange" collisions, in which an electron with a certain spin in one H
atom swaps places with an electron having a different spin coming from the
colliding H atom. A given atom can swap its way up to the hyperfine excited
state, or swap its way down to the hyperfine ground state. Here we follow PF
and place the relative probabilities of undergoing a swap-up versus a swap-down
into $\qot$ and $\qto$.}

\def\TK{{T_{\rm K}}}
\def\TR{{T_{\rm R}}}
\subsection{} Write down the equation of global (not detailed!) balance for
this transition. That is, write down the statement that the rate of excitations
(from all possible channels) per volume per time equals the rate of
de-excitations (from all possible channels) per volume per time.

Use only the following variables: $n_1$ and $n_2$ are the number densities of atoms
in the ground and excited states, respectively, $n = n_1 + n_2$, $\TK$ is the kinetic
temperature of the atoms that move according to a Maxwellian, $\qot$, any Einstein
coefficients you want, $\Jbar$,and the statistical weights $g_1$ and $g_2$
of the ground and excited states, respectively. 

What you have written down is an equation for the excitation temperature 
$(\Tex\leftrightarrow n_1/n_2)$
in terms of the radiation field and the rate of collisions. Regard the
latter two as given throughout this problem.

\subsection{} DEFINE a ``radiation temperature," $\TR$ , from $\Jbar$ as
\begin{equation}
\Jbar\equiv B_\nu(\TR)
\end{equation}

Note that we are NOT saying the ambient radiation field is Planckian. We are
merely DEFINING a number $\TR$ by using Planck's function, $B_\nu$, 
where $\nu=1420$ MHz,
the frequency of the 21 cm line.

Re-write your equation in the previous section to solve for $\Tex$ in terms of the following
variables: $\TK$, $\TR$, $T_∗\equiv h\nu/k$ (recall last problem set), and the dimensionless
variable
\begin{equation}
z\equiv\frac{g_1 n\qot T_*}{g_2 A_{21}\TK}
\end{equation}

where $\Ato=2.85\cdot10^{-15}~{\rm s}^{-1}$ is the Einstein decay coefficient.
Use the very likely condition that $\TK$ , $\TR\gg T_*$ to rid your equation of all
exponentials.

Verify that if $z\gg1$, $\Tex\approx\TK$ (collisions beat radiation; the transition is in
LTE at $\TK$), but that if $z\ll1$, $\Tex\approx\TR$ (radiation beats collisions; the
transition is not in LTE at $\TK$).

\subsection{} To order of magnitude (actually much better than that), what
fraction of HI is in the excited hyperfine state? Recall that $g_1 = 1$ and $g_2 = 3$
and use the very likely condition that $\TK$, $\TR\gg T_*$.

\def\nuPF{{\nu_{\rm PF}}}
\subsection{} Estimate the value for $z$ for an HI cloud at $\TK = 100K$, $n =
1 {\rm cm}^{-3}$. Use Table 1 and equation (9) of PF; note that PF's collision
frequency $\nu=n\langle\sigma v\rangle$ is not the same as our line frequency $\nu$ ; 
call PF's $\nu=\nuPF$; then $n\qot=3\nuPF/8$.  (For those interested, the 3/8 can be understood
easily; skim the first 3 pages of PF and use your answer from the previous section). 

Based on your answer, would you expect collisions or radiation to be more
important in determining the degree of excitation?

\def\ncrit{{n_{crit}}}
\subsection{} ``Critical densities," $\ncrit$, for exciting the line by collisions are 
defined by setting the rate of spontaneous decays equal to the rate of collisional 
de-excitations. Show that such a procedure gives
\begin{equation}
\ncrit=\frac{\Ato}{\qto}
\end{equation}
and solve for its value for this line at $\TK = 100$K.

One can define $\ncrit$ for any line transition at any temperature; it is a
crude gauge of the density of colliders required for collisions to be important
in setting the level populations.

\subsection{} Suppose radio observations are made that spatially resolve
emission from a uniform HI cloud that is optically thick to its own 21 cm line
radiation. Prove that the observed specific intensity, $I_\nu$, equals 
\begin{equation}
I_\nu=\frac{2k\Tex}{\lambda^2}.
\end{equation}

\end{document}
