\documentclass[11pt]{article}
\usepackage[top=1in, bottom=1in, left=1in, right=1in]{geometry}
\usepackage{amsmath}
\usepackage{amssymb}
\usepackage{graphicx}
\usepackage{titlesec}
\titleformat{\subsection}[runin]{\normalfont\large\bfseries}{\thesubsection}{1em}{}

\begin{document}
\pagestyle{empty}
%\parindent=0pt

\def\nufs{{\nu_{fs}}}
\section*{\centering Problem Set 12}

\section{Flipping Spins at the Epoch of Reionization}

Observations of the 21 cm line at redshifts of $z\ge10$ could probe the
distribution of neutral hydrogen at the epoch of reionization, providing new
insight into astrophysics and cosmology. This is the goal of a new generation
of radio interferometers e.g., the Murchison Widefield Array (MWA), the LOw
Frequency ARray (LOFAR), the Precision Array to Probe the Epoch of Reionization
(PAPER), the 21 cm Array (21CMA), and the Giant Meterwave Radio Telescope
(GMRT), along with next generation instruments like the Square Kilometer Array
(SKA).

\begin{figure}[!ht]
\includegraphics[height=3in]{spintemp.png}
\caption{
Schematic picture of the predicted cosmological 21 cm signature through the
epoch of reionizaiton. From Prichard and Loeb (2012).
}\label{fig:spintemp}
\end{figure}

Due to cosmological redshift, the 21 cm signal at different epochs will map to
different wavelengths; hence we may be able to read off the evolution of
hydrogen gas from a spectrum. Figure \ref{fig:spintemp}, from the nice review article of
Prichard and Loeb (2012) , shows what the spectrum might look like. At
different redshifts, the 21cm line may be seen in either emission or absorption
relative to the background radiation source, which is generally the CMB. Here
we consider the basic physics of the 21 cm fluctuations, and familiarize
ourselves with the terminology used in the literature. In particular, the 21 cm
community is fond of defining a great variety of ``temperatures".

First, define the constant $T_*=h\nufs/k=0.068 K$, where $\nufs$ is the
frequency of the 21 cm line. We will be in the regime $T\gg T_*$ for any reasonable
temperature we may encounter, and are thus in the Raleigh-Jeans limit. In this
case, people describe the observed specific intensity using the brightness
temperature, $T_b$, defined by
\begin{equation}
I_\nu=\frac{2kT_b}{\lambda^2}=\frac{2\nufs^2}{c^2}kT_b
\end{equation}
The value $T_b$ may or may not have anything to do with the actual kinetic
temperature, $T_K$ of the gas being observed. As defined, it is merely an
alternative way of expressing $I_\nu$.

We define another temperature, the spin temperature, $T_s$, which describes the
hyperfine level populations\footnote{To be consistent with Prichard and Loeb,
I'll use a subscript 1 to describe the excited ($F=1$) hyperfine state of 
hydrogen and subscript 0 for the ground state ($F=0$). For other transitions, the
spin temperature is usually called the excitation temperature.}
\begin{equation}
\frac{n_1}{n_0}=\frac{g_1}{g_0}e^{-h\nufs/kT_s}
\end{equation}
When the gas is in LTE we have the identification $T_s = T_K$; i.e., the spin
temperature is the same as the actual gas temperature. If LTE does not hold, $T_s$
does not correspond to any real thermodynamic temperature, and is simply a
convenient parameter to describe the ratio $n_1/n_0$.

\def\Tg{{T_\gamma}}
The hyperfine level populations will be influenced by radiative transitions. We
define the radiation temperature, $\Tg$, in terms of the local mean intensity of
the radiation field, $J_\nu$
\begin{equation}
J_\nu(\nufs)=B_\nu(\Tg,\nufs)
\end{equation}
This definition does not necessarily assume the radiation field is a blackbody
--- we are simply defining $\Tg$ as the temperature for which the Planck function
equals $J_\nu$ at the frequency $\nufs$. In our case, the radiation field is due
(primarily) to the CMB, which actually is a blackbody with $\Tg=2.7(1+z)$.

\def\dTb{{\delta T_b}}
\subsection{}
Consider a specific intensity beam from the background CMB passing through a
hydrogen cloud of optical depth $\tau$. Show that the observed fluctuation in
the brightness temperature of the 21 cm line (relative to the background CMB
brightness temperature) is given by\footnote{We won’t gone into the effects of
cosmological redshift, which reduce the overall intensity of the fluctuation.
If included, the right hand side of this expression should be divided by a
factor of $(1 + z)$.}
\begin{equation}
\dTb=(T_s-\Tg)(1-e^{-\tau})
\end{equation}
If we can determine $T_s$, we can predict whether the 21 cm line should be seen
in emission $(\dTb > 0)$ or absorption $(\dTb < 0)$. Calculating the spin
temperature, however, is difficult because it is affected by various
astrophysical processes.

\subsection{}
The hyperfine levels will generally be in statistical equilibrium (not
necessarily LTE) in which the spin flip transitions are in steady state. Assume
that transition between the levels are due to either collisions (e.g., impacts
with other hydrogen atoms, see $q_{10}$ below) or radiation. Show using the Einstein coefficients
that the expression for the spin temperature\footnote{Recall $T\gg T_*$ for any $T$} is
\begin{equation}
T_s^{-1}=\frac{\Tg^{-1}+x_cT_K^{-1}}{1+x_c}
\end{equation}
where
\begin{equation}
x_c=\frac{q_{10}}{A_{10}}\frac{T_*}{\Tg}.
\end{equation}
Determine the value of $T_s$ in the two limits $x_c\ll1$ and $x_c\gg1$ and
briefly explain why these limits makes sense.

\subsection{}
\label{sec:subc}
There is a critical density\footnote{This is a slightly different critical
density than we discussed in class, in which we only looked at the ratio of
collisional to spontaneous de-excitation $q_{10}/A_{10}$.  We see that taking into
account the transitions driven by the incident radiation field introduces the
additional factor $T_*/\Tg$.} where $x_c\sim1$. As a rough estimate of its value,
consider the case where $q_{10}$ is due to neutral hydrogen atoms colliding with
each other, and assume the cross-section for collisional de-excitation is just
the geometrical cross section of an
H atom. Determine the critical hydrogen density for the specific conditions 
$\Tg=2.7K$ and $T_K=100K$.

There is another important effect that can ``flip the spin" of a hydrogen atom
--- the scattering of Ly-$\alpha$ photons\footnote{ Lyman alpha photons will be
pro- duced in abundance once stars/AGN form and produce HII regions.  }. A
Ly-$\alpha$ photon can excite a hydrogen atom in either of the two $n = 1$
hyperfine states to an $n = 2$ level.  The subsequent emission of a Ly-$\alpha$
photon can return the electron to either of the two $n = 1$ hyperfine states,
effectively causing a transition between these levels. This is known as the
Wouthuysen-Field effect.

The Ly-$\alpha$ line is so optically thick that we expect the radiation field
near line center to be coupled to the gas temperature and reach its LTE
value\footnote{The coupling of the Ly-$\alpha$ photons and the gas thermal energy pool is
subtle, but related to the energy exchange that occurs when an atom recoils
upon scattering a Ly-$\alpha$ photon. Though the energy exchange of any one scattering
is small, many repeated scatterings can effectively couple the Ly-$\alpha$ radiation to
the gas temperature. This is similar to the comptonization we discussed, in
which radiation exchanges energy with the gas through electron scattering.}
\def\nuly{{\nu_{Ly\alpha}}}
\begin{equation}
J_\nu(\nuly)=B_\nu(T_K,\nuly)
\end{equation}
Let $P_{10}$ be the rate at which Ly-$\alpha$ photons drive transitions from the ground to
the excited hyperfine level. From LTE arguments we would also then expect that
the total rate of transitions from 0 to 1 is
\begin{equation}
P_{01}=P_{10}\frac{g_1}{g_0}e^{-h\nufs/kT_K}
\end{equation}

\subsection{}
Show that a more general expression for the spin temperature
which includes the Wouthuysen-Field effect (and assumes the 
Ly-$\alpha$ radiation field is well-coupled to the gas temperature) is
\begin{equation}
T_s^{-1}=\frac{\Tg^{-1}+(x_c+x_\alpha)T_K^{-1}}{1+x_c+x_\alpha}
\end{equation}
and write down the expression for $x_\alpha$ in terms of $P_{10}$.

\subsection{}
Now we can make qualitative predictions of the kind of signal we expect from
cosmological 21 cm measurements. For each epoch below, give a very brief
explanation as to why you predict we should see the 21 cm fluctuations
(relative to the CMB) in emission, in absorption, or not at all.
\begin{enumerate}

\item $(200\le z\le 1100)$: After recombination at $z\approx1100$, the gas density is
high enough that the gas and CMB radiation remain thermally coupled10 and have
the same temperature. There are as yet no sources of Ly-$\alpha$ or ionizing
photons.

\item $(40\le z\le 200)$: As cosmological expansion continues, the gas and
radiation go out of equilibrium and their temperatures evolve independently and
adiabatically. The gas density is still well above the critical density defined
in \ref{sec:subc}, though.

\item $(30\le z\le 40)$: The gas density drops below the critical density such
that radiative transitions from the CMB set the level populations.

\item $(15\le z\le 30)$: The first sources (stars, AGN) turn on, and produce
enough Ly-$\alpha$ photons that the hyperfine level populations are set by the
Wouthuysen-Field effect. The gas is still cool from adiabatic expansion, so 
$T_K < \Tg$.

\item $(7\le z\le 15)$: The radiation (mostly x-rays) from sources heat the gas
to the point that $T_K$ becomes larger than the CMB temperature. Lyman alpha
coupling is still effective (i.e., $x_\alpha\gg1$).

\item $(z\le 7)$: Enough ionizing radiation from the sources has been emitted
that reionization is complete --- i.e., essentially all of the neutral hydrogen
in the intergalactic medium has been ionized.

\end{enumerate}

Comment: We can now better understand the predicted signal drawn in Figure \ref{fig:spintemp}.
The above run down is of course just a guess at what the 21 signature should
look like, and at what redshifts we might expect transitions. We still have a
lot to learn about cosmological reionization and the sources of radiation
that influence the intergalactic medium. Having actual 21 cm observations at
these epochs obviously would teach us a lot.

\end{document}
