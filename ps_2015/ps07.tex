\documentclass[11pt]{article}
\usepackage[top=1in, bottom=1in, left=1in, right=1in]{geometry}
\usepackage{amsmath}
\usepackage{amssymb}
\usepackage{graphicx}
\usepackage{titlesec}
\titleformat{\subsection}[runin]{\normalfont\large\bfseries}{\thesubsection}{1em}{}

\def\Te{{T_e}}
\def\Tp{{T_p}}
\def\tep{{t_{ep}}}
\def\trec{{t_{rec}}}
\begin{document}
\pagestyle{empty}
%\parindent=0pt

\section*{\centering Problem Set 7}

\section{Great Balls of (Relativistic) Fire}

Based on Rybicki \& Lightman 4.1.

In astronomy, it is frequently argued that a source of radiation that
undergoes a fluctuation of duration $\Delta t$ must have a physical
diameter of order $D\lesssim c\Delta t$.  This argument is based on the 
fact that even if all portions of the source undergo a disturbance at the same instant
and for an infinitesimal period of time, the resulting signal at the
observer will be smeared out over a time interval of $\Delta t_{min}\sim D/c$
because of the finite light travel time across the source.  Suppose, however,
that the source is an optically thick spherical shell of radius $R(t)$ that
is expanding with relativistic velocity $\beta\sim1$,$\gamma\gg1$ and energized
by a stationary point at its center.  
By consideration of relativistic beaming effects,
show that if the observer sees a fluctuation from the shell of duration $\Delta t$
at time $t$, the source may actually be of radius
\begin{equation}
R<2\gamma^2c\Delta t
\end{equation}
rather than the much smaller limit given by the nonrelativistic considerations.
In the rest frame of the shell surface, each surface element may be treated
as an isotropic emitter.

This latter argument has been used to show that the active regions
in quasars may be much larger than $c\Delta t\sim1$ light month across,
and thus avoids much energy being crammed into so small a volume.

\section{The Blob}

Based on Rybicki \& Lightman 4.7.

An object emits a blob of material at speed $v$ at an angle $\theta$ to
the line-of-sight of a distant observer.

\subsection{}
Show that the apparent transverse velocity inferred by the observer
(i.e. the angular velocity on the sky times the distance to the object) is
\begin{equation}
v_{app}=\frac{v\sin\theta}{1-(v/c)\cos\theta}
\end{equation}

\subsection{}
Show that $v_{app}$ can exceed $c$; find the angle for which $v_{app}$ is maximum,
and show that this maximum is $v_{max}=\gamma v$.

\subsection{}
Plot $v_{app}/c$ versus $\theta$ for $\gamma=10^2$.  Does the viewing angle $\theta$
need to be especially small for superluminal motion to be perceived?

\section{Powering Radio Lobes}

\begin{figure}[!ht]\centering
\includegraphics[height=3in]{cygnus_a.png}
\caption{
Radio image of Cygnus A, showing the extended lobes powered by narrow jets from
the central supermassive black hole. The lobes are about 30 kpc in radius,
and the emission is strongest at the front regions which are presumably
interacting most strongly with the ambient medium.
}\label{fig:cygnus_a}
\end{figure}

The galaxy Cygnus A is one of the most powerful radio sources in the sky.
Striking radio images (Figure \ref{fig:cygnus_a}) reveal a pair of immense lobes of emission,
which sit about $\sim$100 kpc outside the central galaxy. It is thought that these
lobes are powered by a super-massive black hole (SMBH) at the galactic center.
The SMBH somehow produces narrow, bipolar, relativistic jets, which propagate
to the outskirts of the galaxy, interact with the ambient medium, and form
shocks. The radio luminosity is the result of synchrotron emission from
relativistic electrons produced in those shocks.

\begin{figure}[!ht]\centering
\includegraphics[height=3in]{baars_et_al.png}
\caption{
Radio spectrum of Cygnus A (and other objects) from Baars et al. (1977)
}\label{fig:baars_et_al}
\end{figure}

Figure \ref{fig:baars_et_al} shows the observed radio spectrum of Cyg A, which
resembles what we might expect from synchrotron emission --- broad-band, roughly
power-law, clearly non-thermal. The observed flux increases down to a frequency
of at least 10 MHz, where the value is
$F_\nu\approx10^4 Jy$.  In this region, one can reasonably fit a power law to
the spectrum and write the specific luminosity
\begin{equation}
L_\nu=4\pi d^2F_\nu=5\times 10^{36}\left(\frac{\nu}{10~{\rm MHz}}\right)^{-0.8} {\rm erg~s}^{-1}{\rm Hz}^{-1}
\label{eq:cyg_luminosity}
\end{equation}
where we have taken the distance to Cyg A to be $d\approx230~{\rm Mpc}$.  Integrating over
equation \ref{eq:cyg_luminosity} over the observed range $10^7$ to $10^{11}$ Hz gives a luminosity
of $L\sim10^{-45}{\rm erg~s}^{-1}$ (about 1000 times the energy radiated by a supernova at peak).

What are the energetics and magnetic field strengths involved in these luminous
radio lobes, and can a SMBH really provide the necessary oomph? The total
energy\footnote{We are going to ignore the energy in protons, since there is no 
easy way to measure them directly. But presumably there is as much or more 
energy in protons than electrons}
in the lobes can be written
\begin{equation}
E=(U_e+U_B)\times 2V
\end{equation}
where $U_e$,$U_B$ are the electron and magnetic field energy densities, respectively,
and $V$ is the volume of a lobe\footnote{The lobes each have a radius of around 30 kpc. 
We have included a factor of 2 in the equation since there are two of them.}
As it turns out, we do not have enough information to directly determine these energy densities, but 
we can follow the famous arguments of Geoffery Burbidge to derive a lower limit. The implied 
energies, we will find, are massive --- the equivalent of a billion supernova explosions or more.
As usual, we’ll take the electron number density to be described by a power law distribution in 
Lorentz factor
\begin{equation}
n~d\gamma=C\gamma^p~d\gamma
\label{eq:gamma_plaw}
\end{equation}
where $C$ is some constant, and the power law cuts off at some lower value, $\gamma_{\rm min}$.

\subsection{}
Infer the value of $p$ from the observations.

\subsection{}
Integrate equation \ref{eq:gamma_plaw} to get an expression for the energy density of
electrons, $U_e$, in terms of $C$ and $\gamma_{\rm min}$.

What should we use for $\gamma_{\rm min}$ and $C$?  It looks like the observed spectrum
might be peaking at $v_m\approx 10~{\rm MHz}$.  We can therefore associate\footnote{
This may or may not be correct; the data are getting a little sketchy around 10 MHz,
owing to the growing opacity of the ionosphere at these frequencies.  It could also
be that the turnover is real, but due to synchrotron self-absorption, not
the minimum Lorentz factor.  If the $\nu_m$ is, in fact, smaller than 10 MHz, this would
imply an even smaller $\gamma_{\rm min}$, and hence even more energy.  Thus, what
follows is at least a lower limit to the energy.}
10 MHz with the critical frequency of electrons with Lorentz factor $\gamma_{\rm min}$.  To
determine the constant $C$, we can use the observed specific luminosity at the frequency
$\nu_m=10~{\rm MHz}$.  From synchrotron theory, we know that the specific luminosity
(ergs s$^{-1}$ Hz$^{-1}$) from a power-law distribution of electrons is
\begin{equation}
L_\nu\approx\frac{2C}3\frac{U_B\sigma_Tc}{\nu_{cyc}}\left(\frac{\nu}{\nu_{cyc}}\right)^\frac{1+p}2\times V
\end{equation}

\subsection{}
Show that the energy density of electrons can be written
\begin{equation}
U_e=A\frac{L_m\nu_m^\frac12}{V}B^{-\frac32}
\label{eq:e_energy_density}
\end{equation}
where $L_m$ is the observed specific luminosity at $\nu_m$, and $A$ is some
combination of numerical factors and fundamental constants.

To proceed, we need to know the magnetic field strength. Unfortunately, there
is no easy way to measure this directly. What is typically done instead is to
make a minimum energy argument. The magnetic energy density increases with $B$,
whereas equation \ref{eq:e_energy_density} shows that $U_e$ grows as $B$ decreases. 
Thus, there must be some
value of $B$ that minimizes the total energy. This value will at least
give us a lower limit to the necessary energetics.

\subsection{}

Show that the total energy is minimized when $U_B=\frac34 U_e$ --- that is, when $U_e$ and
$U_b$ are roughly equal. For this reason, the minimum energy argument is often
also called an equipartition argument.

\subsection{}
Adopting the equipartition argument above, what is (numerically) the magnetic
field strength in the emitting regions? What is the minimum Lorentz factor,
$\gamma_{\rm min}$?

\subsection{}
What is the total energy in the radio lobes? How does this compare to the typical energy of a supernova?

\subsection{}
How much mass mass would a black hole have to eat in order to create this
energy? Assume that the rest mass energy of the accreted material is
processed into jet energy with 10\% efficiency. Does the SMBH hypothesis hold
together? --- i.e., is your estimate for the swallowed mass consistent with SMBH
masses?

\subsection*{}

Comment: We have provided no physical justification for why this system should
generate magnetic fields with $U_B\approx U_e$. We have only shown that equipartiation
is the optimal configuration for radiating efficiently. However, if the
magnetic field is produced by turbulent motions in the plasma, which also play
a role in accelerating the high energy particles, perhaps there is some
rationale for thinking that we may reach something close to equipartition. This
conjecture can now be tested by detailed particle-in-cell simulations, which
follow the motions of individual particles in a plasma while simultaneously
solving Maxwell's equations to determine the fields they generate. In the
absence of any better information, people often simply assume an equipartition
$B$-field in order to carry out synchrotron analyses. Note that we have also
neglected here the energy in relativistic protons, which could exceed that in
electrons and magnetic fields.

\end{document}
