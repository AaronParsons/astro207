\documentclass[11pt]{article}
\usepackage[top=1in, bottom=1in, left=1in, right=1in]{geometry}
\usepackage{amsmath}
\usepackage{amssymb}
\usepackage{titlesec}
\titleformat{\subsection}[runin]{\normalfont\large\bfseries}{\thesubsection}{1em}{}

\begin{document}
\pagestyle{empty}
%\parindent=0pt

\section*{\centering Problem Set 8}

\section{A Compton Monte Carlo}

An intuitive and commonly applied method for modeling inverse Compton
emission is to us a Monte Carlo approach, in which the radiation field is
represented by discrete, individual packets of photons. The fate of each photon
packet is determined by ``rolling the dice", i.e., choosing random numbers drawn
from the appropriate probability distributions. Here you will write your own
simple Monte Carlo code, which could in principle be used to model real data.
This is a numerical problem, so please provide a printout of your code, in
addition to your plots.

Consider a supernova that produces optical photons, which then travel through a
shell of shocked circumstellar gas, where they may be scattered by non-thermal,
relativistic electrons. Assume that the supernova emits a luminosity of
$L_s=10^{43}~{\rm erg}/{\rm s}$.  The liminosity carried by each Monte
Carlo packet in our code is then $L_p=L_s/N_p$, where $N_p$ is the number of
photon packets we choose to use.  We will asuume for simplicity that all photons are emitted with
the same frequency $h\nu_{in}=E_{in}=1~{\rm eV}$.  
These photons travel through a shocked shell of relativistic electrons with optical depth
$\tau=0.01$.

The Monte Carlo method proceeds by generating and scattering photon packets one
by one. The energy of a scattered photon is given by applying two Doppler
shifts --- one into the rest frame, and one back out --- which gives a final
outgoing energy
\begin{equation}
E_{out}=E_{in}\gamma^2(1-\beta\cos\theta_{in})(1+\beta\cos\theta^\prime_{out})
\end{equation}
where $\theta_{in}$ is the incoming angle (in the lab frame) and
$\theta^\prime_{out}$ is the outgoing angle (in the rest frame).  We simulate
a scattering by sampling the incoming and outgoing directions randomly. (Since
we are considering $\tau\ll1$, we will only need to scatter each packet once).
Assume, for simplicity, that $\theta_{in}$ and $\theta^\prime_{out}$ are isotropic,
% XXX for next time, these should be uniform in cos theta to mimic dipole response
and generate them randomly.  After a random scattering, the packet has a new luminosity
\begin{equation}
L_{out} = L_p\frac{E_{out}}{E_{in}}(1-e^{-\tau})
\end{equation}
where the term in parenthesis takes into account the fact that only a fraction
$(1-e^{-\tau})$ of the photons from the source are actually scattered.
The packet can be ``observed" by collecting it and binning
its energy into a spectrum array. If we repeat this procedure with 
the $N_p$ packets, we will gradually build up the observed spectrum. 
Because of the random nature of the algorithm, this spectrum will 
possess noise, and we will need a fairly large value for 
$N_p$ to achieve a reasonable signal to noise (try $10^5$ for testing,
$10^6$ for the final result).

\subsection{}
Consider the case where we only have electrons of a single energy, $\gamma=10$. Run
the Monte Carlo procedure and plot the spectrum of scattered photons on a
log-log plot. Your x-axis can be either eV or Hz, and y-axis should be a
monochromatic luminosity, i.e., units of ergs s$^{-1}$ eV$^{-1}$ or ergs s$^{-1}$ Hz$^{-1}$. 
Check that where the spectrum cuts off at
high energy makes sense.

\subsection{}
Now consider the more realistic case where the electrons have a power law distribution in energy, between the bounds
$\gamma_{min}=10$ and $\gamma_{max}=100$.  The probability, in any given scattering, that the
electron has an energy $\gamma$ is
\begin{equation}
P=A\gamma^{-p}
\end{equation}
for $\gamma_{min}<\gamma<\gamma_{max}$, and 0 otherwise.  $A$ is a constant, and we'll take
$p=2.5$.  For each scattering, we should then choose the $\gamma$ of the electron 
by randomly sampling from this probability distribution (which can be easily
done\footnote{http://mathworld.wolfram.com/RandomNumber.html}). Run 
the Monte Carlo for this case and plot the spectrum due to inverse
Compton scattering. 
Overplot the slope of the power law that you expect from analytic
arguments.

By the way, your Monte Carlo code could probably already be used in a simple
research paper. For example, you could model the x-ray emission from SN 2011fe
and constrain the density of the circumstellar environment, as considered in
Horesh et al. (2012). To improve the code, you would want to relax the
assumption that all photons start with the same energy $E_{in}$. Instead, you can
imagine randomly sampling the initial frequency of the photons from a real
distribution (e.g., a blackbody). In addition, you could fairly easily
generalize your code to treat multiple scatterings, or for the more realistic
case where the scattering is not isotropic or inelastic. If you also kept track
of the position of the photons, you could even consider the case where the
scattering cloud is not spherically symmetric.

\section{Compton Saturation}

Based on Rybicki \& Lightman, Problem 7.1.

A cloud of non-relativistic electrons is maintained at a temperature, $T$.  The cloud is thick to electron
scattering, $\tau_{es}\gg1$, but very thin to absorption, $\tau_*(h\nu=kT)\ll1$.  A copious
supply of ``soft" photons, each of characteristic energy $E_i\ll kT$ is injected into the cloud.
As a result of inverse Compton scattering, these initially soft photons emerge from the cloud 
with characteristic energies $E_f\gg E_i$.  It is found that $E_f$ increases rapidly with
increasing $\tau_{es}$ as the latter is varied, until $\tau_{es}$ reaches a critical value
$\tau_{crit}$, above which the Comptonization process saturates.

\subsection{}
Find an approximate expression for $E_f$ as a function of $E_i$, $\tau_{es}$, $T$, and
fundamental constants.

\subsection{}
Find an approximate expression for $\tau_{crit}$.

\subsection{}
Find a single parameter of the fixed medium that determines whether inverse Compton is a significant
effect.

\section{Cold Plasma}

Consider a non-relativistic electron moving at speed $v\ll c$.  A single photon
of original energy $E$ Compton scatters off the electron and suffers an energy shift
of $\Delta E$.  Assume we are in the Thomson limit, as seen in the frame of the electron.

\subsection{}\label{sec:parta}
Estimate a typical value for the fraction change in photon energy, $\Delta E/E$.  Express your
answer symbolically.  By ``typical," we mean any old photon that comes from any old direction.

\subsection{}\label{sec:partb}
Consider this same electron flying through a sea of photons all of energy $E$.  Many photons
get inverse Compton scattered to and fro.

Averaged over all the photons that got scattered, what is the mean fractional change
in phtoton energy, $\langle \Delta E/E\rangle$?  You may use our expression for the
inverse Compton power scattered by a single electron if you wish.  Neglect the change in the electron's velocity
as it is bombarded by photons.

\subsection{}
Explain physically why parts \ref{sec:parta} and \ref{sec:partb} give different answers.

\end{document}
