\documentclass[11pt]{article}
\usepackage{fullpage}

\begin{document}
\pagestyle{empty}
\parindent=0pt

\section*{\centering Problem Set 2 (part 1)}

\section{Brightnesses, Magnitudes, and Photons}

In the old days, astronomers classified the brightness of celestial
objects using the magnitude system, which roughly corresponded to the
sensitivity of the human eye. Today we continue to use this ancient system
because\dots well, I don't know, but we do. The magnitude of
an object in a given filter $i$ is 
\begin{equation}
m_i = −2.5 \log_{10}(F_i/F_{0,i}) + m_{0,i}
\end{equation}
where $F_i$ is the observed flux integrated over a given filter
\begin{equation}
F_i = \int_0^\infty{F_\nu(\nu)\phi(\nu)~d\nu}
\end{equation}
where $\phi(\nu)$ is the filter transmission function (a number
between 0 and 1).  
The magnitude system is calibrated by specifying $F_{0,i}$ and $m_{0,i}$ which 
may be done in a number of ways. A common system is
Vega-magnitudes, in which the star Vega is defined to have $m\approx0$ in all bands.

\subsection{}
Grab the filter transmission functions and the spectrum of Vega 
(bessel\_V.dat and vega\_spectrum.dat in the github repository, noting that
the Vega spectrum is given as $F_\lambda$, with units ergs s$^{-1}$ cm$^{-2}$ \AA$^{-1}$).
numerically integrate the apparent V-band magnitude of Vega to determine how many
photons the world's largest ground-based optical telescope (Keck, with a mirror diameter of 10m)
would collect per second from Vega.  

\subsection{}
Use the fact that Vega is about 8 pc away, and has a diameter of approximately 2.5 $R_\odot$, to
determine the specific intensity of Vega in cgs units, assuming Vega appears as a uniform disk on the sky.

\subsection{}
If we were twice as close to Vega (say, 4 pc), figure how many photons per second Keck would receive
in V band, and re-determine the specific intensity of Vega.  Are the scalings of photons per second and
specific intensity with distance consistent with one another?  How so?

\subsection{}
Just as Vega is one of the brightest things in the optical sky, Cygnus A, an incredibly bright
pair of jets emanating from an active galactic nucleus, is
one of the brightest in the radio sky. At 150 MHz, Cygnus A is approximately 10 kJy.
How many photons would the world's largest radio telescope (Arecibo, with a diameter of 300m)
collect per second from Cygnus A, integrating from 155 to 165 MHz.  Following the scaling of
photons received by the biggest telescope on the planet from the brightest things on the sky as
a function of frequency predicted from Arecibo observing Cygnus A and Keck observing Vega, at
what frequency might you expect to worry about shot noise from individual photons for even observations
of bright sources.  Which telescope might this correspond to?

\section*{\centering Problem Set 2 (part 2)}

\section{Make Like a Tree}

If all the leaves of a tree fall to the ground, how thick is the layer of leaves on the ground?
Give your answer in leaf thicknesses, to order-of-magnitude.

What is the order-of-magnitude of the optical depth presented by the leafy tree to
someone lying below the tree? Why does this answer make sense from the perspective
of the vegetation? (Be the tree.)
These are really flip sides of the same question. You can try the second
paragraph before the first, or vice versa.

\section{Dust Bowl}

Dust particles are about 100 microns in diameter.  In the years of the Dust
Bowl in the 1930s, it was not uncommon for dust clouds to completely obscure
the Sun.  According to Wikipedia:
\begin{quote}
On April 14, 1935, known as ``Black Sunday", 20 of the worst ``black
blizzards" occurred across the entire sweep of the Great Plains, from Canada
south to Texas. The dust storms caused extensive damage and turned the day to
night; witnesses reported they could not see five feet in front of them at
certain points. 
\end{quote}

\subsection{}
Using this observation, estimate the number density of dust particles in the air.
We'll do a more sophisticated
treatment of dust grains later in the class, but for now, assume these dust grains to be perfectly
opaque, and large enough to be treated geometrically (without diffraction, etc.).  

\subsection{}
If these clouds extended half a mile into the air, how much topsoil was lost to each one of these clouds?


\end{document}
