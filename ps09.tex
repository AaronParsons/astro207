\documentclass[11pt]{article}
\usepackage[top=1in, bottom=1in, left=1in, right=1in]{geometry}
\usepackage{amsmath}
\usepackage{amssymb}
\usepackage{titlesec}
\titleformat{\subsection}[runin]{\normalfont\large\bfseries}{\thesubsection}{1em}{}

\begin{document}
\pagestyle{empty}
%\parindent=0pt

\section*{\centering Problem Set 9 (part 1)}

\section{Great Balls of (Relativistic) Fire}

Based on Rybicki \& Lightman 4.1.

In astronomy, it is frequently argued that a source of radiation that
undergoes a fluctuation of duration $\Delta t$ must have a physical
diameter of order $D\lesssim c\Delta t$.  This argument is based on the 
fact that even if all portions of the source undergo a disturbance at the same instant
and for an infinitesimal period of time, the resulting signal at the
observer will be smeared out over a tie interval of $\Delta t_{min}\sim D/c$
because of the finite light travel time across the source.  Suppose, however,
that the source is an optically thick spherical shell of radius $R(t)$ that
is expanding with relativistic velocity $\beta\sim1$,$\gamma\gg1$ and energized
by a stationary point at its center.  
By consideration of relativistic beaming effects,
show that if the observer sees a fluctaion from the shell of duration $\Delta t$
at time $t$, the source may actually be of radius
\begin{equation}
R<2\gamma^2c\Delta t
\end{equation}
rather than the much smaller limit given by the nonrelativistic considerations.
In the rest frame of the shell surface, each surface element may be treated
as an isotropic emitter.

This latter argument has been used to show that the active regions
in quasars may be much larger than $c\Delta t\sim1$ light month across,
and thus avoids much energy being crammed into so small a volume.

\section{The Blob}

Based on Rybicki \& Lightman 4.7.

An object emits a blob of material at speed $v$ at an angle $\theta$ to
the line-of-sight of a distant observer.

\subsection{}
Show that the apparent tansverse velocity inferred by the observer
(i.e. the angular velocity on the sky times the distance to the object) is
\begin{equation}
v_{app}=\frac{v\sin\theta}{1-(v/c)\cos\theta}
\end{equation}

\subsection{}
Show that $v_{app}$ can exceed $c$; find the angle for which $v_{app}$ is maximum,
and show that this maximum is $v_{max}=\gamma v$.

\subsection{}
Plot $v_{app}/c$ versus $\theta$ for $\gamma=10^2$.  Does the viewing angle $\theta$
need to be especially small for superluminal motion to be perceived?

\section*{Problem Set 9 (part 2)}

\section{Synchrotron Losses}

\subsection{}
Obtain an analytic expression for the energy of a single relativistic
electron as a function of time, $E(t)$, taking into account its energy loss by
synchrotron radiation. Your expression should contain only the variables $E(0)$
(the initial energy of the electron), $B$ (the magnetic field, here held fixeId
with time, following the rest of the world, though one should worry in general
about the field changing with time just as the electron energy spectrum changes
with time), time $t$, and fundamental constants. Assume $\sin\alpha=1$ (the electron
pitch angle is 90 degrees) for simplicity.

For (synchrotron) problems of interest to us, the electron always remains
relativistic. It merely evolves from a large $\gamma\gg1$ to a smaller $\gamma\gg1$.

\subsection{}
How can you reconcile the loss of energy of the electron with the bald
statement of Rybicki \& Lightman on page 168 that 
\begin{equation}
\frac{d}{dt}(\gamma mc^2)=q\vec v\cdot \vec E=0
\end{equation}
implies``$\gamma$ is constant"?

\subsection{}\label{partc}
We have made arguments in class that power-law distributions of electrons in
astrophysical sources are maintained against synchrotron losses by continuous
energization by central engines (a.k.a. injection). The injection (input)
spectrum of electrons is modified by synchrotron losses to produce a
steady-state (output) distribution.

Call $\eta(E,t)=dN/dE$ the differential energy spectrum of electrons as
discussed repeatedly in class. Continuity of electrons in energy space reads
\begin{equation}
\frac{\partial\eta(E,t)}{\partial t}+\frac{\partial}{\partial E}[\dot E\eta(E,t)]=I
\end{equation}
where $I$ is the rate of injection of electrons with some input distribution
and $\dot E$ is the rate of energy loss of a single electron by synchrotron
radiation. This equation should not mystify you; it merely describes how the
number of electrons in a given energy
bin changes with time, taking into account a flux divergence (the second term
on the left-hand-side) and a source term (the right-hand-side).

We have assumed in class a steady-state distribution of electron energies for which
$\eta\propto E^p$.  Given $p$, how must $I$ scale with $E$?  Give on the scaling and
forget about the numerical coefficients.

As with most scaling problems, you don’t have to solve anything in detail. Ruthlessly work to order of magnitude.

\subsection{}\label{partd}
Electrons having a given energy must wait a characteristic time 
before synchrotron losses become important. Before this time elapses 
for all such electrons, how does $\eta$ scale with $E$?

\subsection{}
The spectral index ($\alpha = d\ln F_\nu/d\ln\nu$) of radiation in a fixed frequency
range from a radio jet flattens with increasing distance from the central
galaxy. That is, $\alpha = -0.5$ at the remote edge of the jet (the ``hot spot"), and $\alpha=-1$
closer in. Given your understanding in \ref{partc} and \ref{partd}, where are the
``freshest" electrons located, i.e., those newly injected into the energy
spectrum? Are they at the end of the jet, or are they closer in? In other
words, where is the principal site of particle acceleration?

\end{document}
