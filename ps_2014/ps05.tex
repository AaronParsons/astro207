\documentclass[11pt]{article}
\usepackage[top=1in, bottom=1in, left=1in, right=1in]{geometry}
\usepackage{amsmath}
\usepackage{amssymb}
\usepackage{graphicx}
\usepackage{titlesec}
\titleformat{\subsection}[runin]{\normalfont\large\bfseries}{\thesubsection}{1em}{}

\def\Te{{T_e}}
\def\Tp{{T_p}}
\def\tep{{t_{ep}}}
\def\trec{{t_{rec}}}
\begin{document}
\pagestyle{empty}
%\parindent=0pt

\section*{\centering Problem Set 5}

\section{Blowing Str\"omgren Bubbles}\label{p2}

Consider a lone O star emitting $\eta$ Lyman limit photons per second. It sits
inside hydrogen gas of infinite extent and of number density n. The star
ionizes an HII region--a.k.a. a ``Str\"omgren sphere," after Bengt Str\"omgren, who
understood that such spheres have sharp boundaries inside of which hydrogen is nearly
completely ionized.

Every Lyman limit photon goes towards ionizing a neutral hydrogen atom. That
is, every photon emitted by the star goes towards maintaining the Str\"omgren
bubble. Put yet another way, no Lyman limit photon emitted by the star travels
past the radius of the Str\"omgren sphere.

The rate at which Lyman limit photons are emitted by the central star equals
the rate of radiative recombinations in the ionized gas. The sphere is nearly
completely ionized\footnote{The
sphere cannot be 100\% ionized because then there would be no neutrals to
absorb the Lyman limit photons that are continuously streaming out of the star.}.
These facts of photo-ionization equilibrium determine the
approximate radius of the Str\"omgren sphere.

The temperature inside the sphere is about 10000 K. (A class on the interstellar medium can show you why.)

\subsection{} OPTIONAL

Derive a symbolic expression for the radius of the Str\"omgren sphere using the
above variables and whatever variables were introduced in lecture.

\subsection{}\label{p2partb}

What is the timescale, $\trec$, over which a free proton radiatively recombines in
the sphere? That is, how long would a free proton have to wait before
undergoing a radiative recombination? Give both a symbolic expression, and a
numerical evaluation for $n = 1~{\rm cm}^{-3}$.

\subsection{}

If the star were initially ``off," and the gas surrounding it initially neutral,
what is the timescale for the Str\"omgren sphere to develop after the star were
turned ``on"? That is, how long does the star take to blow an ionized bubble?
Think simply and to order-of-magnitude; you should get the same answer as \ref{p2partb}.

\section{Time to Relax in the Str\"omgren Sphere}

It is often assumed that velocity distributions of particles are Maxwellian.
The validity of this assumption rests on the ability of particles to collide
elastically with one another and share their kinetic energy. For a Maxwellian
to be appropriate, the timescale for a collision must be short compared to
other timescales of interest. This problem tests these assumptions for the case
of the Str\"omgren sphere of nearly completely ionized hydrogen of problem \ref{p2}.

\subsection{}\label{p3parta}

Establishing the electron (kinetic) temperature: what is the timescale, $t_e$, for
free electrons in the Str\"omgren sphere to collide with one another? Consider
collisions occurring at relative velocities typical of those in an electron
gas at temperature $\Te$. Work only to order-of-magnitude and express your answer
in terms of $n$, $\Te$, and other fundamental constants.

I think it is fair to say that this problem is done most easily in cgs units.

\subsection{}\label{p3partb}

Establishing the proton (kinetic) temperature: repeat \ref{p3parta}, but for
protons, and consider collisions at relative velocities typical of those in a
proton gas of temperature $\Tp$. Call the proton relaxation time tp.

\subsection{}\label{p3partc}

Establishing a common (kinetic) temperature: suppose that initially, $\Te>\Tp$.
What is the timescale over which electrons and protons equilibrate to a
common kinetic temperature? This is not merely the timescale for a proton to
collide with an electron. You must consider also the amount of energy exchanged
between an electron and proton during each encounter. Estimate, to
order-of-magnitude, the time it takes a cold proton to acquire the same kinetic
energy as a hot electron. Call this time $\tep$. Again, express your answer
symbolically.

Hint: you might find it helpful to switch the charge on the electron and
consider head-on collisions between the positive electron and positive proton.

\subsection{}\label{p3partd}

Numerically evaluate $t_e/\trec$, $t_p/\trec$, and $\tep/\trec$, for $\Te\sim\Tp\sim 10^4$ K
(but $\Te\ne\Tp$ so that $\tep\ne 0$). Is assuming a Maxwellian distribution of
velocities at a common temperature for both electrons and ions a good
approximation in Str\"omgren spheres?


\section{Pulsar Dispersion Measure}

A radio astronomer uses the Green Bank Telescope to observe a pulsar
in a band from 1 to 2 GHz, recording a frequency power spectrum about every 10 ms.  The resulting
(noisy) frequency spectrum is recorded as pulsar.dat (available on the course GitHub repo).  It looks like
noise, but
I promise you that, for the right dispersion measure, there is a pulse in there.

\subsection{}
De-disperse the measured power spectra to determine the
DM of the pulsar.  You may have to try many DMs to find the one that works.
To help you, the beginning of the file is timed to the arrival time without dispersion delay.  For bonus
points, you could make your code not depend on this fact.

Just a warning: be careful about the difference between $\omega$ and $\nu$.  The reported frequencies in
the data file are $\nu$, of course.

\subsection{}
For an assumed density of electrons in the interstellar medium of 0.03 cm$^{-3}$, calculate the
distance to this pulsar. Does this seem reasonable (what scale would you use to judge ``reasonable")?

\subsection{}
For such an assumed electron density, is the radio astronomer safely observing above the plasma cut-off frequency?

\subsection{}
In an earlier problem set (the Eddington limit),
we made use of the Thomson cross-section, $\sigma_T=6.65\cdot10^{-25}~{\rm cm}^2$,
for a photon scattering off an electron. We will discuss Thomson scattering in more detail later, but
for now,
calculate the optical depth to Thomson scattering along the line-of-sight to this pulsar.

\section{Faraday Rotation}

Consider the propagation of light thorugh a magnetized plasma.  The magnetic field is uniform: $\vec B_0=B_0\hat z$.
Light travels parallel to $\hat z$.

An electron in the plasma feels a force from the electromagnetic wave and a force from the externally imposed
$B$-field.  Its equation of motion reads
\begin{equation}
m\dot \vec v=-e\vec E-\frac{e}{c}\vec v\times \vec B_0
\end{equation}
where the electric field $\vec E$ can be decomp[osed into right-circularly polarized (RCP) and
left-circularly-polarized (LCP) waves:
\begin{equation}
\vec E=E_0(\hat x\mp i\hat y)e^{i(k_\mp z-\omega t)}
\end{equation}
where it is understood that the real part of $\vec E$ should be taken.  The upper sign ($-$) corresponds
to RCP waves, while the lower sign ($+$) corresponds to LCP waves.

In the equation of motion above, we have neglected the Lorentz force from the wave's
$B$-field, since it is small (by $v/c$) compared to the force from the wave's $E$-field.

\subsection{}
Prove that the solution of the equation of motion reads
\begin{equation}
\vec v=\frac{-ie}{m(\omega\pm\omega_{cyc})}\vec E
\end{equation}
where $\omega_{cyc}\equiv eB_0/mc$.  This is the hardest part of the derivation for the dispersion
relation of RCP and LCP waves.

\subsection{}
Based on Rybicki \& Lightman Problem 8.3.

The signal from a pulsed, polarized source is measured to have an arrival time delay that varies
with frequency as $dt_p/d\omega=1.1\cdot10^{-5}~{\rm s}^2$, and a Faraday rotation that
varies with frequency as $d\Delta\theta/d\omega=1.9\cdot10^{-4}~{\rm s}$.  The measurements are
made around the frequency $\omega=10^8~{\rm s}^{-1}$, and the source is at unknown distance from
the earth.  Find the mean magnetic field, $\langle B_\parallel\rangle$, in the interstellar space
between the earth and the source:
\begin{equation}
\langle B_\parallel\rangle\equiv\frac{\int{n B_\parallel~ds}}{\int{n~ds}}
\end{equation}

\end{document}
