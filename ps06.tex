\documentclass[11pt]{article}
\usepackage[top=1in, bottom=1in, left=1in, right=1in]{geometry}
\usepackage{amsmath}
\usepackage{amssymb}
\usepackage{titlesec}
\titleformat{\subsection}[runin]{\normalfont\large\bfseries}{\thesubsection}{1em}{}

\begin{document}
\pagestyle{empty}
\parindent=0pt

\section*{\centering Problem Set 6 (part 1)}

\section{Saha and the Redshift of Recombination}

Based on Rybicki \& Lightman 9.4.

The thermal de Broglie wavelength of electrons at temperature $T$ is
defined by $\lambda=h/(2\pi mkT)^{1/2}$.  The degree of degeneracy of
the electrions can be measured by the number of electrons in a cube $\lambda$
on a side:
\begin{equation}
\xi\equiv n_e\lambda^3\approx 4.1\cdot10^{-16}~n_e T^{-\frac32}
\end{equation}
For many cases of physical interest, the electrons are very non-degenerate,
with the quantity $\gamma\equiv{\rm ln}~\xi^{-1}$ being of order 10 to 30.
We want to investigate the consequences of the Blotzmann and Saha equations of
$\gamma$ being large and only weakly dependent on temperature.  For the present
purposes, assume that the partition functions are independent of temperature
and of order unity.

\subsection{}
\label{sec:p1a}
Show that the value of temperature at which the stage of ionization passes from
$j$ to $j+1$ (i.e. when the $j+1$th electron is knocked off) is given approximately by
\begin{equation}
kt\sim\frac\chi\gamma
\end{equation}
where $\chi$ is the ionization potential between stages $j$ and $j+1$.  Therefore,
this temperature is much smaller than the ionization potential expressed in
temperature units.

\subsection{}
The rapidity with which the ionization stage changes is measured by the
temperature range $\Delta T$ over which the ration of populations 
$N_j/N_{j+1}$ changes substantially.  Show that
\begin{equation}
\frac{\Delta T}{T}\sim\left[\frac{d{\rm log}(N_{j+1}/N_j)}{d\rm{log} T}\right]^{-1}\sim\gamma^{-1}
\end{equation}
Therefore, $\Delta T$ is much smaller than $T$ itself, and the change occurs rapidly.

\subsection{}
Using the Boltzmann equation and the result from \S\ref{sec:p1a}, show that when $\gamma$
is large, an atom or ion stays mostly in its ground state before being ionized.

\subsection{}
Calculate the redshift, $z_{rec}$, at which recombination occurred 
in the early universe. Use the temperature-redshift relation
\begin{equation}
T_{\rm photon} = T_0(1+z) 
\end{equation}
and the relation between baryonic number density and redshift
\begin{equation}
n=n_0(1 + z)^3.
\end{equation}

Here $T_0 = 2.73K$ is the temperature of the cosmic microwave photon gas today,
and $n_0 = 10^{-7} cm^{-3}$ is the number density of baryons (read: hydrogen atoms)
today, grossly averaged over the entire universe. You may define the epoch of
recombination to be when 0.5 of the protons have recombined to form neutral
hydrogen.  

(Redshift is a cosmologist’s measure of time. A redshift $z = 0$
corresponds to today. As one goes back in time, the redshift increases. Photons
that have travelled from an epoch corresponding to a redshift $z$ have their
original wavelengths, $\lambda$, stretched to a longer wavelength, $\lambda^\prime$, by the expansion
of the universe. By definition, $\lambda^\prime/\lambda=1+z$.)

\end{document}
